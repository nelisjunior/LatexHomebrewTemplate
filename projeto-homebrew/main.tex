% Documento principal - Compila todos os componentes
\documentclass[
    12pt,                     % tamanho da fonte
    twoside,                  % para impressão em verso e anverso
    openany,                  % capítulos podem começar em qualquer página
    a4paper,                  % tamanho do papel
    brazil,                   % idioma principal
    english,                  % idioma adicional
    french                    % idioma adicional
]{article}

% OPÇÕES DE COMPATIBILIDADE COM OVERLEAF
% Descomente a linha abaixo se estiver tendo problemas no Overleaf
% % Configurações específicas para compilação no Overleaf
% Inclua este arquivo no seu main.tex para resolver problemas de compilação

% Verificar e carregar pacotes necessários para os mapas mentais
\usepackage{tikz}
\usetikzlibrary{mindmap,trees,shadows,arrows,positioning}
\usetikzlibrary{decorations.pathmorphing}
\usetikzlibrary{decorations.markings}
\usetikzlibrary{shapes.geometric}

% Pacotes necessários para manipulação de listas e strings
\usepackage{etoolbox}
\usepackage{xstring}

% Em caso de erro no comando \protected@edef, descomente esta linha:
% \makeatletter\let\protected@edef\edef\makeatother

% Desativação temporária de recursos avançados para verificação
% Se o mapa mental automático estiver causando problemas, descomente esta linha:
% \newcommand{\generateautodocmap}{\textbf{[Mapa mental automático desativado temporariamente]}}

% Se o mapa mental complexo estiver causando problemas, descomente esta linha:
% \newcommand{\generatecomplexdocmap}{\textbf{[Mapa mental complexo desativado temporariamente]}}

% Se o ambiente docstructmap estiver causando problemas, descomente este bloco:
%\renewenvironment{docstructmap}[1][]
%  {\begin{center}\textbf{[Visualização de mapa mental]}\\}
%  {\end{center}}
%\newcommand{\docmaproot}[2][]{#2}
%\newcommand{\docmaplevelone}[4][]{#4}
%\newcommand{\docmapleveltwo}[5][]{#5}
%\newcommand{\docmaplevelthree}[5][]{#5}
%\newcommand{\docmaplevelfour}[5][]{#5}
%\newcommand{\docmaplegend}[1]{#1}

% Nota: Ative as linhas acima conforme necessário para identificar a fonte dos problemas
% Uma vez identificado, você pode resolver os problemas específicos ao invés de
% desativar funcionalidades inteiras.

% OPÇÃO PARA MAPAS MENTAIS
% Escolha uma das opções abaixo (comente a que não vai usar):
% 1. Sistema completo de mapas mentais (pode dar erro em alguns sistemas):
% Pacotes básicos
\usepackage{geometry}
\usepackage{hyperref}
\usepackage{graphicx}
\usepackage{fancyhdr}
\usepackage{titlesec}
\usepackage{xcolor}
\usepackage{microtype}
\usepackage{lipsum}
\usepackage{tcolorbox}
\usepackage{enumitem}
\usepackage{booktabs}
\usepackage{array}
\usepackage{multicol}
\usepackage{afterpage}
\usepackage{tikz}
\usepackage{setspace}
\usepackage{bookmark}

% Suporte a língua portuguesa
\usepackage[brazilian]{babel}

% Bibliografia
\usepackage[backend=biber, style=alphabetic]{biblatex}
\addbibresource{referencias.bib}

% Configurações de fontes usando fontspec (para XeLaTeX)
\usepackage{fontspec}
\defaultfontfeatures{Ligatures=TeX}

% Tentativa de usar a fonte Dominican ou Luxurious Roman
\IfFileExists{Dominican.otf}{%
    \setmainfont{Dominican}
}{%
    \setmainfont{Luxurious Roman}[
        Path = /usr/share/fonts/truetype/luxurious-roman/,
        Extension = .ttf,
        UprightFont = *-Regular,
        BoldFont = *-Regular, % Se não houver variante negrito
        ItalicFont = *-Regular, % Se não houver variante itálica
        Renderer = Basic
    ]
}

% Definição de cores inspiradas em RPG
\definecolor{pergaminho}{RGB}{249, 240, 181}
\definecolor{capa}{RGB}{121, 26, 25}
\definecolor{titulo}{RGB}{72, 26, 19}
\definecolor{boxbg}{RGB}{253, 245, 196}
\definecolor{boxborder}{RGB}{190, 150, 86}
\definecolor{secaotitulo}{RGB}{140, 26, 20}

% Configuração de margens
\geometry{
    a4paper,
    top=2.5cm,
    bottom=2.5cm,
    left=3cm,
    right=3cm,
    headheight=15pt
}

% Configuração de espaçamento
\setlength{\parindent}{1.5cm}
\onehalfspacing

% Estilo de cabeçalho e rodapé
\pagestyle{fancy}
\fancyhf{}
\fancyhead[L]{\leftmark}
\fancyhead[R]{\thepage}
\fancyfoot[C]{\textit{Projeto Homebrew}}
\renewcommand{\headrulewidth}{0.4pt}
\renewcommand{\footrulewidth}{0.4pt}

% Personalização dos títulos de capítulos
\titleformat{\chapter}[display]
{\normalfont\huge\bfseries\color{secaotitulo}}
{\chaptertitlename\ \thechapter}{20pt}{\Huge}
\titlespacing*{\chapter}{0pt}{50pt}{40pt}

% Cor de fundo da página
\pagecolor{pergaminho}

% Ambientes personalizados com tcolorbox para estilo RPG

% Configurações comuns para caixas RPG
\tcbset{
    common/.style={
        enhanced,
        frame hidden,
        interior hidden,
        colback=boxbg,
        colframe=boxborder,
        fonttitle=\bfseries\Large,
        coltitle=white,
        colbacktitle=boxborder,
        attach boxed title to top left={yshift=-2mm, xshift=5mm},
        boxed title style={sharp corners, frame hidden},
        underlay={\begin{tcbclipinterior}
            \draw[boxborder, line width=2pt] 
            (frame.south west) rectangle (frame.north east);
            \end{tcbclipinterior}},
        breakable,
        drop shadow=boxborder,
    }
}

% Ambiente para descrição de itens mágicos
\newtcolorbox{magicitem}[1][]{
    common,
    title=Item Mágico,
    #1
}

% Ambiente para feitiços
\newtcolorbox{spell}[1][]{
    common,
    colbacktitle={rgb:blue,2;green,1;red,6},
    title=Feitiço,
    #1
}

% Ambiente para personagens
\newtcolorbox{character}[1][]{
    common,
    colbacktitle={rgb:red,2;green,2;blue,0},
    title=Personagem,
    #1
}

% Ambiente para notas do mestre
\newtcolorbox{dmnote}[1][]{
    common,
    colbacktitle={rgb:red,0;green,3;blue,5},
    title=Nota do Mestre,
    #1
}

% Ambiente para regras
\newtcolorbox{rule}[1][]{
    common,
    colbacktitle={rgb:red,4;green,0;blue,0},
    title=Regra,
    #1
}

% Ambiente para tabelas
\newtcolorbox{rpgtable}[1][]{
    common,
    colbacktitle={rgb:black,1;yellow,0.1},
    title=Tabela,
    #1
}

% Ambiente para citações
\newtcolorbox{quotebox}[1][]{
    common,
    fonttitle=\itshape\large,
    colbacktitle={rgb:gray,3;black,1},
    title=Citação,
    #1
}

% Ambiente para destaque
\newtcolorbox{highlight}[1][]{
    common,
    colbacktitle={rgb:orange,5;yellow,1},
    title=Destaque,
    #1
}

% Comandos para estatísticas de personagem
\newcommand{\statnumber}[1]{%
    \begingroup
    \setlength{\fboxsep}{2pt}%
    \colorbox{boxbg}{\textbf{#1}}%
    \endgroup
}

\newcommand{\stat}[2]{%
    \textbf{#1} \statnumber{#2}%
}

% Comando para criar barras de atributos
\newcommand{\attrbar}[2]{%
    \begingroup
    \setlength{\unitlength}{1mm}%
    \begin{picture}(30,5)%
    \put(0,0){\color{boxborder}\rule{30mm}{5mm}}%
    \put(0,0){\color{secaotitulo}\rule{#2mm}{5mm}}%
    \put(15,2.5){\makebox(0,0)[c]{\textcolor{white}{\textbf{#1}}}}%
    \end{picture}%
    \endgroup
}

% Comando para titulos estilizados
\newcommand{\rpgtitle}[1]{%
    \begin{center}
        \begingroup
        \setlength{\fboxsep}{5pt}%
        \colorbox{capa}{\textcolor{white}{\Large\bfseries #1}}%
        \endgroup
    \end{center}
}


% 2. Sistema simplificado de mapas mentais (mais compatível com Overleaf):
% % Pacotes básicos
\usepackage{geometry}
\usepackage{hyperref}
\usepackage{graphicx}
\usepackage{fancyhdr}
\usepackage{titlesec}
\usepackage{xcolor}
\usepackage{microtype}
\usepackage{lipsum}
\usepackage{tcolorbox}
\usepackage{enumitem}
\usepackage{booktabs}
\usepackage{array}
\usepackage{multicol}
\usepackage{afterpage}
\usepackage{tikz}
\usepackage{setspace}
\usepackage{bookmark}

% Suporte a língua portuguesa
\usepackage[brazilian]{babel}

% Bibliografia
\usepackage[backend=biber, style=alphabetic]{biblatex}
\addbibresource{referencias.bib}

% Configurações de fontes usando fontspec (para XeLaTeX)
\usepackage{fontspec}
\defaultfontfeatures{Ligatures=TeX}

% Tentativa de usar a fonte Dominican ou Luxurious Roman
\IfFileExists{Dominican.otf}{%
    \setmainfont{Dominican}
}{%
    \setmainfont{Luxurious Roman}[
        Path = /usr/share/fonts/truetype/luxurious-roman/,
        Extension = .ttf,
        UprightFont = *-Regular,
        BoldFont = *-Regular, % Se não houver variante negrito
        ItalicFont = *-Regular, % Se não houver variante itálica
        Renderer = Basic
    ]
}

% Definição de cores inspiradas em RPG
\definecolor{pergaminho}{RGB}{249, 240, 181}
\definecolor{capa}{RGB}{121, 26, 25}
\definecolor{titulo}{RGB}{72, 26, 19}
\definecolor{boxbg}{RGB}{253, 245, 196}
\definecolor{boxborder}{RGB}{190, 150, 86}
\definecolor{secaotitulo}{RGB}{140, 26, 20}

% Configuração de margens
\geometry{
    a4paper,
    top=2.5cm,
    bottom=2.5cm,
    left=3cm,
    right=3cm,
    headheight=15pt
}

% Configuração de espaçamento
\setlength{\parindent}{1.5cm}
\onehalfspacing

% Estilo de cabeçalho e rodapé
\pagestyle{fancy}
\fancyhf{}
\fancyhead[L]{\leftmark}
\fancyhead[R]{\thepage}
\fancyfoot[C]{\textit{Projeto Homebrew}}
\renewcommand{\headrulewidth}{0.4pt}
\renewcommand{\footrulewidth}{0.4pt}

% Personalização dos títulos de capítulos
\titleformat{\chapter}[display]
{\normalfont\huge\bfseries\color{secaotitulo}}
{\chaptertitlename\ \thechapter}{20pt}{\Huge}
\titlespacing*{\chapter}{0pt}{50pt}{40pt}

% Cor de fundo da página
\pagecolor{pergaminho}

% % Ambientes personalizados com tcolorbox para estilo RPG

% Configurações comuns para caixas RPG
\tcbset{
    common/.style={
        enhanced,
        frame hidden,
        interior hidden,
        colback=boxbg,
        colframe=boxborder,
        fonttitle=\bfseries\Large,
        coltitle=white,
        colbacktitle=boxborder,
        attach boxed title to top left={yshift=-2mm, xshift=5mm},
        boxed title style={sharp corners, frame hidden},
        underlay={\begin{tcbclipinterior}
            \draw[boxborder, line width=2pt] 
            (frame.south west) rectangle (frame.north east);
            \end{tcbclipinterior}},
        breakable,
        drop shadow=boxborder,
    }
}

% Ambiente para descrição de itens mágicos
\newtcolorbox{magicitem}[1][]{
    common,
    title=Item Mágico,
    #1
}

% Ambiente para feitiços
\newtcolorbox{spell}[1][]{
    common,
    colbacktitle={rgb:blue,2;green,1;red,6},
    title=Feitiço,
    #1
}

% Ambiente para personagens
\newtcolorbox{character}[1][]{
    common,
    colbacktitle={rgb:red,2;green,2;blue,0},
    title=Personagem,
    #1
}

% Ambiente para notas do mestre
\newtcolorbox{dmnote}[1][]{
    common,
    colbacktitle={rgb:red,0;green,3;blue,5},
    title=Nota do Mestre,
    #1
}

% Ambiente para regras
\newtcolorbox{rule}[1][]{
    common,
    colbacktitle={rgb:red,4;green,0;blue,0},
    title=Regra,
    #1
}

% Ambiente para tabelas
\newtcolorbox{rpgtable}[1][]{
    common,
    colbacktitle={rgb:black,1;yellow,0.1},
    title=Tabela,
    #1
}

% Ambiente para citações
\newtcolorbox{quotebox}[1][]{
    common,
    fonttitle=\itshape\large,
    colbacktitle={rgb:gray,3;black,1},
    title=Citação,
    #1
}

% Ambiente para destaque
\newtcolorbox{highlight}[1][]{
    common,
    colbacktitle={rgb:orange,5;yellow,1},
    title=Destaque,
    #1
}

% Comandos para estatísticas de personagem
\newcommand{\statnumber}[1]{%
    \begingroup
    \setlength{\fboxsep}{2pt}%
    \colorbox{boxbg}{\textbf{#1}}%
    \endgroup
}

\newcommand{\stat}[2]{%
    \textbf{#1} \statnumber{#2}%
}

% Comando para criar barras de atributos
\newcommand{\attrbar}[2]{%
    \begingroup
    \setlength{\unitlength}{1mm}%
    \begin{picture}(30,5)%
    \put(0,0){\color{boxborder}\rule{30mm}{5mm}}%
    \put(0,0){\color{secaotitulo}\rule{#2mm}{5mm}}%
    \put(15,2.5){\makebox(0,0)[c]{\textcolor{white}{\textbf{#1}}}}%
    \end{picture}%
    \endgroup
}

% Comando para titulos estilizados
\newcommand{\rpgtitle}[1]{%
    \begin{center}
        \begingroup
        \setlength{\fboxsep}{5pt}%
        \colorbox{capa}{\textcolor{white}{\Large\bfseries #1}}%
        \endgroup
    \end{center}
}

% % Versão simplificada dos mapas mentais para maior compatibilidade
% Este arquivo fornece uma versão mais leve e compatível com Overleaf

% Definição de cores para os diferentes níveis do mapa mental
\definecolor{level0color}{RGB}{121, 26, 25}  % Raiz (capa)
\definecolor{level1color}{RGB}{70, 26, 100}  % Nível 1 (spell)
\definecolor{level2color}{RGB}{26, 70, 100}  % Nível 2 (magicitem)
\definecolor{level3color}{RGB}{140, 20, 20}  % Nível 3 (rule)

% Ambiente para o mapa mental da estrutura do documento
\newenvironment{docstructmap}[1][]{%
    \begin{center}
    \begin{tikzpicture}[
        mindmap,
        level 1/.style={font=\large\bfseries, sibling angle=60, level distance=5cm},
        level 2/.style={font=\normalsize\bfseries, sibling angle=45, level distance=3.5cm},
        level 3/.style={font=\small, sibling angle=40, level distance=2.5cm},
        level 4/.style={font=\scriptsize, sibling angle=30, level distance=2cm},
        every node/.style={
            text width=4cm, 
            font=\bfseries, 
            minimum size=2cm, 
            fill=boxbg, 
            text=secaotitulo, 
            line width=1pt, 
            draw=boxborder
        },
        every edge/.style={line width=1pt, draw=boxborder}
    ]
}{%
    \end{tikzpicture}
    \end{center}
}

% Comando para adicionar o nó raiz do mapa mental
\newcommand{\docmaproot}[2][]{%
    \node[font=\Large\bfseries, minimum size=3cm, fill=level0color!40!boxbg, text=white, #1] (docroot) {#2};
}

% Comando para adicionar um nó de nível 1
\newcommand{\docmaplevelone}[4][]{%
    % #1 = opções adicionais
    % #2 = ID do nó
    % #3 = ângulo de crescimento
    % #4 = conteúdo do nó
    \node[fill=level1color!10!boxbg, #1] (#2) at (\thesection*#3:5cm) {#4};
    \draw[->] (docroot) -- (#2);
}

% Comando para adicionar um nó de nível 2
\newcommand{\docmapleveltwo}[5][]{%
    % #1 = opções adicionais
    % #2 = ID do nó
    % #3 = ID do nó pai
    % #4 = ângulo de crescimento
    % #5 = conteúdo do nó
    \node[fill=level2color!5!boxbg, #1] (#2) at (#3) ++(\thesection*#4:3.5cm) {#5};
    \draw[->] (#3) -- (#2);
}

% Comando para adicionar um nó de nível 3
\newcommand{\docmaplevelthree}[5][]{%
    % #1 = opções adicionais
    % #2 = ID do nó
    % #3 = ID do nó pai
    % #4 = ângulo de crescimento
    % #5 = conteúdo do nó
    \node[scale=0.7, fill=level3color!3!boxbg, #1] (#2) at (#3) ++(\thesection*#4:2.5cm) {#5};
    \draw[->] (#3) -- (#2);
}

% Comando para adicionar um nó de nível 4
\newcommand{\docmaplevelfour}[5][]{%
    % #1 = opções adicionais
    % #2 = ID do nó
    % #3 = ID do nó pai
    % #4 = ângulo de crescimento
    % #5 = conteúdo do nó
    \node[scale=0.5, fill=level3color!2!boxbg, #1] (#2) at (#3) ++(\thesection*#4:2cm) {#5};
    \draw[->] (#3) -- (#2);
}

% Comando para criar uma legenda para o mapa mental
\newcommand{\docmaplegend}[1][]{%
    \begin{center}
    \begin{tcolorbox}[
        colback=boxbg,
        colframe=boxborder,
        width=0.8\textwidth,
        arc=5mm,
        boxrule=1mm,
        title=Sobre este Mapa Mental
    ]
    #1
    \end{tcolorbox}
    \end{center}
}

% Comandos simplificados para gerar mapas mentais
\newcommand{\generateautodocmap}{%
    \section*{Mapa da Estrutura do Documento}
    
    \begin{docstructmap}
        % Nó raiz - Documento principal
        \docmaproot{Projeto Homebrew\\LaTeX Modular}
        
        % Elementos principais (nível 1)
        \docmaplevelone{pretextual}{1}{Elementos\\Pré-textuais}
        \docmaplevelone{textual}{3}{Elementos\\Textuais}
        \docmaplevelone{postextual}{5}{Elementos\\Pós-textuais}
        
        % Elementos textuais - Capítulos (nível 2)
        \docmapleveltwo{cap1}{textual}{2}{Capítulo 1\\Introdução}
        \docmapleveltwo{cap2}{textual}{3}{Capítulo 2\\Conteúdo}
        \docmapleveltwo{cap3}{textual}{4}{Capítulo 3\\Conclusão}
    \end{docstructmap}
    
    \docmaplegend{
        Este mapa mental foi gerado automaticamente e mostra a estrutura básica do documento.
        A versão simplificada é projetada para maior compatibilidade com diversos sistemas LaTeX.
    }
}

% Comando simplificado para mapa mental complexo
\newcommand{\generatecomplexdocmap}{%
    \section*{Mapa da Estrutura do Documento}
    
    \begin{docstructmap}
        % Nó raiz - Documento principal
        \docmaproot{Projeto Homebrew\\LaTeX Modular}
        
        % Elementos principais (nível 1)
        \docmaplevelone{pretextual}{1}{Elementos\\Pré-textuais}
        \docmaplevelone{textual}{3}{Elementos\\Textuais}
        \docmaplevelone{postextual}{5}{Elementos\\Pós-textuais}
        
        % Elementos pré-textuais (nível 2)
        \docmapleveltwo{capa}{pretextual}{0}{Capa e Título}
        \docmapleveltwo{toc}{pretextual}{1}{Sumário}
        \docmapleveltwo{mapament}{pretextual}{2}{Mapa Mental}
        
        % Elementos textuais - Capítulos (nível 2)
        \docmapleveltwo{cap1}{textual}{2}{Capítulo 1\\Introdução}
        \docmapleveltwo{cap2}{textual}{3}{Capítulo 2\\Conteúdo}
        \docmapleveltwo{cap3}{textual}{4}{Capítulo 3\\Conclusão}
        
        % Elementos pós-textuais (nível 2)
        \docmapleveltwo{ref}{postextual}{4}{Referências\\Bibliográficas}
        \docmapleveltwo{gloss}{postextual}{5}{Glossário}
        \docmapleveltwo{ind}{postextual}{6}{Índice Remissivo}
    \end{docstructmap}
    
    \docmaplegend{
        Este mapa mental apresenta a estrutura completa do documento, incluindo capítulos
        e elementos pré e pós-textuais. Esta versão simplificada é otimizada para 
        compatibilidade com o Overleaf.
    }
}

% Comando para mapa mental de capítulo específico
\newcommand{\generatesectiondocmap}[1][Capítulo Atual]{%
    \section*{Mapa da Estrutura do Capítulo}
    
    \begin{docstructmap}
        % Nó raiz - Capítulo atual
        \docmaproot{#1}
        
        % Seções do capítulo (nível 1)
        \docmaplevelone{sec1}{1}{Seção 1}
        \docmaplevelone{sec2}{3}{Seção 2}
        \docmaplevelone{sec3}{5}{Seção 3}
        
        % Subseções (nível 2)
        \docmapleveltwo{subsec1}{sec1}{0}{Subseção 1.1}
        \docmapleveltwo{subsec2}{sec1}{1}{Subseção 1.2}
        
        \docmapleveltwo{subsec3}{sec2}{3}{Subseção 2.1}
        \docmapleveltwo{subsec4}{sec2}{4}{Subseção 2.2}
    \end{docstructmap}
    
    \docmaplegend{
        Este mapa mental mostra a estrutura deste capítulo e suas seções.
        Use este mapa para orientar-se na leitura ou edição do capítulo.
    }
}

% Configurações adicionais para índice e glossário
\usepackage{imakeidx}         % Para criação de índice
\usepackage{glossaries}       % Para glossário
\makeindex                    % Compila o índice
\makeglossaries               % Compila o glossário

% Início do documento
\begin{document}

% Componentes pré-textuais
% Elementos pré-textuais

% Capa
\begin{titlepage}
    \begin{center}
        \vspace*{2cm}
        
        \begin{tikzpicture}
            \fill[capa] (0,0) rectangle (12,4);
            \node[white, font=\Huge\bfseries] at (6,2) {Projeto Homebrew};
        \end{tikzpicture}
        
        \vspace{1cm}
        {\Large\textit{Modelo Modular LaTeX com Estética RPG}}
        
        \vfill
        
        \Large\textbf{Autor}
        
        \vspace{0.5cm}
        
        \includegraphics[width=5cm]{imgs/logo.png}
        
        \vfill
        
        \large\textbf{\today}
    \end{center}
\end{titlepage}

\cleardoublepage

% Folha de rosto
\thispagestyle{empty}
\begin{center}
    \vspace*{2cm}
    
    {\Huge\bfseries Projeto Homebrew}
    
    \vspace{2cm}
    
    {\large\textit{Documento criado como modelo para projetos LaTeX com estilo RPG}}
    
    \vfill
    
    \begin{flushright}
        \begin{minipage}{8cm}
            \singlespacing
            Modelo de documento modular, combinando\\ 
            estética RPG e organização acadêmica.\\
            Ideal para manuais de jogos, suplementos e\\
            documentos acadêmicos com visual diferenciado.
        \end{minipage}
    \end{flushright}
    
    \vfill
    
    \today
\end{center}

\cleardoublepage

% Lista de siglas
\chapter*{Lista de Abreviaturas e Siglas}
\begin{description}[style=nextline, leftmargin=1cm]
    \item[RPG] Role-Playing Game
    \item[LaTeX] Sistema de preparação de documentos
    \item[DM] Dungeon Master (Mestre do Jogo)
    \item[PDF] Portable Document Format
\end{description}

\cleardoublepage

% Sumário
\tableofcontents

% Carrega os demais elementos pré-textuais
% Elementos pré-textuais

% Página de título
% Página de título personalizada
\begin{titlepage}
    \pagecolor{pergaminho}
    \begin{center}
        \vspace*{2cm}
        
        {\color{titulo}\fontsize{24}{28}\selectfont \textbf{Modelo LaTeX Modular com Estilo RPG}}
        
        \vspace{1cm}
        
        {\color{secaotitulo}\Large Combinando Estética RPG com Rigor Acadêmico}
        
        \vspace{1.5cm}
        
        \begin{tcolorbox}[
            colback=boxbg,
            colframe=boxborder,
            width=0.8\textwidth,
            arc=5mm,
            boxrule=1mm
        ]
            \begin{center}
                {\large\textbf{Projeto Homebrew}}
                
                \vspace{0.5cm}
                
                Um modelo LaTeX para documentação acadêmica\\
                com estética visual inspirada em livros de RPG
                
                \vspace{0.5cm}
                
                \DTMtoday
            \end{center}
        \end{tcolorbox}
        
        \vfill
        
        \begin{tikzpicture}
            \node[draw=boxborder, fill=boxbg, line width=1pt, inner sep=10pt] 
                 {\large\textit{Para uso em trabalhos acadêmicos, manuais,\\
                 documentações técnicas e livros de RPG}};
        \end{tikzpicture}
        
        \vspace{1cm}
    \end{center}
\end{titlepage}

% Reset page color after title page
\newpage
\pagecolor{pergaminho}


% Sumário
\tableofcontents

% Espaçamento após o sumário
\vspace{1cm}

% Mapa mental da estrutura do documento
\clearpage
% Mapa da estrutura do documento
\section*{Mapa do Documento}

\begin{center}
\begin{tikzpicture}[
    mindmap,
    level 1 concept/.append style={font=\large\bfseries, sibling angle=60, level distance=5cm},
    level 2 concept/.append style={font=\normalsize\bfseries, sibling angle=45, level distance=3.5cm},
    level 3 concept/.append style={font=\small, sibling angle=40, level distance=2.5cm},
    concept/.append style={
        text width=4cm, 
        font=\bfseries, 
        minimum size=2cm, 
        fill=boxbg, 
        text=secaotitulo, 
        line width=1pt, 
        draw=boxborder
    },
    concept connection/.append style={line width=1pt, draw=boxborder}
]

% Nó central/raiz - Título do documento
\node[concept, font=\Large\bfseries, minimum size=3cm, fill=capa!40!boxbg, text=white] (doc) {Projeto Homebrew\\LaTeX Modular};

% Elementos pré-textuais - Nível 1
\node[concept, fill=spell!10!boxbg] (pretextual) [grow=30] at (doc.30) {Elementos Pré-textuais};

% Elementos textuais - Nível 1
\node[concept, fill=magicitem!10!boxbg] (textual) [grow=150] at (doc.150) {Elementos Textuais};

% Elementos pós-textuais - Nível 1
\node[concept, fill=rule!10!boxbg] (postextual) [grow=270] at (doc.270) {Elementos Pós-textuais};

% Conexões do nó central aos nós de nível 1
\path (doc) to[circle connection bar] (pretextual);
\path (doc) to[circle connection bar] (textual);
\path (doc) to[circle connection bar] (postextual);

% Elementos pré-textuais - Nível 2
\node[concept, fill=spell!5!boxbg] (title) [grow=0] at (pretextual.0) {Título e Capa};
\node[concept, fill=spell!5!boxbg] (summary) [grow=90] at (pretextual.90) {Sumário};

% Elementos textuais - Nível 2
\node[concept, fill=magicitem!5!boxbg] (chap1) [grow=120] at (textual.120) {Capítulo 1\\Introdução};
\node[concept, fill=magicitem!5!boxbg] (chap2) [grow=180] at (textual.180) {Capítulo 2\\Bibliografia};
\node[concept, fill=magicitem!5!boxbg] (chap3) [grow=240] at (textual.240) {Capítulo 3\\Glossário e Índice};

% Elementos pós-textuais - Nível 2
\node[concept, fill=rule!5!boxbg] (refs) [grow=270] at (postextual.270) {Referências};
\node[concept, fill=rule!5!boxbg] (glossary) [grow=320] at (postextual.320) {Glossário};
\node[concept, fill=rule!5!boxbg] (index) [grow=220] at (postextual.220) {Índice Remissivo};

% Conexões dos nós de nível 1 aos nós de nível 2
\path (pretextual) to[circle connection bar] (title);
\path (pretextual) to[circle connection bar] (summary);

\path (textual) to[circle connection bar] (chap1);
\path (textual) to[circle connection bar] (chap2);
\path (textual) to[circle connection bar] (chap3);

\path (postextual) to[circle connection bar] (refs);
\path (postextual) to[circle connection bar] (glossary);
\path (postextual) to[circle connection bar] (index);

% Elementos do capítulo 1 - Nível 3
\node[concept, scale=0.7, fill=magicitem!2!boxbg] (chap1_1) [grow=90] at (chap1.90) {Ambientes RPG};
\node[concept, scale=0.7, fill=magicitem!2!boxbg] (chap1_2) [grow=150] at (chap1.150) {Comandos Personalizados};

% Elementos do capítulo 2 - Nível 3
\node[concept, scale=0.7, fill=magicitem!2!boxbg] (chap2_1) [grow=150] at (chap2.150) {Sistema de Citações};
\node[concept, scale=0.7, fill=magicitem!2!boxbg] (chap2_2) [grow=210] at (chap2.210) {Referências Cruzadas};

% Elementos do capítulo 3 - Nível 3
\node[concept, scale=0.7, fill=magicitem!2!boxbg] (chap3_1) [grow=210] at (chap3.210) {Glossário};
\node[concept, scale=0.7, fill=magicitem!2!boxbg] (chap3_2) [grow=270] at (chap3.270) {Índice Remissivo};

% Conexões dos nós de nível 2 aos nós de nível 3
\path (chap1) to[circle connection bar] (chap1_1);
\path (chap1) to[circle connection bar] (chap1_2);

\path (chap2) to[circle connection bar] (chap2_1);
\path (chap2) to[circle connection bar] (chap2_2);

\path (chap3) to[circle connection bar] (chap3_1);
\path (chap3) to[circle connection bar] (chap3_2);

\end{tikzpicture}
\end{center}

\vspace{1cm}

\begin{center}
\begin{tcolorbox}[
    colback=boxbg,
    colframe=boxborder,
    width=0.8\textwidth,
    arc=5mm,
    boxrule=1mm,
    title=Sobre este Mapa Mental
]
Este mapa mental apresenta a estrutura geral do documento, mostrando a organização hierárquica do conteúdo. Ele está dividido em três seções principais: elementos pré-textuais, textuais e pós-textuais, cada um contendo seus respectivos componentes.
\end{tcolorbox}
\end{center}


% Mapa mental gerado automaticamente
\clearpage
\generatecomplexdocmap


% Componentes textuais
% Elementos textuais

% Carrega todos os elementos textuais
% Elementos textuais (capítulos)

% Capítulo 1
% Conteúdo do Capítulo 1
\section{Introdução aos Ambientes de RPG}

Este capítulo apresenta uma visão geral dos ambientes personalizados disponíveis neste modelo LaTeX inspirado em RPG.

\subsection{A Estética RPG em Documentos Acadêmicos}

Combinar a estética visual de livros de RPG com o rigor e organização de documentos acadêmicos permite criar materiais que são visualmente atraentes e, ao mesmo tempo, informativos e bem estruturados.

\rpgnote{Este modelo é particularmente útil para criar manuais de RPG, livros de regras para jogos de tabuleiro, ou qualquer documento que se beneficie de uma estética de "pergaminho".}

\begin{dmnote}
Esta caixa destaca informações importantes para o mestre (ou, no contexto acadêmico, para o professor ou orientador). É ideal para notas metodológicas ou instruções especiais.
\end{dmnote}

\subsection{Ambientes Principais}

O modelo inclui diversos ambientes personalizados, cada um com um estilo visual distinto:

\begin{spell}
\textbf{Ambiente Spell (Feitiço)}

Este ambiente é ideal para destacar elementos especiais, como fórmulas matemáticas importantes, algoritmos, ou definições fundamentais.
\end{spell}

\begin{character}
\textbf{Ambiente Character (Personagem)}

Perfeito para biografias, perfis de estudo de caso, ou descrições de entidades importantes no seu trabalho acadêmico.
\end{character}

\begin{rule}
\textbf{Ambiente Rule (Regra)}

Use este ambiente para destacar regras, princípios, teoremas, ou outras declarações normativas que são centrais ao seu trabalho.
\end{rule}

\begin{magicitem}
\textbf{Ambiente Magic Item (Item Mágico)}

Ideal para destacar ferramentas, recursos ou conceitos especiais que têm um papel significativo no seu trabalho.
\end{magicitem}

\rpgsection{Comandos Personalizados}

Além dos ambientes, o modelo também oferece comandos personalizados:

\begin{itemize}
\rpgitem{O comando \texttt{\\rpgnote\{\}} permite adicionar notas na margem}
\rpgitem{O comando \texttt{\\rpgsection\{\}} cria cabeçalhos estilizados}
\rpgitem{O comando \texttt{\\rpgtitle\{\}} cria títulos destacados}
\rpgitem{Os comandos \texttt{\\stat\{\}\{\}} e \texttt{\\attrbar\{\}\{\}} são úteis para apresentar métricas}
\end{itemize}

Exemplo de métricas usando \texttt{\\stat}:

\stat{Clareza}{18} \stat{Organização}{16} \stat{Impacto Visual}{20}

E exemplo de barras de atributos usando \texttt{\\attrbar}:

\attrbar{Facilidade de Uso}{25}
\attrbar{Flexibilidade}{20}
\attrbar{Compatibilidade}{18}

\begin{quotebox}
"A visualização adequada de informações é tão importante quanto o próprio conteúdo. Um documento bem formatado convida à leitura e facilita a compreensão."

— Autor Desconhecido
\end{quotebox}

\begin{highlight}
O estilo visual inspirado em RPG não substitui o conteúdo acadêmico rigoroso. Use estes recursos para enriquecer e destacar seu conteúdo, mantendo sempre o rigor e a clareza característicos de trabalhos acadêmicos.
\end{highlight}

\section{Mapas Mentais da Estrutura do Documento}

Uma das características mais úteis deste modelo é a capacidade de gerar automaticamente mapas mentais que visualizam a estrutura do documento. Esta funcionalidade ajuda leitores e autores a compreender a organização do conteúdo de forma visual e intuitiva.

\subsection{Mapas Mentais Predefinidos}

O modelo inclui mapas mentais pré-definidos que podem ser incluídos em qualquer parte do documento usando o comando \texttt{\\input\{caps/pretextual/docmap\}}. Este mapa é estático e apresenta uma estrutura típica de documento acadêmico.

\subsection{Mapas Mentais Automatizados}

Além dos mapas pré-definidos, o modelo também oferece funcionalidades para gerar mapas mentais automaticamente baseados na estrutura real do documento. Existem dois comandos principais para esta função:

\begin{itemize}
\rpgitem{O comando \texttt{\\generateautodocmap} gera um mapa básico baseado na estrutura atual do documento}
\rpgitem{O comando \texttt{\\generatecomplexdocmap} gera um mapa detalhado com vários níveis de hierarquia}
\end{itemize}

\rpgnote{Os mapas mentais são gerados usando o pacote TikZ e podem ser personalizados conforme necessário. Veja o arquivo \texttt{ambientes/docmap.tex} para mais detalhes.}

\begin{spell}
\textbf{Criando Mapas Mentais Personalizados}

Para criar um mapa mental personalizado, use o ambiente \texttt{docstructmap} e os comandos \texttt{\\docmaproot}, \texttt{\\docmaplevelone}, \texttt{\\docmapleveltwo}, etc. Exemplo:

\begin{verbatim}
\begin{docstructmap}
    \docmaproot{Título Principal}
    \docmaplevelone{secao1}{30}{Seção 1}
    \docmapleveltwo{subsec1}{secao1}{60}{Subseção 1.1}
\end{docstructmap}
\end{verbatim}
\end{spell}

\begin{dmnote}
Os mapas mentais são especialmente úteis para documentos longos ou complexos, pois fornecem uma visão geral clara da estrutura. Considere incluir um no início de cada capítulo principal para orientar o leitor.
\end{dmnote}

\subsection{Exemplo de Mapa Mental Simples}

Abaixo está um exemplo de um mapa mental personalizado para um capítulo:

\begin{docstructmap}
    % Nó raiz - Documento principal
    \docmaproot{Capítulo 1\\Introdução}
    
    % Seções principais (nível 1)
    \docmaplevelone{secao1}{30}{Ambientes\\de RPG}
    \docmaplevelone{secao2}{150}{Comandos\\Personalizados}
    \docmaplevelone{secao3}{270}{Mapas\\Mentais}
    
    % Subseções da Seção 1 (nível 2)
    \docmapleveltwo{subsec1_1}{secao1}{0}{Estética\\Visual}
    \docmapleveltwo{subsec1_2}{secao1}{60}{Ambientes\\Principais}
    
    % Subseções da Seção 2 (nível 2)
    \docmapleveltwo{subsec2_1}{secao2}{120}{Comandos\\de Formatação}
    \docmapleveltwo{subsec2_2}{secao2}{180}{Comandos\\de Estilo}
    
    % Subseções da Seção 3 (nível 2)
    \docmapleveltwo{subsec3_1}{secao3}{240}{Mapas\\Predefinidos}
    \docmapleveltwo{subsec3_2}{secao3}{300}{Mapas\\Automáticos}
\end{docstructmap}

\docmaplegend{
    Este mapa mental ilustra a estrutura do Capítulo 1, mostrando as principais seções e 
    suas respectivas subseções. Os mapas mentais ajudam a visualizar a organização do 
    conteúdo e podem ser criados manualmente ou gerados automaticamente.
}

% Capítulo 2
% Conteúdo do Capítulo 2
\section{Bibliografia e Referências}

Este capítulo demonstra como trabalhar com bibliografia e referências no modelo.

\subsection{Sistema de Citações}

O modelo utiliza o pacote biblatex para gerenciamento de referências. Isso permite citar obras de forma consistente e gerar uma bibliografia formatada automaticamente.

\begin{rule}
\textbf{Citações no Texto}

Para citar uma referência no texto, utilize o comando \texttt{\\cite\{chave\}}, onde \texttt{chave} corresponde ao identificador da referência no arquivo \texttt{referencias.bib}.
\end{rule}

Por exemplo, podemos citar a obra seminal de Donald Knuth sobre tipografia digital \cite{knuth1984texbook} ou discutir as contribuições de Leslie Lamport para o LaTeX \cite{lamport1994latex}.

\rpgnote{O estilo de citação pode ser modificado alterando o parâmetro \texttt{style} na configuração do pacote biblatex em \texttt{configuracoes.tex}.}

\subsection{Gerenciando o Arquivo de Referências}

O arquivo \texttt{referencias.bib} contém todas as entradas bibliográficas no formato BibTeX. Cada entrada segue um padrão como este:

\begin{verbatim}
@book{knuth1984texbook,
  title={The TeXbook},
  author={Knuth, Donald E},
  year={1984},
  publisher={Addison-Wesley}
}
\end{verbatim}

\begin{dmnote}
Recomenda-se utilizar ferramentas como JabRef, Zotero ou Mendeley para gerenciar suas referências bibliográficas. Estas ferramentas permitem exportar entradas no formato BibTeX diretamente para o arquivo \texttt{referencias.bib}.
\end{dmnote}

\subsection{Tipos de Citações}

Existem diferentes formas de citar referências:

\begin{rpgtable}
\begin{tabular}{|l|l|p{8cm}|}
\hline
\textbf{Comando} & \textbf{Exemplo} & \textbf{Resultado} \\
\hline
\texttt{\\cite\{chave\}} & \texttt{\\cite\{lamport1994latex\}} & Citação padrão \cite{lamport1994latex} \\
\hline
\texttt{\\parencite\{chave\}} & \texttt{\\parencite\{knuth1984texbook\}} & Citação entre parênteses \parencite{knuth1984texbook} \\
\hline
\texttt{\\textcite\{chave\}} & \texttt{\\textcite\{lamport1994latex\}} & Citação no texto \textcite{lamport1994latex} \\
\hline
\texttt{\\footcite\{chave\}} & \texttt{\\footcite\{knuth1984texbook\}} & Citação em nota de rodapé\footcite{knuth1984texbook} \\
\hline
\end{tabular}
\end{rpgtable}

\subsection{Referências Cruzadas}

Além de referências bibliográficas, o LaTeX permite criar referências cruzadas dentro do seu próprio documento.

\begin{spell}
\textbf{Mecanismo de Referências Cruzadas}\\
\textit{*Feature de nível 3*}

Para criar uma referência cruzada, primeiro rotule um elemento com \texttt{\\label\{identificador\}} e depois referencie-o usando \texttt{\\ref\{identificador\}}.
\end{spell}

Por exemplo, podemos rotular esta seção com \label{sec:refcruzadas} e referenciá-la como "Seção \ref{sec:refcruzadas}".

\begin{highlight}
\textbf{Dica importante:} 
Sempre compile seu documento pelo menos duas vezes para garantir que todas as referências cruzadas sejam resolvidas corretamente. Na primeira compilação, o LaTeX identifica os rótulos e suas posições; na segunda, ele substitui as referências pelos números corretos.
\end{highlight}

\begin{quotebox}
"O verdadeiro poder do LaTeX está em sua capacidade de gerenciar automaticamente aspectos como numeração, referências e formatação bibliográfica, permitindo que o autor se concentre exclusivamente no conteúdo."

— Um entusiasta do LaTeX
\end{quotebox}

% Capítulo 3
% Conteúdo do Capítulo 3
\section{Glossário e Índice Remissivo}

Esta seção demonstra como utilizar o glossário e o índice remissivo no projeto.

\subsection{Utilizando o Glossário}

O glossário permite criar uma lista de termos importantes e suas definições. Para adicionar um termo ao glossário, use o comando:

\begin{verbatim}
\glossaryentry{termo}{definição}
\end{verbatim}

Exemplo de termos adicionados ao glossário:

\glossaryentry{arcano}{Termo que se refere à magia ou conhecimento místico.}
\glossaryentry{grimório}{Livro contendo feitiços, encantamentos e instruções mágicas.}
\glossaryentry{runa}{Símbolo mágico utilizado em rituais e encantamentos.}

Para referenciar um termo do glossário no texto, use \verb|\gls{termo}|. Por exemplo: 
\gls{arcano}, \gls{grimório} e \gls{runa}.

\subsection{Utilizando o Índice Remissivo}

O índice remissivo permite que os leitores encontrem facilmente termos específicos no documento. Para adicionar um termo ao índice, use:

\begin{verbatim}
\index{termo}
\end{verbatim}

Exemplos de termos indexados:

Magos\index{magos} são estudantes de magia\index{magia} arcana. Eles aprendem feitiços\index{feitiço} 
através de estudo intenso e prática constante. Muitos magos mantêm grimórios\index{grimório} 
onde registram seus conhecimentos arcanos\index{arcano}.

Druidas\index{druida} são praticantes de magia\index{magia!natural} natural. Eles canalizam a 
energia da natureza\index{natureza} e podem assumir formas animais\index{forma animal}.

\subsection{Demonstração de Ambientes}

\begin{spell}
\textbf{Proteção Arcana}\\
\textit{Abjuração de 2º nível}\\
\textbf{Tempo de Conjuração:} 1 ação\\
\textbf{Alcance:} Toque\\
\textbf{Componentes:} V, S, M (um pequeno diamante)\\
\textbf{Duração:} 1 hora

Você toca uma criatura disposta e cria uma barreira mágica ao seu redor. Até o fim da duração, o alvo recebe +2 na CA e tem vantagem em testes de resistência contra magias.
\end{spell}

\rpgnote{Esta magia é especialmente útil antes de enfrentar criaturas que utilizam magias ofensivas.}

\begin{magicitem}
\textbf{Amuleto de Proteção}\\
\textit{Item maravilhoso, raro (requer sintonização)}

Enquanto estiver usando este amuleto, você recebe +1 nas jogadas de resistência e é imune a magias de adivinhação e efeitos sensoriais mágicos que detectariam sua presença.
\end{magicitem}

\begin{dmnote}
Considere dar este item a personagens que enfrentarão inimigos com forte capacidade de rastreamento mágico ou que precisam realizar missões furtivas.
\end{dmnote}

\rpgsection{Lista de Atributos e Estatísticas}

Aqui demonstramos como apresentar estatísticas de personagens usando os comandos personalizados:

\stat{Força}{16}  \stat{Destreza}{14}  \stat{Constituição}{15}  

\stat{Inteligência}{18}  \stat{Sabedoria}{12}  \stat{Carisma}{10}

Barras de atributos também podem ser usadas para representar níveis de habilidade:

\attrbar{Conjuração}{25}

\attrbar{Combate}{15}

\attrbar{Furtividade}{10}

\begin{rule}
\textbf{Sintonização com Itens Mágicos}

Para sintonizar-se com um item, você deve passar 1 hora em contato físico ininterrupto com ele, concentrando-se nele e tentando compreender suas propriedades. Esta concentração pode ocorrer durante um descanso curto e não pode ser interrompida por nenhuma outra atividade.
\end{rule}

\rpgtitle{Tabela de Encontros Aleatórios}

\begin{rpgtable}
\begin{tabular}{|c|p{10cm}|}
\hline
\textbf{d20} & \textbf{Encontro} \\
\hline
1-3 & 1d4 bandidos tentando roubar viajantes \\
\hline
4-6 & Um mercador com carroça quebrada pedindo ajuda \\
\hline
7-10 & 1d6 lobos caçando na área \\
\hline
11-14 & Um druida realizando um ritual para purificar a terra \\
\hline
15-17 & Uma patrulha de 1d4+1 guardas da cidade \\
\hline
18-19 & Um mago errante estudando fenômenos mágicos locais \\
\hline
20 & Um dragão jovem sobrevoando a região \\
\hline
\end{tabular}
\end{rpgtable}

\begin{quotebox}
"A verdadeira magia consiste em compreender a conexão entre todas as coisas, visíveis e invisíveis. Não é apenas poder, mas sabedoria."

— Arquimago Elminster
\end{quotebox}

\begin{highlight}
\textbf{Dica importante para jogadores:} 
Sempre verifique portas, baús e outros objetos suspeitos em busca de armadilhas antes de interagir com eles. Uma simples verificação pode salvar sua vida!
\end{highlight}

% Capítulo sobre Mapas Mentais
% Capítulo especializado sobre Mapas Mentais
\section{Mapas Mentais para Visualização da Estrutura do Documento}

O Projeto Homebrew inclui um sistema completo para geração de mapas mentais para visualizar a estrutura do documento. Este capítulo apresenta exemplos práticos e diretrizes para o uso efetivo desses mapas.

\subsection{O Que São Mapas Mentais de Estrutura}

Os mapas mentais são representações visuais que organizam informações de forma hierárquica, radiando a partir de um conceito central. No contexto da estruturação de documentos, eles ajudam a visualizar a organização geral e as relações entre diferentes partes do texto.

\rpgnote{Um mapa mental bem construído pode servir tanto como guia inicial para o leitor quanto como ferramenta de planejamento para o autor.}

\begin{dmnote}
Os mapas mentais são especialmente úteis em documentos complexos com muitos níveis hierárquicos, como teses, dissertações e livros técnicos.
\end{dmnote}

\subsection{Tipos de Mapas Mentais no Projeto Homebrew}

O sistema oferece três abordagens principais para criação de mapas mentais:

\begin{itemize}
\rpgitem{Mapas pré-definidos, que seguem uma estrutura padrão de documento acadêmico}
\rpgitem{Mapas automáticos, gerados a partir da estrutura real do documento}
\rpgitem{Mapas personalizados, criados manualmente para visualizar conceitos específicos}
\end{itemize}

\subsubsection{Mapas Pré-definidos}

Para incluir um mapa mental pré-definido em seu documento, basta utilizar o comando \texttt{\\generatecomplexdocmap}:

\begin{spell}
\textbf{Exemplo de Código}

\begin{verbatim}
% No arquivo .tex:
\clearpage
\generatecomplexdocmap
\clearpage
\end{verbatim}

Este comando gerará um mapa mental detalhado com a estrutura típica de um documento acadêmico.
\end{spell}

\subsubsection{Mapas Automáticos}

Os mapas automáticos são mais dinâmicos e são gerados a partir da estrutura real do documento:

\begin{spell}
\textbf{Exemplo de Código para Mapa Automático}

\begin{verbatim}
% No arquivo .tex:
\clearpage
\generateautodocmap
\clearpage
\end{verbatim}

Este comando analisa a estrutura atual do documento (suas seções, subseções, etc.) e cria um mapa visual correspondente.
\end{spell}

\rpgnote{Para que o mapa automático funcione corretamente, ele deve ser colocado no documento após as seções que deseja mapear.}

\subsubsection{Mapas Personalizados}

Para casos mais específicos, você pode criar mapas mentais personalizados:

\begin{spell}
\textbf{Exemplo de Código para Mapa Personalizado}

\begin{verbatim}
\begin{docstructmap}
    % Nó raiz
    \docmaproot{Título Principal}
    
    % Nós de primeiro nível
    \docmaplevelone{id1}{30}{Tópico 1}
    \docmaplevelone{id2}{150}{Tópico 2}
    \docmaplevelone{id3}{270}{Tópico 3}
    
    % Nós de segundo nível
    \docmapleveltwo{id1_1}{id1}{0}{Subtópico 1.1}
    \docmapleveltwo{id2_1}{id2}{150}{Subtópico 2.1}
\end{docstructmap}

\docmaplegend{
    Texto explicativo sobre este mapa mental.
}
\end{verbatim}
\end{spell}

\subsection{Exemplos Práticos}

\subsubsection{Mapa Mental para Planejamento de Tese}

Um exemplo de uso prático é o planejamento visual de uma tese:

\begin{docstructmap}
    % Nó raiz
    \docmaproot{Tese de\\Doutorado}
    
    % Capítulos principais (nível 1)
    \docmaplevelone{introducao}{30}{Introdução}
    \docmaplevelone{revisao}{90}{Revisão de\\Literatura}
    \docmaplevelone{metodologia}{150}{Metodologia}
    \docmaplevelone{resultados}{210}{Resultados}
    \docmaplevelone{discussao}{270}{Discussão}
    \docmaplevelone{conclusao}{330}{Conclusão}
    
    % Elementos da Introdução (nível 2)
    \docmapleveltwo{contexto}{introducao}{0}{Contextualização}
    \docmapleveltwo{problema}{introducao}{40}{Problema de\\Pesquisa}
    \docmapleveltwo{objetivos}{introducao}{80}{Objetivos}
    
    % Elementos da Metodologia (nível 2)
    \docmapleveltwo{desenho}{metodologia}{130}{Desenho do\\Estudo}
    \docmapleveltwo{amostra}{metodologia}{150}{Amostragem}
    \docmapleveltwo{analise}{metodologia}{170}{Análise de\\Dados}
    
    % Elementos do Resultado (nível 2)
    \docmapleveltwo{res1}{resultados}{190}{Resultado 1}
    \docmapleveltwo{res2}{resultados}{210}{Resultado 2}
    \docmapleveltwo{res3}{resultados}{230}{Resultado 3}
    
    % Elementos de um resultado específico (nível 3)
    \docmaplevelthree{res1_1}{res1}{180}{Achado 1.1}
    \docmaplevelthree{res1_2}{res1}{200}{Achado 1.2}
\end{docstructmap}

\docmaplegend{
    Este mapa mental ilustra a estrutura típica de uma tese de doutorado, destacando os capítulos 
    principais e detalhando especialmente as seções de Introdução, Metodologia e Resultados.
}

\subsubsection{Mapa Mental para Visualização de Conceitos}

Os mapas mentais também podem ser usados para visualizar conceitos teóricos:

\begin{docstructmap}
    % Nó raiz
    \docmaproot{Teoria da\\Aprendizagem}
    
    % Principais teorias (nível 1)
    \docmaplevelone{comportamental}{30}{Comportamentalismo}
    \docmaplevelone{cognitiva}{150}{Cognitivismo}
    \docmaplevelone{construtivista}{270}{Construtivismo}
    
    % Teóricos do comportamentalismo (nível 2)
    \docmapleveltwo{pavlov}{comportamental}{0}{Pavlov}
    \docmapleveltwo{skinner}{comportamental}{60}{Skinner}
    
    % Conceitos do cognitivismo (nível 2)
    \docmapleveltwo{esquemas}{cognitiva}{120}{Esquemas\\Mentais}
    \docmapleveltwo{memoria}{cognitiva}{180}{Memória\\de Trabalho}
    
    % Vertentes do construtivismo (nível 2)
    \docmapleveltwo{piaget}{construtivista}{240}{Piaget}
    \docmapleveltwo{vygotsky}{construtivista}{300}{Vygotsky}
    
    % Conceitos específicos (nível 3)
    \docmaplevelthree{condclass}{pavlov}{-15}{Condicionamento\\Clássico}
    \docmaplevelthree{condoper}{skinner}{75}{Condicionamento\\Operante}
\end{docstructmap}

\docmaplegend{
    Este mapa mental apresenta as principais teorias da aprendizagem, seus principais 
    representantes e alguns conceitos centrais de cada abordagem teórica.
}

\subsection{Dicas para Criação de Mapas Mentais Efetivos}

\begin{rule}
\textbf{Princípios para Criação de Mapas Mentais Efetivos}

\begin{enumerate}
    \item \textbf{Hierarquia clara:} Mantenha uma estrutura hierárquica bem definida
    \item \textbf{Concisão:} Use palavras-chave ou frases curtas nos nós
    \item \textbf{Equilíbrio visual:} Distribua os nós de forma equilibrada
    \item \textbf{Profundidade adequada:} Limite-se a 3-4 níveis para evitar poluição visual
    \item \textbf{Legendas explicativas:} Inclua sempre uma legenda explicativa
\end{enumerate}
\end{rule}

\rpgnote{Para mapas muito complexos, considere dividir em múltiplos mapas menores, cada um focando em uma parte específica da estrutura.}

\subsection{Integração com o Fluxo do Documento}

Os mapas mentais podem ser integrados em diferentes pontos do documento:

\begin{itemize}
\rpgitem{No início, como visão geral da estrutura completa}
\rpgitem{No início de cada capítulo, mostrando a estrutura específica daquela seção}
\rpgitem{Em apêndices, para visualizar conceitos complexos}
\end{itemize}

\begin{quotebox}
"Um mapa mental bem construído é como um mapa de navegação para o leitor. Ele mostra não apenas onde cada conteúdo está localizado, mas também como os diferentes elementos se relacionam entre si, criando um entendimento holístico da obra."
\end{quotebox}

\subsection{Caso de Uso: Mapa Mental para Planejamento de Escrita}

Um uso particularmente valioso dos mapas mentais é o planejamento do processo de escrita:

\begin{docstructmap}
    % Nó raiz
    \docmaproot{Processo de\\Escrita}
    
    % Fases principais (nível 1)
    \docmaplevelone{planejamento}{30}{Planejamento}
    \docmaplevelone{primeira}{150}{Primeira\\Versão}
    \docmaplevelone{revisao}{270}{Revisão e\\Finalização}
    
    % Elementos do planejamento (nível 2)
    \docmapleveltwo{tema}{planejamento}{0}{Definição\\do Tema}
    \docmapleveltwo{pesquisa}{planejamento}{60}{Pesquisa\\Bibliográfica}
    
    % Elementos da primeira versão (nível 2)
    \docmapleveltwo{rascunho}{primeira}{120}{Rascunho\\Inicial}
    \docmapleveltwo{expansao}{primeira}{180}{Expansão\\de Ideias}
    
    % Elementos da revisão (nível 2)
    \docmapleveltwo{revisao1}{revisao}{240}{Revisão\\de Conteúdo}
    \docmapleveltwo{revisao2}{revisao}{300}{Revisão\\de Forma}
    
    % Elementos específicos (nível 3)
    \docmaplevelthree{bib}{pesquisa}{30}{Seleção de\\Bibliografia}
    \docmaplevelthree{notas}{pesquisa}{90}{Tomada\\de Notas}
    
    \docmaplevelthree{gramatical}{revisao2}{270}{Revisão\\Gramatical}
    \docmaplevelthree{formatacao}{revisao2}{330}{Formatação\\e Estilo}
\end{docstructmap}

\docmaplegend{
    Este mapa mental apresenta o processo de escrita acadêmica, destacando as principais
    fases e atividades em cada etapa. Pode ser usado como guia para organizar o trabalho
    de redação de um documento acadêmico.
}

\section{Conclusão}

Os mapas mentais são uma ferramenta poderosa para visualização da estrutura do documento, planejamento da escrita e organização de conceitos complexos. O sistema implementado no Projeto Homebrew oferece flexibilidade para criar diferentes tipos de mapas, desde estruturas predefinidas até visualizações completamente personalizadas.

\begin{highlight}
Para mais detalhes sobre a implementação técnica e opções avançadas de personalização, consulte o arquivo de documentação \texttt{docs/tutorial-mapamental.md} e os exemplos em \texttt{exemplos/exemplo-mapamental.tex}.
\end{highlight}



% Componentes pós-textuais
% Elementos pós-textuais

% Carrega todos os elementos pós-textuais
% Conteúdo dos elementos pós-textuais

\cleardoublepage
\chapter*{Conclusão}

Este documento demonstra as capacidades do modelo modular LaTeX com estética de RPG. Combine a organização de documentos acadêmicos com o estilo visual dos livros de RPG para criar documentos únicos e interessantes.

A estética inspirada nos livros de RPG, com seus ambientes customizados, formatação especial e uso de cores, permite criar documentos técnicos ou acadêmicos com uma apresentação visualmente atraente e diferenciada \cite{rpglayout}.

\cleardoublepage
\chapter*{Glossário}

\newglossaryentry{latex}{
    name={LaTeX},
    description={Sistema de tipografia que permite a criação de documentos com alta qualidade tipográfica}
}

\newglossaryentry{rpg}{
    name={RPG},
    description={Role-Playing Game, jogo onde os jogadores assumem os papéis de personagens em um cenário fictício}
}

\newglossaryentry{homebrewery}{
    name={Homebrewery},
    description={Estilo de formatação inspirado em livros de RPG, especialmente Dungeons \& Dragons}
}

\newglossaryentry{modular}{
    name={Modular},
    description={Estrutura organizada em componentes independentes que podem ser combinados}
}

\newglossaryentry{tcolorbox}{
    name={tcolorbox},
    description={Pacote LaTeX para criação de caixas coloridas e personalizadas \cite{tcolorbox}}
}

\printglossaries

\cleardoublepage
\chapter*{Apêndices}

\begin{rpgtable}
\begin{tabular}{lcc}
\toprule
\textbf{Item} & \textbf{Categoria} & \textbf{Valor} \\
\midrule
Espada Longa & Arma & 15 po \\
Poção de Cura & Consumível & 50 po \\
Grimório & Mágico & 100 po \\
Botas Élficas & Equipamento & 25 po \\
Manto da Invisibilidade & Mágico & 500 po \\
\bottomrule
\end{tabular}
\caption{Tabela de itens comuns em aventuras de RPG}
\label{tab:itens}
\end{rpgtable}

\cleardoublepage
\chapter*{Comandos de Referência}

\begin{rpgtable}
\begin{tabularx}{\textwidth}{llX}
\toprule
\textbf{Comando} & \textbf{Tipo} & \textbf{Descrição} \\
\midrule
\verb|\begin{magicitem}| & Ambiente & Cria uma caixa para descrição de itens mágicos \\
\verb|\begin{spell}| & Ambiente & Cria uma caixa para descrição de magias \\
\verb|\begin{character}| & Ambiente & Cria uma caixa para ficha de personagens \\
\verb|\begin{dmnote}| & Ambiente & Cria uma caixa para notas do mestre \\
\verb|\begin{rule}| & Ambiente & Cria uma caixa para regras especiais \\
\verb|\begin{rpgtable}| & Ambiente & Cria uma caixa para tabelas \\
\verb|\begin{quotebox}| & Ambiente & Cria uma caixa para citações \\
\verb|\begin{highlight}| & Ambiente & Cria uma caixa para destacar informações \\
\verb|\rpgtitle{Título}| & Comando & Cria um título estilizado \\
\verb|\rpgnote{Texto}| & Comando & Cria uma nota na margem \\
\verb|\rpgsection{Título}| & Comando & Cria um cabeçalho de seção estilizado \\
\verb|\stat{Atributo}{Valor}| & Comando & Exibe um atributo com valor destacado \\
\verb|\attrbar{Texto}{Valor}| & Comando & Cria uma barra de atributo \\
\bottomrule
\end{tabularx}
\caption{Comandos personalizados disponíveis neste modelo}
\label{tab:comandos}
\end{rpgtable}

\cleardoublepage
\printbibliography[title=Referências]



% Imprimir o glossário
\printglossary[title=Glossário de Termos]

% Imprimir o índice remissivo
\printindex

\end{document}
