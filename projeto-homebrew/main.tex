% Documento principal - Compila todos os componentes
\documentclass[
    12pt,                     % tamanho da fonte
    twoside,                  % para impressão em verso e anverso
    openany,                  % capítulos podem começar em qualquer página
    a4paper,                  % tamanho do papel
    brazil,                   % idioma principal
    english,                  % idioma adicional
    french                    % idioma adicional
]{article}

% OPÇÕES DE COMPATIBILIDADE COM OVERLEAF
% Descomente a linha abaixo se estiver tendo problemas no Overleaf
% % Configurações específicas para compilação no Overleaf
% Inclua este arquivo no seu main.tex para resolver problemas de compilação

% Verificar e carregar pacotes necessários para os mapas mentais
\usepackage{tikz}
\usetikzlibrary{mindmap,trees,shadows,arrows,positioning}
\usetikzlibrary{decorations.pathmorphing}
\usetikzlibrary{decorations.markings}
\usetikzlibrary{shapes.geometric}

% Pacotes necessários para manipulação de listas e strings
\usepackage{etoolbox}
\usepackage{xstring}

% Em caso de erro no comando \protected@edef, descomente esta linha:
% \makeatletter\let\protected@edef\edef\makeatother

% Desativação temporária de recursos avançados para verificação
% Se o mapa mental automático estiver causando problemas, descomente esta linha:
% \newcommand{\generateautodocmap}{\textbf{[Mapa mental automático desativado temporariamente]}}

% Se o mapa mental complexo estiver causando problemas, descomente esta linha:
% \newcommand{\generatecomplexdocmap}{\textbf{[Mapa mental complexo desativado temporariamente]}}

% Se o ambiente docstructmap estiver causando problemas, descomente este bloco:
%\renewenvironment{docstructmap}[1][]
%  {\begin{center}\textbf{[Visualização de mapa mental]}\\}
%  {\end{center}}
%\newcommand{\docmaproot}[2][]{#2}
%\newcommand{\docmaplevelone}[4][]{#4}
%\newcommand{\docmapleveltwo}[5][]{#5}
%\newcommand{\docmaplevelthree}[5][]{#5}
%\newcommand{\docmaplevelfour}[5][]{#5}
%\newcommand{\docmaplegend}[1]{#1}

% Nota: Ative as linhas acima conforme necessário para identificar a fonte dos problemas
% Uma vez identificado, você pode resolver os problemas específicos ao invés de
% desativar funcionalidades inteiras.

% OPÇÃO PARA MAPAS MENTAIS
% Escolha uma das opções abaixo (comente a que não vai usar):
% 1. Sistema completo de mapas mentais (pode dar erro em alguns sistemas):
% Pacotes básicos
\usepackage{geometry}
\usepackage{hyperref}
\usepackage{graphicx}
\usepackage{fancyhdr}
\usepackage{titlesec}
\usepackage[usenames,dvipsnames]{xcolor}
\usepackage{microtype}
\usepackage{lipsum}
\usepackage{tcolorbox}
\usepackage{enumitem}
\usepackage{booktabs}
\usepackage{array}
\usepackage{multicol}
\usepackage{afterpage}
\usepackage{tikz}
\usepackage{setspace}
\usepackage{bookmark}
\usepackage{float}
\usepackage{lastpage}
\usepackage[framemethod=TikZ]{mdframed}
\usepackage{subfig}
\usepackage{csquotes}
\usepackage{url}
\usepackage{tabularx}
\usepackage{verbatim}
\usepackage{tocbibind}
\usepackage{newfloat}
\usepackage[useregional]{datetime2}
\usepackage{ragged2e}
\usepackage{marginnote}
\usepackage{pifont}
\usepackage{etoolbox}

% Pacotes para mapa mental/estrutura do documento
\usetikzlibrary{mindmap,trees,shadows,arrows,positioning}
\usetikzlibrary{decorations.pathmorphing}
\usetikzlibrary{decorations.markings}
\usetikzlibrary{shapes.geometric}

% Suporte a língua portuguesa
\usepackage[brazilian]{babel}

% Bibliografia
\usepackage[backend=biber, style=alphabetic]{biblatex}
\addbibresource{referencias.bib}

% Configurações de hyperlinks
\hypersetup{
    colorlinks=true,
    linkcolor=secaotitulo,
    citecolor=blue,
    urlcolor=blue
}

% Configurações de fontes usando fontspec (para XeLaTeX)
\usepackage{fontspec}
\defaultfontfeatures{Ligatures=TeX}

% Configuração de fontes para documentos RPG
\setmainfont{Latin Modern Roman}

% Definição de cores inspiradas em RPG
\definecolor{pergaminho}{RGB}{249, 240, 181}
\definecolor{capa}{RGB}{121, 26, 25}
\definecolor{titulo}{RGB}{72, 26, 19}
\definecolor{boxbg}{RGB}{253, 245, 196}
\definecolor{boxborder}{RGB}{190, 150, 86}
\definecolor{secaotitulo}{RGB}{140, 26, 20}
\definecolor{spell}{RGB}{70, 26, 100}
\definecolor{magicitem}{RGB}{26, 70, 100}
\definecolor{rule}{RGB}{140, 20, 20}
\definecolor{note}{RGB}{20, 100, 120}

% Configuração de margens
\geometry{
    a4paper,
    top=2.5cm,
    bottom=2.5cm,
    left=3cm,
    right=3cm,
    headheight=15pt,
    marginparwidth=2.5cm,
    marginparsep=0.5cm
}

% Configuração de espaçamento
\setlength{\parindent}{1.5cm}
\onehalfspacing

% Estilo de cabeçalho e rodapé
\pagestyle{fancy}
\fancyhf{}
\fancyhead[L]{\leftmark}
\fancyhead[R]{\thepage}
\fancyfoot[C]{\textit{Projeto Homebrew}}
\renewcommand{\headrulewidth}{0.4pt}
\renewcommand{\footrulewidth}{0.4pt}

% Personalização dos títulos de capítulos
\titleformat{\chapter}[display]
{\normalfont\huge\bfseries\color{secaotitulo}}
{\chaptertitlename\ \thechapter}{20pt}{\Huge}
\titlespacing*{\chapter}{0pt}{50pt}{40pt}

% Cor de fundo da página
\pagecolor{pergaminho}

% Ambientes personalizados com tcolorbox para estilo RPG

% Configurações comuns para caixas RPG
\tcbset{
    common/.style={
        enhanced,
        frame hidden,
        interior hidden,
        colback=boxbg,
        colframe=boxborder,
        fonttitle=\bfseries\Large,
        coltitle=white,
        colbacktitle=boxborder,
        attach boxed title to top left={yshift=-2mm, xshift=5mm},
        boxed title style={sharp corners, frame hidden},
        underlay={\begin{tcbclipinterior}
            \draw[boxborder, line width=2pt] 
            (frame.south west) rectangle (frame.north east);
            \end{tcbclipinterior}},
        breakable,
        drop shadow=boxborder,
        before skip=0.5cm,
        after skip=0.5cm,
    }
}

% Estilo alternativo usando mdframed
\mdfdefinestyle{rpgstyle}{%
    linecolor=boxborder,
    linewidth=2pt,
    backgroundcolor=boxbg,
    roundcorner=5pt,
    shadow=true,
    shadowcolor=black!30,
}

% Ambiente para descrição de itens mágicos
\newtcolorbox{magicitem}[1][]{
    common,
    colbacktitle=magicitem,
    title=Item Mágico,
    #1
}

% Ambiente para feitiços
\newtcolorbox{spell}[1][]{
    common,
    colbacktitle=spell,
    title=Feitiço,
    #1
}

% Ambiente para personagens
\newtcolorbox{character}[1][]{
    common,
    colbacktitle=Brown!80!black,
    title=Personagem,
    #1
}

% Ambiente para notas do mestre
\newtcolorbox{dmnote}[1][]{
    common,
    colbacktitle=note,
    title=Nota do Mestre,
    #1
}

% Ambiente para regras
\newtcolorbox{rule}[1][]{
    common,
    colbacktitle=rule,
    title=Regra,
    #1
}

% Ambiente para tabelas
\newtcolorbox{rpgtable}[1][]{
    common,
    colbacktitle=Mahogany!70!black,
    title=Tabela,
    #1
}

% Ambiente para citações
\newtcolorbox{quotebox}[1][]{
    common,
    fonttitle=\itshape\large,
    colbacktitle=Gray!70!black,
    title=Citação,
    #1
}

% Ambiente para destaque
\newtcolorbox{highlight}[1][]{
    common,
    colbacktitle=Orange!70!black,
    title=Destaque,
    #1
}

% Comandos para estatísticas de personagem
\newcommand{\statnumber}[1]{%
    \begingroup
    \setlength{\fboxsep}{2pt}%
    \colorbox{boxbg}{\textbf{#1}}%
    \endgroup
}

\newcommand{\stat}[2]{%
    \textbf{#1} \statnumber{#2}%
}

% Comando para criar barras de atributos
\newcommand{\attrbar}[2]{%
    \begingroup
    \setlength{\unitlength}{1mm}%
    \begin{picture}(30,5)%
    \put(0,0){\color{boxborder}\rule{30mm}{5mm}}%
    \put(0,0){\color{secaotitulo}\rule{#2mm}{5mm}}%
    \put(15,2.5){\makebox(0,0)[c]{\textcolor{white}{\textbf{#1}}}}%
    \end{picture}%
    \endgroup
}

% Comando para criar listas de itens estilizadas
\newcommand{\rpgitem}[1]{%
    \item[\textcolor{secaotitulo}{\small\ding{108}}] #1
}

% Comando para titulos estilizados
\newcommand{\rpgtitle}[1]{%
    \begin{center}
        \begingroup
        \setlength{\fboxsep}{5pt}%
        \colorbox{capa}{\textcolor{white}{\Large\bfseries #1}}%
        \endgroup
    \end{center}
}

% Comando para notas de margem (anotações)
\newcommand{\rpgnote}[1]{%
    \marginpar{%
        \begin{mdframed}[style=rpgstyle]
        {\small\itshape #1}
        \end{mdframed}%
    }%
}

% Comando para adição de glossário
\newcommand{\glossaryentry}[2]{%
    \newglossaryentry{#1}{%
        name=#1,%
        description={#2}%
    }%
}

% Comando para encabezamento de seção RPG
\newcommand{\rpgsection}[1]{%
    \vspace{0.5cm}
    \begin{center}
        \begin{tikzpicture}
            \node[draw=boxborder, fill=boxbg, line width=2pt, inner sep=8pt] 
                 {\large\bfseries\color{secaotitulo} #1};
        \end{tikzpicture}
    \end{center}
    \vspace{0.3cm}
}

% Importação dos ambientes para mapa mental de estrutura do documento
% Funções automatizadas para geração de mapas mentais de estrutura de documentos

% Definição de cores para os diferentes níveis do mapa mental
\definecolor{level0color}{RGB}{121, 26, 25}  % Raiz (capa)
\definecolor{level1color}{RGB}{70, 26, 100}  % Nível 1 (spell)
\definecolor{level2color}{RGB}{26, 70, 100}  % Nível 2 (magicitem)
\definecolor{level3color}{RGB}{140, 20, 20}  % Nível 3 (rule)
\definecolor{level4color}{RGB}{20, 100, 120} % Nível 4 (note)

% Ambiente para o mapa mental da estrutura do documento
\newenvironment{docstructmap}[1][]{%
    \begin{center}
    \begin{tikzpicture}[
        mindmap,
        level 1 concept/.append style={font=\large\bfseries, sibling angle=60, level distance=5cm},
        level 2 concept/.append style={font=\normalsize\bfseries, sibling angle=45, level distance=3.5cm},
        level 3 concept/.append style={font=\small, sibling angle=40, level distance=2.5cm},
        level 4 concept/.append style={font=\scriptsize, sibling angle=30, level distance=2cm},
        concept/.append style={
            text width=4cm, 
            font=\bfseries, 
            minimum size=2cm, 
            fill=boxbg, 
            text=secaotitulo, 
            line width=1pt, 
            draw=boxborder
        },
        concept connection/.append style={line width=1pt, draw=boxborder}
    ]
}{%
    \end{tikzpicture}
    \end{center}
}

% Comando para adicionar o nó raiz do mapa mental
\newcommand{\docmaproot}[2][]{%
    \node[concept, font=\Large\bfseries, minimum size=3cm, fill=level0color!40!boxbg, text=white, #1] (docroot) {#2};
}

% Comando para adicionar um nó de nível 1
\newcommand{\docmaplevelone}[4][]{%
    % #1 = opções adicionais
    % #2 = ID do nó
    % #3 = ângulo de crescimento
    % #4 = conteúdo do nó
    \node[concept, fill=level1color!10!boxbg, #1] (#2) [grow=#3] at (docroot.#3) {#4};
    \path (docroot) to[circle connection bar] (#2);
}

% Comando para adicionar um nó de nível 2
\newcommand{\docmapleveltwo}[5][]{%
    % #1 = opções adicionais
    % #2 = ID do nó
    % #3 = ID do nó pai
    % #4 = ângulo de crescimento
    % #5 = conteúdo do nó
    \node[concept, fill=level2color!5!boxbg, #1] (#2) [grow=#4] at (#3.#4) {#5};
    \path (#3) to[circle connection bar] (#2);
}

% Comando para adicionar um nó de nível 3
\newcommand{\docmaplevelthree}[5][]{%
    % #1 = opções adicionais
    % #2 = ID do nó
    % #3 = ID do nó pai
    % #4 = ângulo de crescimento
    % #5 = conteúdo do nó
    \node[concept, scale=0.7, fill=level3color!3!boxbg, #1] (#2) [grow=#4] at (#3.#4) {#5};
    \path (#3) to[circle connection bar] (#2);
}

% Comando para adicionar um nó de nível 4
\newcommand{\docmaplevelfour}[5][]{%
    % #1 = opções adicionais
    % #2 = ID do nó
    % #3 = ID do nó pai
    % #4 = ângulo de crescimento
    % #5 = conteúdo do nó
    \node[concept, scale=0.5, fill=level4color!2!boxbg, #1] (#2) [grow=#4] at (#3.#4) {#5};
    \path (#3) to[circle connection bar] (#2);
}

% Comando para criar uma legenda para o mapa mental
\newcommand{\docmaplegend}[1][]{%
    \begin{center}
    \begin{tcolorbox}[
        colback=boxbg,
        colframe=boxborder,
        width=0.8\textwidth,
        arc=5mm,
        boxrule=1mm,
        title=Sobre este Mapa Mental
    ]
    #1
    \end{tcolorbox}
    \end{center}
}
% Implementação de um sistema automatizado para geração de mapa mental da estrutura do documento
% Este arquivo fornece funcionalidades para analisar automaticamente a estrutura do documento

% Pacotes necessários para manipulação de listas e strings
\usepackage{etoolbox}
\usepackage{xstring}

% CORREÇÕES PARA COMPATIBILIDADE COM XELATEX 2023+
% Estas correções previnem a exposição de comandos internos no documento final

% Proteções para evitar que comandos internos apareçam no documento
\makeatletter
% Este comando formata texto para evitar que variáveis internas sejam expostas
\newcommand{\protect@internal@cmd}[1]{%
  % Remove prefixos comuns de comandos internos
  \IfBeginWith{#1}{@}{}{%
    \IfBeginWith{#1}{autodocmap@}{}{%
      \IfBeginWith{#1}{sectionbaseangle}{}{%
        \IfBeginWith{#1}{subsectionbaseangle}{}{%
          \IfBeginWith{#1}{sec1textual}{}{%
            #1%
          }%
        }%
      }%
    }%
  }%
}
\makeatother

% Contador para manter controle de seções para o mapa mental
\newcounter{docmapnodecounter}
\setcounter{docmapnodecounter}{0}

% Armazenamento dos títulos das seções utilizando listas mais robustas
\def\autodocmap@sectionlist{}
\def\autodocmap@subsectionlist{}
\def\autodocmap@subsubsectionlist{}

% Contadores para o número de seções, subseções, etc.
\newcounter{sectioncount}
\newcounter{subsectioncount}
\newcounter{subsubsectioncount}
\newcounter{sectionangle}
\newcounter{subsectionangle}
\newcounter{subsubsectionangle}

% Salva os comandos originais
\let\oldsection\section
\let\oldsubsection\subsection
\let\oldsubsubsection\subsubsection
\let\oldchapter\chapter

% Define os ângulos base para cada nível como comandos protegidos
% Isso previne que esses valores apareçam no documento final
\makeatletter
\protected\def\autodocmap@sectionbaseangle{120}
\protected\def\autodocmap@subsectionbaseangle{30}
\protected\def\autodocmap@subsubsectionbaseangle{45}
\makeatother

% Armazenar informações do capítulo atual
\protected\def\autodocmap@currentchapter{}

% Redefine o comando de capítulo para rastrear títulos
\renewcommand{\chapter}[2][]{%
    \renewcommand{\autodocmap@currentchapter}{#2}%
    \oldchapter[#1]{#2}%
}

% Redefine o comando de seção para rastrear títulos
\renewcommand{\section}[2][]{%
    \stepcounter{sectioncount}%
    \setcounter{subsectioncount}{0}%
    \setcounter{sectionangle}{\value{sectioncount}}%
    \multiply\value{sectionangle} by 60%
    % Armazena os dados da seção para uso posterior no mapa
    \protected@edef\autodocmap@sectionlist{%
        \autodocmap@sectionlist
        \noexpand\docmapleveltwo{sec\arabic{sectioncount}}{textual}{\thesectionangle}{#2}%
    }%
    \oldsection[#1]{#2}%
}

% Redefine o comando de subseção para rastrear títulos
\renewcommand{\subsection}[2][]{%
    \stepcounter{subsectioncount}%
    \setcounter{subsubsectioncount}{0}%
    \setcounter{subsectionangle}{\autodocmap@subsectionbaseangle}%
    \multiply\value{subsectionangle} by \value{subsectioncount}%
    % Armazena os dados da subseção para uso posterior no mapa
    \protected@edef\autodocmap@subsectionlist{%
        \autodocmap@subsectionlist
        \noexpand\docmaplevelthree{subsec\arabic{sectioncount}_\arabic{subsectioncount}}{sec\arabic{sectioncount}}{\thesubsectionangle}{#2}%
    }%
    \oldsubsection[#1]{#2}%
}

% Redefine o comando de subsubseção para rastrear títulos
\renewcommand{\subsubsection}[2][]{%
    \stepcounter{subsubsectioncount}%
    \setcounter{subsubsectionangle}{\autodocmap@subsubsectionbaseangle}%
    \multiply\value{subsubsectionangle} by \value{subsubsectioncount}%
    % Armazena os dados da subsubseção para uso posterior no mapa
    \protected@edef\autodocmap@subsubsectionlist{%
        \autodocmap@subsubsectionlist
        \noexpand\docmaplevelfour{subsubsec\arabic{sectioncount}_\arabic{subsectioncount}_\arabic{subsubsectioncount}}{subsec\arabic{sectioncount}_\arabic{subsectioncount}}{\thesubsubsectionangle}{#2}%
    }%
    \oldsubsubsection[#1]{#2}%
}

% Comando para gerar automaticamente o mapa mental com base nos dados coletados
\newcommand{\generateautodocmap}{%
    \section*{Mapa Automático da Estrutura do Documento}
    
    \begin{docstructmap}
        % Nó raiz - Documento principal
        \docmaproot{Projeto Homebrew\\LaTeX Modular}
        
        % Elementos principais (nível 1)
        \docmaplevelone{pretextual}{30}{Elementos\\Pré-textuais}
        \docmaplevelone{textual}{150}{Elementos\\Textuais}
        \docmaplevelone{postextual}{270}{Elementos\\Pós-textuais}
        
        % Nós pré-definidos para elementos comuns (nível 2)
        \docmapleveltwo{capa}{pretextual}{0}{Capa e Título}
        \docmapleveltwo{toc}{pretextual}{90}{Sumário}
        
        \docmapleveltwo{ref}{postextual}{240}{Referências\\Bibliográficas}
        \docmapleveltwo{gloss}{postextual}{300}{Glossário}
        \docmapleveltwo{ind}{postextual}{340}{Índice Remissivo}
        
        % Gerar nós para cada seção registrada (nível 2)
        \autodocmap@sectionlist
        
        % Gerar nós para cada subseção registrada (nível 3)
        \autodocmap@subsectionlist
        
        % Gerar nós para cada subsubseção registrada (nível 4)
        \autodocmap@subsubsectionlist
        
    \end{docstructmap}
    
    \docmaplegend{
        Este mapa mental foi gerado automaticamente com base na estrutura real do documento.
        Ele representa a hierarquia de capítulos, seções e subseções no formato de um mapa mental.
        As cores indicam diferentes níveis na hierarquia do documento.
    }
}

% Comando para gerar automaticamente um mapa mental focado na estrutura atual
\newcommand{\generatesectiondocmap}[1][Capítulo Atual]{%
    \section*{Mapa da Estrutura do Capítulo}
    
    \begin{docstructmap}
        % Nó raiz - Capítulo atual
        \docmaproot{#1}
        
        % Gerar nós para cada seção registrada (nível 1)
        % Esta é uma versão simplificada que mostra apenas o capítulo atual
        % Em uma implementação mais completa, seria filtrado pelo capítulo atual
        \autodocmap@sectionlist
        
        % Gerar nós para cada subseção registrada (nível 2)
        \autodocmap@subsectionlist
        
    \end{docstructmap}
    
    \docmaplegend{
        Este mapa mental mostra a estrutura específica deste capítulo, 
        destacando suas seções e subseções principais.
    }
}

% Comando para gerar um mapa mental complexo predefinido
\newcommand{\generatecomplexdocmap}{%
    \section*{Mapa da Estrutura do Documento}
    
    \begin{docstructmap}
        % Nó raiz - Documento principal
        \docmaproot{Projeto Homebrew\\LaTeX Modular}
        
        % Elementos principais (nível 1)
        \docmaplevelone{pretextual}{30}{Elementos\\Pré-textuais}
        \docmaplevelone{textual}{150}{Elementos\\Textuais}
        \docmaplevelone{postextual}{270}{Elementos\\Pós-textuais}
        
        % Elementos pré-textuais (nível 2)
        \docmapleveltwo{capa}{pretextual}{0}{Capa e Título}
        \docmapleveltwo{toc}{pretextual}{60}{Sumário}
        \docmapleveltwo{mapament}{pretextual}{120}{Mapa Mental}
        
        % Elementos textuais - Capítulos (nível 2)
        \docmapleveltwo{cap1}{textual}{120}{Capítulo 1\\Introdução}
        \docmapleveltwo{cap2}{textual}{180}{Capítulo 2\\Bibliografia}
        \docmapleveltwo{cap3}{textual}{240}{Capítulo 3\\Glossário e Índice}
        
        % Elementos pós-textuais (nível 2)
        \docmapleveltwo{ref}{postextual}{240}{Referências\\Bibliográficas}
        \docmapleveltwo{gloss}{postextual}{300}{Glossário}
        \docmapleveltwo{ind}{postextual}{340}{Índice Remissivo}
        
        % Conteúdo do Capítulo 1 (nível 3)
        \docmaplevelthree{cap1_1}{cap1}{90}{Ambientes RPG}
        \docmaplevelthree{cap1_2}{cap1}{150}{Comandos\\Personalizados}
        \docmaplevelthree{cap1_3}{cap1}{210}{Mapas Mentais}
        
        % Conteúdo do Capítulo 2 (nível 3)
        \docmaplevelthree{cap2_1}{cap2}{150}{Sistema de Citações}
        \docmaplevelthree{cap2_2}{cap2}{210}{Referências Cruzadas}
        
        % Conteúdo do Capítulo 3 (nível 3)
        \docmaplevelthree{cap3_1}{cap3}{210}{Glossário}
        \docmaplevelthree{cap3_2}{cap3}{270}{Índice Remissivo}
        
        % Exemplos de nível 4 (detalhes mais específicos)
        \docmaplevelfour{amb1}{cap1_1}{60}{Spell}
        \docmaplevelfour{amb2}{cap1_1}{120}{Character}
        \docmaplevelfour{amb3}{cap1_1}{180}{Rule}
        
        \docmaplevelfour{cmd1}{cap1_2}{120}{rpgnote}
        \docmaplevelfour{cmd2}{cap1_2}{180}{rpgsection}
        \docmaplevelfour{cmd3}{cap1_2}{240}{rpgitem}
        
        % Conteúdo do mapa mental (nível 4)
        \docmaplevelfour{map1}{cap1_3}{120}{Estrutura\\Automática}
        \docmaplevelfour{map2}{cap1_3}{180}{Mapas\\Personalizados}
        \docmaplevelfour{map3}{cap1_3}{240}{Visualização\\Hierárquica}
    
    \end{docstructmap}
    
    \docmaplegend{
        Este mapa mental apresenta a estrutura completa do documento, incluindo capítulos, 
        seções e elementos específicos. Os níveis mais externos representam as divisões 
        principais do documento, enquanto os níveis internos mostram os detalhes específicos
        de cada seção.
    }
}

% Comando para simplificar a geração rápida de um mapa mental personalizado
\newcommand{\quickdocmap}[1]{%
    \section*{Mapa da Estrutura: #1}
    
    \begin{docstructmap}
        % Nó raiz personalizado
        \docmaproot{#1}
        
        % Elementos principais predefinidos (nível 1) - podem ser adaptados conforme necessário
        \docmaplevelone{introducao}{30}{Introdução}
        \docmaplevelone{desenvolvimento}{150}{Desenvolvimento}
        \docmaplevelone{conclusao}{270}{Conclusão}
    \end{docstructmap}
    
    \docmaplegend{
        Mapa mental simplificado para "#1". Este tipo de mapa pode ser usado
        para planejar novos documentos ou visualizar conceitos específicos.
    }
}


% 2. Sistema simplificado de mapas mentais (mais compatível com Overleaf):
% % Pacotes básicos
\usepackage{geometry}
\usepackage{hyperref}
\usepackage{graphicx}
\usepackage{fancyhdr}
\usepackage{titlesec}
\usepackage[usenames,dvipsnames]{xcolor}
\usepackage{microtype}
\usepackage{lipsum}
\usepackage{tcolorbox}
\usepackage{enumitem}
\usepackage{booktabs}
\usepackage{array}
\usepackage{multicol}
\usepackage{afterpage}
\usepackage{tikz}
\usepackage{setspace}
\usepackage{bookmark}
\usepackage{float}
\usepackage{lastpage}
\usepackage[framemethod=TikZ]{mdframed}
\usepackage{subfig}
\usepackage{csquotes}
\usepackage{url}
\usepackage{tabularx}
\usepackage{verbatim}
\usepackage{tocbibind}
\usepackage{newfloat}
\usepackage[useregional]{datetime2}
\usepackage{ragged2e}
\usepackage{marginnote}
\usepackage{pifont}
\usepackage{etoolbox}

% Pacotes para mapa mental/estrutura do documento
\usetikzlibrary{mindmap,trees,shadows,arrows,positioning}
\usetikzlibrary{decorations.pathmorphing}
\usetikzlibrary{decorations.markings}
\usetikzlibrary{shapes.geometric}

% Suporte a língua portuguesa
\usepackage[brazilian]{babel}

% Bibliografia
\usepackage[backend=biber, style=alphabetic]{biblatex}
\addbibresource{referencias.bib}

% Configurações de hyperlinks
\hypersetup{
    colorlinks=true,
    linkcolor=secaotitulo,
    citecolor=blue,
    urlcolor=blue
}

% Configurações de fontes usando fontspec (para XeLaTeX)
\usepackage{fontspec}
\defaultfontfeatures{Ligatures=TeX}

% Configuração de fontes para documentos RPG
\setmainfont{Latin Modern Roman}

% Definição de cores inspiradas em RPG
\definecolor{pergaminho}{RGB}{249, 240, 181}
\definecolor{capa}{RGB}{121, 26, 25}
\definecolor{titulo}{RGB}{72, 26, 19}
\definecolor{boxbg}{RGB}{253, 245, 196}
\definecolor{boxborder}{RGB}{190, 150, 86}
\definecolor{secaotitulo}{RGB}{140, 26, 20}
\definecolor{spell}{RGB}{70, 26, 100}
\definecolor{magicitem}{RGB}{26, 70, 100}
\definecolor{rule}{RGB}{140, 20, 20}
\definecolor{note}{RGB}{20, 100, 120}

% Configuração de margens
\geometry{
    a4paper,
    top=2.5cm,
    bottom=2.5cm,
    left=3cm,
    right=3cm,
    headheight=15pt,
    marginparwidth=2.5cm,
    marginparsep=0.5cm
}

% Configuração de espaçamento
\setlength{\parindent}{1.5cm}
\onehalfspacing

% Estilo de cabeçalho e rodapé
\pagestyle{fancy}
\fancyhf{}
\fancyhead[L]{\leftmark}
\fancyhead[R]{\thepage}
\fancyfoot[C]{\textit{Projeto Homebrew}}
\renewcommand{\headrulewidth}{0.4pt}
\renewcommand{\footrulewidth}{0.4pt}

% Personalização dos títulos de capítulos
\titleformat{\chapter}[display]
{\normalfont\huge\bfseries\color{secaotitulo}}
{\chaptertitlename\ \thechapter}{20pt}{\Huge}
\titlespacing*{\chapter}{0pt}{50pt}{40pt}

% Cor de fundo da página
\pagecolor{pergaminho}

% % Ambientes personalizados com tcolorbox para estilo RPG

% Configurações comuns para caixas RPG
\tcbset{
    common/.style={
        enhanced,
        frame hidden,
        interior hidden,
        colback=boxbg,
        colframe=boxborder,
        fonttitle=\bfseries\Large,
        coltitle=white,
        colbacktitle=boxborder,
        attach boxed title to top left={yshift=-2mm, xshift=5mm},
        boxed title style={sharp corners, frame hidden},
        underlay={\begin{tcbclipinterior}
            \draw[boxborder, line width=2pt] 
            (frame.south west) rectangle (frame.north east);
            \end{tcbclipinterior}},
        breakable,
        drop shadow=boxborder,
        before skip=0.5cm,
        after skip=0.5cm,
    }
}

% Estilo alternativo usando mdframed
\mdfdefinestyle{rpgstyle}{%
    linecolor=boxborder,
    linewidth=2pt,
    backgroundcolor=boxbg,
    roundcorner=5pt,
    shadow=true,
    shadowcolor=black!30,
}

% Ambiente para descrição de itens mágicos
\newtcolorbox{magicitem}[1][]{
    common,
    colbacktitle=magicitem,
    title=Item Mágico,
    #1
}

% Ambiente para feitiços
\newtcolorbox{spell}[1][]{
    common,
    colbacktitle=spell,
    title=Feitiço,
    #1
}

% Ambiente para personagens
\newtcolorbox{character}[1][]{
    common,
    colbacktitle=Brown!80!black,
    title=Personagem,
    #1
}

% Ambiente para notas do mestre
\newtcolorbox{dmnote}[1][]{
    common,
    colbacktitle=note,
    title=Nota do Mestre,
    #1
}

% Ambiente para regras
\newtcolorbox{rule}[1][]{
    common,
    colbacktitle=rule,
    title=Regra,
    #1
}

% Ambiente para tabelas
\newtcolorbox{rpgtable}[1][]{
    common,
    colbacktitle=Mahogany!70!black,
    title=Tabela,
    #1
}

% Ambiente para citações
\newtcolorbox{quotebox}[1][]{
    common,
    fonttitle=\itshape\large,
    colbacktitle=Gray!70!black,
    title=Citação,
    #1
}

% Ambiente para destaque
\newtcolorbox{highlight}[1][]{
    common,
    colbacktitle=Orange!70!black,
    title=Destaque,
    #1
}

% Comandos para estatísticas de personagem
\newcommand{\statnumber}[1]{%
    \begingroup
    \setlength{\fboxsep}{2pt}%
    \colorbox{boxbg}{\textbf{#1}}%
    \endgroup
}

\newcommand{\stat}[2]{%
    \textbf{#1} \statnumber{#2}%
}

% Comando para criar barras de atributos
\newcommand{\attrbar}[2]{%
    \begingroup
    \setlength{\unitlength}{1mm}%
    \begin{picture}(30,5)%
    \put(0,0){\color{boxborder}\rule{30mm}{5mm}}%
    \put(0,0){\color{secaotitulo}\rule{#2mm}{5mm}}%
    \put(15,2.5){\makebox(0,0)[c]{\textcolor{white}{\textbf{#1}}}}%
    \end{picture}%
    \endgroup
}

% Comando para criar listas de itens estilizadas
\newcommand{\rpgitem}[1]{%
    \item[\textcolor{secaotitulo}{\small\ding{108}}] #1
}

% Comando para titulos estilizados
\newcommand{\rpgtitle}[1]{%
    \begin{center}
        \begingroup
        \setlength{\fboxsep}{5pt}%
        \colorbox{capa}{\textcolor{white}{\Large\bfseries #1}}%
        \endgroup
    \end{center}
}

% Comando para notas de margem (anotações)
\newcommand{\rpgnote}[1]{%
    \marginpar{%
        \begin{mdframed}[style=rpgstyle]
        {\small\itshape #1}
        \end{mdframed}%
    }%
}

% Comando para adição de glossário
\newcommand{\glossaryentry}[2]{%
    \newglossaryentry{#1}{%
        name=#1,%
        description={#2}%
    }%
}

% Comando para encabezamento de seção RPG
\newcommand{\rpgsection}[1]{%
    \vspace{0.5cm}
    \begin{center}
        \begin{tikzpicture}
            \node[draw=boxborder, fill=boxbg, line width=2pt, inner sep=8pt] 
                 {\large\bfseries\color{secaotitulo} #1};
        \end{tikzpicture}
    \end{center}
    \vspace{0.3cm}
}

% Importação dos ambientes para mapa mental de estrutura do documento
% Funções automatizadas para geração de mapas mentais de estrutura de documentos

% Definição de cores para os diferentes níveis do mapa mental
\definecolor{level0color}{RGB}{121, 26, 25}  % Raiz (capa)
\definecolor{level1color}{RGB}{70, 26, 100}  % Nível 1 (spell)
\definecolor{level2color}{RGB}{26, 70, 100}  % Nível 2 (magicitem)
\definecolor{level3color}{RGB}{140, 20, 20}  % Nível 3 (rule)
\definecolor{level4color}{RGB}{20, 100, 120} % Nível 4 (note)

% Ambiente para o mapa mental da estrutura do documento
\newenvironment{docstructmap}[1][]{%
    \begin{center}
    \begin{tikzpicture}[
        mindmap,
        level 1 concept/.append style={font=\large\bfseries, sibling angle=60, level distance=5cm},
        level 2 concept/.append style={font=\normalsize\bfseries, sibling angle=45, level distance=3.5cm},
        level 3 concept/.append style={font=\small, sibling angle=40, level distance=2.5cm},
        level 4 concept/.append style={font=\scriptsize, sibling angle=30, level distance=2cm},
        concept/.append style={
            text width=4cm, 
            font=\bfseries, 
            minimum size=2cm, 
            fill=boxbg, 
            text=secaotitulo, 
            line width=1pt, 
            draw=boxborder
        },
        concept connection/.append style={line width=1pt, draw=boxborder}
    ]
}{%
    \end{tikzpicture}
    \end{center}
}

% Comando para adicionar o nó raiz do mapa mental
\newcommand{\docmaproot}[2][]{%
    \node[concept, font=\Large\bfseries, minimum size=3cm, fill=level0color!40!boxbg, text=white, #1] (docroot) {#2};
}

% Comando para adicionar um nó de nível 1
\newcommand{\docmaplevelone}[4][]{%
    % #1 = opções adicionais
    % #2 = ID do nó
    % #3 = ângulo de crescimento
    % #4 = conteúdo do nó
    \node[concept, fill=level1color!10!boxbg, #1] (#2) [grow=#3] at (docroot.#3) {#4};
    \path (docroot) to[circle connection bar] (#2);
}

% Comando para adicionar um nó de nível 2
\newcommand{\docmapleveltwo}[5][]{%
    % #1 = opções adicionais
    % #2 = ID do nó
    % #3 = ID do nó pai
    % #4 = ângulo de crescimento
    % #5 = conteúdo do nó
    \node[concept, fill=level2color!5!boxbg, #1] (#2) [grow=#4] at (#3.#4) {#5};
    \path (#3) to[circle connection bar] (#2);
}

% Comando para adicionar um nó de nível 3
\newcommand{\docmaplevelthree}[5][]{%
    % #1 = opções adicionais
    % #2 = ID do nó
    % #3 = ID do nó pai
    % #4 = ângulo de crescimento
    % #5 = conteúdo do nó
    \node[concept, scale=0.7, fill=level3color!3!boxbg, #1] (#2) [grow=#4] at (#3.#4) {#5};
    \path (#3) to[circle connection bar] (#2);
}

% Comando para adicionar um nó de nível 4
\newcommand{\docmaplevelfour}[5][]{%
    % #1 = opções adicionais
    % #2 = ID do nó
    % #3 = ID do nó pai
    % #4 = ângulo de crescimento
    % #5 = conteúdo do nó
    \node[concept, scale=0.5, fill=level4color!2!boxbg, #1] (#2) [grow=#4] at (#3.#4) {#5};
    \path (#3) to[circle connection bar] (#2);
}

% Comando para criar uma legenda para o mapa mental
\newcommand{\docmaplegend}[1][]{%
    \begin{center}
    \begin{tcolorbox}[
        colback=boxbg,
        colframe=boxborder,
        width=0.8\textwidth,
        arc=5mm,
        boxrule=1mm,
        title=Sobre este Mapa Mental
    ]
    #1
    \end{tcolorbox}
    \end{center}
}
% Implementação de um sistema automatizado para geração de mapa mental da estrutura do documento
% Este arquivo fornece funcionalidades para analisar automaticamente a estrutura do documento

% Pacotes necessários para manipulação de listas e strings
\usepackage{etoolbox}
\usepackage{xstring}

% CORREÇÕES PARA COMPATIBILIDADE COM XELATEX 2023+
% Estas correções previnem a exposição de comandos internos no documento final

% Proteções para evitar que comandos internos apareçam no documento
\makeatletter
% Este comando formata texto para evitar que variáveis internas sejam expostas
\newcommand{\protect@internal@cmd}[1]{%
  % Remove prefixos comuns de comandos internos
  \IfBeginWith{#1}{@}{}{%
    \IfBeginWith{#1}{autodocmap@}{}{%
      \IfBeginWith{#1}{sectionbaseangle}{}{%
        \IfBeginWith{#1}{subsectionbaseangle}{}{%
          \IfBeginWith{#1}{sec1textual}{}{%
            #1%
          }%
        }%
      }%
    }%
  }%
}
\makeatother

% Contador para manter controle de seções para o mapa mental
\newcounter{docmapnodecounter}
\setcounter{docmapnodecounter}{0}

% Armazenamento dos títulos das seções utilizando listas mais robustas
\def\autodocmap@sectionlist{}
\def\autodocmap@subsectionlist{}
\def\autodocmap@subsubsectionlist{}

% Contadores para o número de seções, subseções, etc.
\newcounter{sectioncount}
\newcounter{subsectioncount}
\newcounter{subsubsectioncount}
\newcounter{sectionangle}
\newcounter{subsectionangle}
\newcounter{subsubsectionangle}

% Salva os comandos originais
\let\oldsection\section
\let\oldsubsection\subsection
\let\oldsubsubsection\subsubsection
\let\oldchapter\chapter

% Define os ângulos base para cada nível como comandos protegidos
% Isso previne que esses valores apareçam no documento final
\makeatletter
\protected\def\autodocmap@sectionbaseangle{120}
\protected\def\autodocmap@subsectionbaseangle{30}
\protected\def\autodocmap@subsubsectionbaseangle{45}
\makeatother

% Armazenar informações do capítulo atual
\protected\def\autodocmap@currentchapter{}

% Redefine o comando de capítulo para rastrear títulos
\renewcommand{\chapter}[2][]{%
    \renewcommand{\autodocmap@currentchapter}{#2}%
    \oldchapter[#1]{#2}%
}

% Redefine o comando de seção para rastrear títulos
\renewcommand{\section}[2][]{%
    \stepcounter{sectioncount}%
    \setcounter{subsectioncount}{0}%
    \setcounter{sectionangle}{\value{sectioncount}}%
    \multiply\value{sectionangle} by 60%
    % Armazena os dados da seção para uso posterior no mapa
    \protected@edef\autodocmap@sectionlist{%
        \autodocmap@sectionlist
        \noexpand\docmapleveltwo{sec\arabic{sectioncount}}{textual}{\thesectionangle}{#2}%
    }%
    \oldsection[#1]{#2}%
}

% Redefine o comando de subseção para rastrear títulos
\renewcommand{\subsection}[2][]{%
    \stepcounter{subsectioncount}%
    \setcounter{subsubsectioncount}{0}%
    \setcounter{subsectionangle}{\autodocmap@subsectionbaseangle}%
    \multiply\value{subsectionangle} by \value{subsectioncount}%
    % Armazena os dados da subseção para uso posterior no mapa
    \protected@edef\autodocmap@subsectionlist{%
        \autodocmap@subsectionlist
        \noexpand\docmaplevelthree{subsec\arabic{sectioncount}_\arabic{subsectioncount}}{sec\arabic{sectioncount}}{\thesubsectionangle}{#2}%
    }%
    \oldsubsection[#1]{#2}%
}

% Redefine o comando de subsubseção para rastrear títulos
\renewcommand{\subsubsection}[2][]{%
    \stepcounter{subsubsectioncount}%
    \setcounter{subsubsectionangle}{\autodocmap@subsubsectionbaseangle}%
    \multiply\value{subsubsectionangle} by \value{subsubsectioncount}%
    % Armazena os dados da subsubseção para uso posterior no mapa
    \protected@edef\autodocmap@subsubsectionlist{%
        \autodocmap@subsubsectionlist
        \noexpand\docmaplevelfour{subsubsec\arabic{sectioncount}_\arabic{subsectioncount}_\arabic{subsubsectioncount}}{subsec\arabic{sectioncount}_\arabic{subsectioncount}}{\thesubsubsectionangle}{#2}%
    }%
    \oldsubsubsection[#1]{#2}%
}

% Comando para gerar automaticamente o mapa mental com base nos dados coletados
\newcommand{\generateautodocmap}{%
    \section*{Mapa Automático da Estrutura do Documento}
    
    \begin{docstructmap}
        % Nó raiz - Documento principal
        \docmaproot{Projeto Homebrew\\LaTeX Modular}
        
        % Elementos principais (nível 1)
        \docmaplevelone{pretextual}{30}{Elementos\\Pré-textuais}
        \docmaplevelone{textual}{150}{Elementos\\Textuais}
        \docmaplevelone{postextual}{270}{Elementos\\Pós-textuais}
        
        % Nós pré-definidos para elementos comuns (nível 2)
        \docmapleveltwo{capa}{pretextual}{0}{Capa e Título}
        \docmapleveltwo{toc}{pretextual}{90}{Sumário}
        
        \docmapleveltwo{ref}{postextual}{240}{Referências\\Bibliográficas}
        \docmapleveltwo{gloss}{postextual}{300}{Glossário}
        \docmapleveltwo{ind}{postextual}{340}{Índice Remissivo}
        
        % Gerar nós para cada seção registrada (nível 2)
        \autodocmap@sectionlist
        
        % Gerar nós para cada subseção registrada (nível 3)
        \autodocmap@subsectionlist
        
        % Gerar nós para cada subsubseção registrada (nível 4)
        \autodocmap@subsubsectionlist
        
    \end{docstructmap}
    
    \docmaplegend{
        Este mapa mental foi gerado automaticamente com base na estrutura real do documento.
        Ele representa a hierarquia de capítulos, seções e subseções no formato de um mapa mental.
        As cores indicam diferentes níveis na hierarquia do documento.
    }
}

% Comando para gerar automaticamente um mapa mental focado na estrutura atual
\newcommand{\generatesectiondocmap}[1][Capítulo Atual]{%
    \section*{Mapa da Estrutura do Capítulo}
    
    \begin{docstructmap}
        % Nó raiz - Capítulo atual
        \docmaproot{#1}
        
        % Gerar nós para cada seção registrada (nível 1)
        % Esta é uma versão simplificada que mostra apenas o capítulo atual
        % Em uma implementação mais completa, seria filtrado pelo capítulo atual
        \autodocmap@sectionlist
        
        % Gerar nós para cada subseção registrada (nível 2)
        \autodocmap@subsectionlist
        
    \end{docstructmap}
    
    \docmaplegend{
        Este mapa mental mostra a estrutura específica deste capítulo, 
        destacando suas seções e subseções principais.
    }
}

% Comando para gerar um mapa mental complexo predefinido
\newcommand{\generatecomplexdocmap}{%
    \section*{Mapa da Estrutura do Documento}
    
    \begin{docstructmap}
        % Nó raiz - Documento principal
        \docmaproot{Projeto Homebrew\\LaTeX Modular}
        
        % Elementos principais (nível 1)
        \docmaplevelone{pretextual}{30}{Elementos\\Pré-textuais}
        \docmaplevelone{textual}{150}{Elementos\\Textuais}
        \docmaplevelone{postextual}{270}{Elementos\\Pós-textuais}
        
        % Elementos pré-textuais (nível 2)
        \docmapleveltwo{capa}{pretextual}{0}{Capa e Título}
        \docmapleveltwo{toc}{pretextual}{60}{Sumário}
        \docmapleveltwo{mapament}{pretextual}{120}{Mapa Mental}
        
        % Elementos textuais - Capítulos (nível 2)
        \docmapleveltwo{cap1}{textual}{120}{Capítulo 1\\Introdução}
        \docmapleveltwo{cap2}{textual}{180}{Capítulo 2\\Bibliografia}
        \docmapleveltwo{cap3}{textual}{240}{Capítulo 3\\Glossário e Índice}
        
        % Elementos pós-textuais (nível 2)
        \docmapleveltwo{ref}{postextual}{240}{Referências\\Bibliográficas}
        \docmapleveltwo{gloss}{postextual}{300}{Glossário}
        \docmapleveltwo{ind}{postextual}{340}{Índice Remissivo}
        
        % Conteúdo do Capítulo 1 (nível 3)
        \docmaplevelthree{cap1_1}{cap1}{90}{Ambientes RPG}
        \docmaplevelthree{cap1_2}{cap1}{150}{Comandos\\Personalizados}
        \docmaplevelthree{cap1_3}{cap1}{210}{Mapas Mentais}
        
        % Conteúdo do Capítulo 2 (nível 3)
        \docmaplevelthree{cap2_1}{cap2}{150}{Sistema de Citações}
        \docmaplevelthree{cap2_2}{cap2}{210}{Referências Cruzadas}
        
        % Conteúdo do Capítulo 3 (nível 3)
        \docmaplevelthree{cap3_1}{cap3}{210}{Glossário}
        \docmaplevelthree{cap3_2}{cap3}{270}{Índice Remissivo}
        
        % Exemplos de nível 4 (detalhes mais específicos)
        \docmaplevelfour{amb1}{cap1_1}{60}{Spell}
        \docmaplevelfour{amb2}{cap1_1}{120}{Character}
        \docmaplevelfour{amb3}{cap1_1}{180}{Rule}
        
        \docmaplevelfour{cmd1}{cap1_2}{120}{rpgnote}
        \docmaplevelfour{cmd2}{cap1_2}{180}{rpgsection}
        \docmaplevelfour{cmd3}{cap1_2}{240}{rpgitem}
        
        % Conteúdo do mapa mental (nível 4)
        \docmaplevelfour{map1}{cap1_3}{120}{Estrutura\\Automática}
        \docmaplevelfour{map2}{cap1_3}{180}{Mapas\\Personalizados}
        \docmaplevelfour{map3}{cap1_3}{240}{Visualização\\Hierárquica}
    
    \end{docstructmap}
    
    \docmaplegend{
        Este mapa mental apresenta a estrutura completa do documento, incluindo capítulos, 
        seções e elementos específicos. Os níveis mais externos representam as divisões 
        principais do documento, enquanto os níveis internos mostram os detalhes específicos
        de cada seção.
    }
}

% Comando para simplificar a geração rápida de um mapa mental personalizado
\newcommand{\quickdocmap}[1]{%
    \section*{Mapa da Estrutura: #1}
    
    \begin{docstructmap}
        % Nó raiz personalizado
        \docmaproot{#1}
        
        % Elementos principais predefinidos (nível 1) - podem ser adaptados conforme necessário
        \docmaplevelone{introducao}{30}{Introdução}
        \docmaplevelone{desenvolvimento}{150}{Desenvolvimento}
        \docmaplevelone{conclusao}{270}{Conclusão}
    \end{docstructmap}
    
    \docmaplegend{
        Mapa mental simplificado para "#1". Este tipo de mapa pode ser usado
        para planejar novos documentos ou visualizar conceitos específicos.
    }
}

% % Versão simplificada dos mapas mentais para maior compatibilidade
% Este arquivo fornece uma versão mais leve e compatível com Overleaf

% Definição de cores para os diferentes níveis do mapa mental
\definecolor{level0color}{RGB}{121, 26, 25}  % Raiz (capa)
\definecolor{level1color}{RGB}{70, 26, 100}  % Nível 1 (spell)
\definecolor{level2color}{RGB}{26, 70, 100}  % Nível 2 (magicitem)
\definecolor{level3color}{RGB}{140, 20, 20}  % Nível 3 (rule)

% Ambiente para o mapa mental da estrutura do documento
\newenvironment{docstructmap}[1][]{%
    \begin{center}
    \begin{tikzpicture}[
        mindmap,
        level 1/.style={font=\large\bfseries, sibling angle=60, level distance=5cm},
        level 2/.style={font=\normalsize\bfseries, sibling angle=45, level distance=3.5cm},
        level 3/.style={font=\small, sibling angle=40, level distance=2.5cm},
        level 4/.style={font=\scriptsize, sibling angle=30, level distance=2cm},
        every node/.style={
            text width=4cm, 
            font=\bfseries, 
            minimum size=2cm, 
            fill=boxbg, 
            text=secaotitulo, 
            line width=1pt, 
            draw=boxborder
        },
        every edge/.style={line width=1pt, draw=boxborder}
    ]
}{%
    \end{tikzpicture}
    \end{center}
}

% Comando para adicionar o nó raiz do mapa mental
\newcommand{\docmaproot}[2][]{%
    \node[font=\Large\bfseries, minimum size=3cm, fill=level0color!40!boxbg, text=white, #1] (docroot) {#2};
}

% Comando para adicionar um nó de nível 1
\newcommand{\docmaplevelone}[4][]{%
    % #1 = opções adicionais
    % #2 = ID do nó
    % #3 = ângulo de crescimento
    % #4 = conteúdo do nó
    \node[fill=level1color!10!boxbg, #1] (#2) at (\thesection*#3:5cm) {#4};
    \draw[->] (docroot) -- (#2);
}

% Comando para adicionar um nó de nível 2
\newcommand{\docmapleveltwo}[5][]{%
    % #1 = opções adicionais
    % #2 = ID do nó
    % #3 = ID do nó pai
    % #4 = ângulo de crescimento
    % #5 = conteúdo do nó
    \node[fill=level2color!5!boxbg, #1] (#2) at (#3) ++(\thesection*#4:3.5cm) {#5};
    \draw[->] (#3) -- (#2);
}

% Comando para adicionar um nó de nível 3
\newcommand{\docmaplevelthree}[5][]{%
    % #1 = opções adicionais
    % #2 = ID do nó
    % #3 = ID do nó pai
    % #4 = ângulo de crescimento
    % #5 = conteúdo do nó
    \node[scale=0.7, fill=level3color!3!boxbg, #1] (#2) at (#3) ++(\thesection*#4:2.5cm) {#5};
    \draw[->] (#3) -- (#2);
}

% Comando para adicionar um nó de nível 4
\newcommand{\docmaplevelfour}[5][]{%
    % #1 = opções adicionais
    % #2 = ID do nó
    % #3 = ID do nó pai
    % #4 = ângulo de crescimento
    % #5 = conteúdo do nó
    \node[scale=0.5, fill=level3color!2!boxbg, #1] (#2) at (#3) ++(\thesection*#4:2cm) {#5};
    \draw[->] (#3) -- (#2);
}

% Comando para criar uma legenda para o mapa mental
\newcommand{\docmaplegend}[1][]{%
    \begin{center}
    \begin{tcolorbox}[
        colback=boxbg,
        colframe=boxborder,
        width=0.8\textwidth,
        arc=5mm,
        boxrule=1mm,
        title=Sobre este Mapa Mental
    ]
    #1
    \end{tcolorbox}
    \end{center}
}

% Comandos simplificados para gerar mapas mentais
\newcommand{\generateautodocmap}{%
    \section*{Mapa da Estrutura do Documento}
    
    \begin{docstructmap}
        % Nó raiz - Documento principal
        \docmaproot{Projeto Homebrew\\LaTeX Modular}
        
        % Elementos principais (nível 1)
        \docmaplevelone{pretextual}{1}{Elementos\\Pré-textuais}
        \docmaplevelone{textual}{3}{Elementos\\Textuais}
        \docmaplevelone{postextual}{5}{Elementos\\Pós-textuais}
        
        % Elementos textuais - Capítulos (nível 2)
        \docmapleveltwo{cap1}{textual}{2}{Capítulo 1\\Introdução}
        \docmapleveltwo{cap2}{textual}{3}{Capítulo 2\\Conteúdo}
        \docmapleveltwo{cap3}{textual}{4}{Capítulo 3\\Conclusão}
    \end{docstructmap}
    
    \docmaplegend{
        Este mapa mental foi gerado automaticamente e mostra a estrutura básica do documento.
        A versão simplificada é projetada para maior compatibilidade com diversos sistemas LaTeX.
    }
}

% Comando simplificado para mapa mental complexo
\newcommand{\generatecomplexdocmap}{%
    \section*{Mapa da Estrutura do Documento}
    
    \begin{docstructmap}
        % Nó raiz - Documento principal
        \docmaproot{Projeto Homebrew\\LaTeX Modular}
        
        % Elementos principais (nível 1)
        \docmaplevelone{pretextual}{1}{Elementos\\Pré-textuais}
        \docmaplevelone{textual}{3}{Elementos\\Textuais}
        \docmaplevelone{postextual}{5}{Elementos\\Pós-textuais}
        
        % Elementos pré-textuais (nível 2)
        \docmapleveltwo{capa}{pretextual}{0}{Capa e Título}
        \docmapleveltwo{toc}{pretextual}{1}{Sumário}
        \docmapleveltwo{mapament}{pretextual}{2}{Mapa Mental}
        
        % Elementos textuais - Capítulos (nível 2)
        \docmapleveltwo{cap1}{textual}{2}{Capítulo 1\\Introdução}
        \docmapleveltwo{cap2}{textual}{3}{Capítulo 2\\Conteúdo}
        \docmapleveltwo{cap3}{textual}{4}{Capítulo 3\\Conclusão}
        
        % Elementos pós-textuais (nível 2)
        \docmapleveltwo{ref}{postextual}{4}{Referências\\Bibliográficas}
        \docmapleveltwo{gloss}{postextual}{5}{Glossário}
        \docmapleveltwo{ind}{postextual}{6}{Índice Remissivo}
    \end{docstructmap}
    
    \docmaplegend{
        Este mapa mental apresenta a estrutura completa do documento, incluindo capítulos
        e elementos pré e pós-textuais. Esta versão simplificada é otimizada para 
        compatibilidade com o Overleaf.
    }
}

% Comando para mapa mental de capítulo específico
\newcommand{\generatesectiondocmap}[1][Capítulo Atual]{%
    \section*{Mapa da Estrutura do Capítulo}
    
    \begin{docstructmap}
        % Nó raiz - Capítulo atual
        \docmaproot{#1}
        
        % Seções do capítulo (nível 1)
        \docmaplevelone{sec1}{1}{Seção 1}
        \docmaplevelone{sec2}{3}{Seção 2}
        \docmaplevelone{sec3}{5}{Seção 3}
        
        % Subseções (nível 2)
        \docmapleveltwo{subsec1}{sec1}{0}{Subseção 1.1}
        \docmapleveltwo{subsec2}{sec1}{1}{Subseção 1.2}
        
        \docmapleveltwo{subsec3}{sec2}{3}{Subseção 2.1}
        \docmapleveltwo{subsec4}{sec2}{4}{Subseção 2.2}
    \end{docstructmap}
    
    \docmaplegend{
        Este mapa mental mostra a estrutura deste capítulo e suas seções.
        Use este mapa para orientar-se na leitura ou edição do capítulo.
    }
}

% Configurações adicionais para índice e glossário
\usepackage{imakeidx}         % Para criação de índice
\usepackage{glossaries}       % Para glossário
\makeindex                    % Compila o índice
\makeglossaries               % Compila o glossário

% Início do documento
\begin{document}

% Componentes pré-textuais
% Elementos pré-textuais
% Inclui capa, sumário e outros elementos iniciais

% Carrega todos os elementos pré-textuais
% Conteúdo dos elementos pré-textuais adicionais

\cleardoublepage
\chapter*{Introdução ao Projeto}

Este é um modelo modular de documento LaTeX, combinando a estética visual de livros de RPG com a organização estrutural de documentos acadêmicos. O modelo oferece uma estrutura organizada em pastas e arquivos, permitindo a edição isolada de cada componente.

\begin{quotebox}
A combinação de elementos visuais inspirados em RPG com a estrutura acadêmica cria um documento único, que pode ser utilizado para manuais de jogos, suplementos, ou até mesmo teses e dissertações com uma estética diferenciada.
\end{quotebox}

\cleardoublepage
\chapter*{Sobre este Modelo}

\begin{dmnote}
Este é um modelo base que pode e deve ser expandido e personalizado conforme as necessidades do usuário. Os arquivos estão organizados de forma modular para facilitar a manutenção e expandibilidade.
\end{dmnote}

\begin{highlight}
As partes do documento estão separadas em arquivos distintos, o que facilita a edição e manutenção. Cada capítulo está em um arquivo separado, assim como as seções pré-textuais, textuais e pós-textuais.
\end{highlight}


% Componentes textuais
% Elementos textuais
% Inclui capítulos do documento

% Carrega todos os elementos textuais
% Elementos textuais (capítulos)

% Capítulo 1
\chapter{Ambientes Personalizados}

Este capítulo apresenta os diferentes ambientes personalizados disponíveis neste modelo, inspirados no estilo visual de livros de RPG.

\section{Caixas de Destaque}

As caixas de destaque são utilizadas para ressaltar informações importantes, regras, itens mágicos, entre outros elementos.

\begin{magicitem}
\rpgtitle{Espada Flamejante}

Esta espada mágica concede ao portador a capacidade de conjurar chamas. Quando empunhada, pequenas labaredas dançam ao longo da lâmina sem causar dano ao portador.

\textbf{Propriedades:}
\begin{itemize}
    \item +1 de bônus em jogadas de ataque e dano
    \item Adiciona 1d6 de dano de fogo ao acertar
    \item Pode lançar a magia \textit{Mãos Flamejantes} uma vez por dia
\end{itemize}
\end{magicitem}

\begin{spell}
\rpgtitle{Rajada Arcana}

\textbf{Tempo de Conjuração:} 1 ação\\
\textbf{Alcance:} 36 metros\\
\textbf{Componentes:} V, S\\
\textbf{Duração:} Instantânea

Três dardos de energia mágica surgem das pontas dos seus dedos e atingem criaturas à sua escolha dentro do alcance. Cada dardo causa 1d4+1 de dano de força.

\textbf{Em Níveis Superiores:} Quando lançada usando um espaço de magia de 2º nível ou superior, a magia cria um dardo adicional para cada nível do espaço acima do 1º.
\end{spell}

\section{Estatísticas de Personagens}

Este modelo também inclui comandos para exibir estatísticas de personagens no estilo RPG:

\begin{character}
\rpgtitle{Thorian, o Sábio}

\begin{multicols}{2}
\textbf{Raça:} Humano\\
\textbf{Classe:} Mago\\
\textbf{Nível:} 5

\columnbreak

\stat{FOR}{8}\\
\stat{DES}{14}\\
\stat{CON}{12}\\
\stat{INT}{17}\\
\stat{SAB}{13}\\
\stat{CAR}{10}
\end{multicols}

\textbf{Habilidades:}
\begin{itemize}
    \item \textbf{Tradição Arcana:} Especialista em magia de evocação
    \item \textbf{Familiar:} Coruja chamada Arquimedes
    \item \textbf{Linguagens:} Comum, Élfico, Anão, Dracônico
\end{itemize}

\textbf{Vitalidade:}\\
\attrbar{25/30}{25}

\textbf{Mana:}\\
\attrbar{18/20}{18}
\end{character}

\section{Notas do Mestre}

Use o ambiente dmnote para incluir informações exclusivas para o Mestre:

\begin{dmnote}
As estatísticas dos inimigos foram balanceadas para um grupo de 4-5 personagens de nível 3-4. Ajuste conforme necessário para se adequar ao seu grupo.

Se os jogadores tentarem negociar com o líder bandido, ele pode oferecer informações sobre o culto secreto em troca de sua liberdade. No entanto, ele tentará enganar o grupo na primeira oportunidade.
\end{dmnote}

\section{Regras Especiais}

O ambiente rule é ideal para destacar regras importantes:

\begin{rule}
\rpgtitle{Descanso Curto}

Um descanso curto é um período de pelo menos 1 hora, durante o qual um personagem não faz nada mais exigente do que comer, beber, ler e tratar de ferimentos.

Um personagem pode gastar um ou mais Dados de Vida ao fim de um descanso curto, até o máximo de Dados de Vida do personagem, que é igual ao seu nível. Para cada Dado de Vida gasto, o jogador rola o dado e adiciona o modificador de Constituição do personagem. O personagem recupera pontos de vida igual ao total. O jogador pode decidir gastar um Dado de Vida adicional depois de cada rolagem.
\end{rule}


% Capítulo 2
% Conteúdo do Capítulo 2
\section{Bibliografia e Referências}

Este capítulo demonstra como trabalhar com bibliografia e referências no modelo.

\subsection{Sistema de Citações}

O modelo utiliza o pacote biblatex para gerenciamento de referências. Isso permite citar obras de forma consistente e gerar uma bibliografia formatada automaticamente.

\begin{rule}
\textbf{Citações no Texto}

Para citar uma referência no texto, utilize o comando \texttt{\\cite\{chave\}}, onde \texttt{chave} corresponde ao identificador da referência no arquivo \texttt{referencias.bib}.
\end{rule}

Por exemplo, podemos citar a obra seminal de Donald Knuth sobre tipografia digital \cite{knuth1984texbook} ou discutir as contribuições de Leslie Lamport para o LaTeX \cite{lamport1994latex}.

\rpgnote{O estilo de citação pode ser modificado alterando o parâmetro \texttt{style} na configuração do pacote biblatex em \texttt{configuracoes.tex}.}

\subsection{Gerenciando o Arquivo de Referências}

O arquivo \texttt{referencias.bib} contém todas as entradas bibliográficas no formato BibTeX. Cada entrada segue um padrão como este:

\begin{verbatim}
@book{knuth1984texbook,
  title={The TeXbook},
  author={Knuth, Donald E},
  year={1984},
  publisher={Addison-Wesley}
}
\end{verbatim}

\begin{dmnote}
Recomenda-se utilizar ferramentas como JabRef, Zotero ou Mendeley para gerenciar suas referências bibliográficas. Estas ferramentas permitem exportar entradas no formato BibTeX diretamente para o arquivo \texttt{referencias.bib}.
\end{dmnote}

\subsection{Tipos de Citações}

Existem diferentes formas de citar referências:

\begin{rpgtable}
\begin{tabular}{|l|l|p{8cm}|}
\hline
\textbf{Comando} & \textbf{Exemplo} & \textbf{Resultado} \\
\hline
\texttt{\\cite\{chave\}} & \texttt{\\cite\{lamport1994latex\}} & Citação padrão \cite{lamport1994latex} \\
\hline
\texttt{\\parencite\{chave\}} & \texttt{\\parencite\{knuth1984texbook\}} & Citação entre parênteses \parencite{knuth1984texbook} \\
\hline
\texttt{\\textcite\{chave\}} & \texttt{\\textcite\{lamport1994latex\}} & Citação no texto \textcite{lamport1994latex} \\
\hline
\texttt{\\footcite\{chave\}} & \texttt{\\footcite\{knuth1984texbook\}} & Citação em nota de rodapé\footcite{knuth1984texbook} \\
\hline
\end{tabular}
\end{rpgtable}

\subsection{Referências Cruzadas}

Além de referências bibliográficas, o LaTeX permite criar referências cruzadas dentro do seu próprio documento.

\begin{spell}
\textbf{Mecanismo de Referências Cruzadas}\\
\textit{*Feature de nível 3*}

Para criar uma referência cruzada, primeiro rotule um elemento com \texttt{\\label\{identificador\}} e depois referencie-o usando \texttt{\\ref\{identificador\}}.
\end{spell}

Por exemplo, podemos rotular esta seção com \label{sec:refcruzadas} e referenciá-la como "Seção \ref{sec:refcruzadas}".

\begin{highlight}
\textbf{Dica importante:} 
Sempre compile seu documento pelo menos duas vezes para garantir que todas as referências cruzadas sejam resolvidas corretamente. Na primeira compilação, o LaTeX identifica os rótulos e suas posições; na segunda, ele substitui as referências pelos números corretos.
\end{highlight}

\begin{quotebox}
"O verdadeiro poder do LaTeX está em sua capacidade de gerenciar automaticamente aspectos como numeração, referências e formatação bibliográfica, permitindo que o autor se concentre exclusivamente no conteúdo."

— Um entusiasta do LaTeX
\end{quotebox}

% Capítulo 3
% Conteúdo do Capítulo 3
\section{Glossário e Índice Remissivo}

Esta seção demonstra como utilizar o glossário e o índice remissivo no projeto.

\subsection{Utilizando o Glossário}

O glossário permite criar uma lista de termos importantes e suas definições. Para adicionar um termo ao glossário, use o comando:

\begin{verbatim}
\glossaryentry{termo}{definição}
\end{verbatim}

Exemplo de termos adicionados ao glossário:

\glossaryentry{arcano}{Termo que se refere à magia ou conhecimento místico.}
\glossaryentry{grimório}{Livro contendo feitiços, encantamentos e instruções mágicas.}
\glossaryentry{runa}{Símbolo mágico utilizado em rituais e encantamentos.}

Para referenciar um termo do glossário no texto, use \verb|\gls{termo}|. Por exemplo: 
\gls{arcano}, \gls{grimório} e \gls{runa}.

\subsection{Utilizando o Índice Remissivo}

O índice remissivo permite que os leitores encontrem facilmente termos específicos no documento. Para adicionar um termo ao índice, use:

\begin{verbatim}
\index{termo}
\end{verbatim}

Exemplos de termos indexados:

Magos\index{magos} são estudantes de magia\index{magia} arcana. Eles aprendem feitiços\index{feitiço} 
através de estudo intenso e prática constante. Muitos magos mantêm grimórios\index{grimório} 
onde registram seus conhecimentos arcanos\index{arcano}.

Druidas\index{druida} são praticantes de magia\index{magia!natural} natural. Eles canalizam a 
energia da natureza\index{natureza} e podem assumir formas animais\index{forma animal}.

\subsection{Demonstração de Ambientes}

\begin{spell}
\textbf{Proteção Arcana}\\
\textit{Abjuração de 2º nível}\\
\textbf{Tempo de Conjuração:} 1 ação\\
\textbf{Alcance:} Toque\\
\textbf{Componentes:} V, S, M (um pequeno diamante)\\
\textbf{Duração:} 1 hora

Você toca uma criatura disposta e cria uma barreira mágica ao seu redor. Até o fim da duração, o alvo recebe +2 na CA e tem vantagem em testes de resistência contra magias.
\end{spell}

\rpgnote{Esta magia é especialmente útil antes de enfrentar criaturas que utilizam magias ofensivas.}

\begin{magicitem}
\textbf{Amuleto de Proteção}\\
\textit{Item maravilhoso, raro (requer sintonização)}

Enquanto estiver usando este amuleto, você recebe +1 nas jogadas de resistência e é imune a magias de adivinhação e efeitos sensoriais mágicos que detectariam sua presença.
\end{magicitem}

\begin{dmnote}
Considere dar este item a personagens que enfrentarão inimigos com forte capacidade de rastreamento mágico ou que precisam realizar missões furtivas.
\end{dmnote}

\rpgsection{Lista de Atributos e Estatísticas}

Aqui demonstramos como apresentar estatísticas de personagens usando os comandos personalizados:

\stat{Força}{16}  \stat{Destreza}{14}  \stat{Constituição}{15}  

\stat{Inteligência}{18}  \stat{Sabedoria}{12}  \stat{Carisma}{10}

Barras de atributos também podem ser usadas para representar níveis de habilidade:

\attrbar{Conjuração}{25}

\attrbar{Combate}{15}

\attrbar{Furtividade}{10}

\begin{rule}
\textbf{Sintonização com Itens Mágicos}

Para sintonizar-se com um item, você deve passar 1 hora em contato físico ininterrupto com ele, concentrando-se nele e tentando compreender suas propriedades. Esta concentração pode ocorrer durante um descanso curto e não pode ser interrompida por nenhuma outra atividade.
\end{rule}

\rpgtitle{Tabela de Encontros Aleatórios}

\begin{rpgtable}
\begin{tabular}{|c|p{10cm}|}
\hline
\textbf{d20} & \textbf{Encontro} \\
\hline
1-3 & 1d4 bandidos tentando roubar viajantes \\
\hline
4-6 & Um mercador com carroça quebrada pedindo ajuda \\
\hline
7-10 & 1d6 lobos caçando na área \\
\hline
11-14 & Um druida realizando um ritual para purificar a terra \\
\hline
15-17 & Uma patrulha de 1d4+1 guardas da cidade \\
\hline
18-19 & Um mago errante estudando fenômenos mágicos locais \\
\hline
20 & Um dragão jovem sobrevoando a região \\
\hline
\end{tabular}
\end{rpgtable}

\begin{quotebox}
"A verdadeira magia consiste em compreender a conexão entre todas as coisas, visíveis e invisíveis. Não é apenas poder, mas sabedoria."

— Arquimago Elminster
\end{quotebox}

\begin{highlight}
\textbf{Dica importante para jogadores:} 
Sempre verifique portas, baús e outros objetos suspeitos em busca de armadilhas antes de interagir com eles. Uma simples verificação pode salvar sua vida!
\end{highlight}

% Capítulo sobre Mapas Mentais
% Capítulo especializado sobre Mapas Mentais
\section{Mapas Mentais para Visualização da Estrutura do Documento}

O Projeto Homebrew inclui um sistema completo para geração de mapas mentais para visualizar a estrutura do documento. Este capítulo apresenta exemplos práticos e diretrizes para o uso efetivo desses mapas.

\subsection{O Que São Mapas Mentais de Estrutura}

Os mapas mentais são representações visuais que organizam informações de forma hierárquica, radiando a partir de um conceito central. No contexto da estruturação de documentos, eles ajudam a visualizar a organização geral e as relações entre diferentes partes do texto.

\rpgnote{Um mapa mental bem construído pode servir tanto como guia inicial para o leitor quanto como ferramenta de planejamento para o autor.}

\begin{dmnote}
Os mapas mentais são especialmente úteis em documentos complexos com muitos níveis hierárquicos, como teses, dissertações e livros técnicos.
\end{dmnote}

\subsection{Tipos de Mapas Mentais no Projeto Homebrew}

O sistema oferece três abordagens principais para criação de mapas mentais:

\begin{itemize}
\rpgitem{Mapas pré-definidos, que seguem uma estrutura padrão de documento acadêmico}
\rpgitem{Mapas automáticos, gerados a partir da estrutura real do documento}
\rpgitem{Mapas personalizados, criados manualmente para visualizar conceitos específicos}
\end{itemize}

\subsubsection{Mapas Pré-definidos}

Para incluir um mapa mental pré-definido em seu documento, basta utilizar o comando \texttt{\\generatecomplexdocmap}:

\begin{spell}
\textbf{Exemplo de Código}

\begin{verbatim}
% No arquivo .tex:
\clearpage
\generatecomplexdocmap
\clearpage
\end{verbatim}

Este comando gerará um mapa mental detalhado com a estrutura típica de um documento acadêmico.
\end{spell}

\subsubsection{Mapas Automáticos}

Os mapas automáticos são mais dinâmicos e são gerados a partir da estrutura real do documento:

\begin{spell}
\textbf{Exemplo de Código para Mapa Automático}

\begin{verbatim}
% No arquivo .tex:
\clearpage
\generateautodocmap
\clearpage
\end{verbatim}

Este comando analisa a estrutura atual do documento (suas seções, subseções, etc.) e cria um mapa visual correspondente.
\end{spell}

\rpgnote{Para que o mapa automático funcione corretamente, ele deve ser colocado no documento após as seções que deseja mapear.}

\subsubsection{Mapas Personalizados}

Para casos mais específicos, você pode criar mapas mentais personalizados:

\begin{spell}
\textbf{Exemplo de Código para Mapa Personalizado}

\begin{verbatim}
\begin{docstructmap}
    % Nó raiz
    \docmaproot{Título Principal}
    
    % Nós de primeiro nível
    \docmaplevelone{id1}{30}{Tópico 1}
    \docmaplevelone{id2}{150}{Tópico 2}
    \docmaplevelone{id3}{270}{Tópico 3}
    
    % Nós de segundo nível
    \docmapleveltwo{id1_1}{id1}{0}{Subtópico 1.1}
    \docmapleveltwo{id2_1}{id2}{150}{Subtópico 2.1}
\end{docstructmap}

\docmaplegend{
    Texto explicativo sobre este mapa mental.
}
\end{verbatim}
\end{spell}

\subsection{Exemplos Práticos}

\subsubsection{Mapa Mental para Planejamento de Tese}

Um exemplo de uso prático é o planejamento visual de uma tese:

\begin{docstructmap}
    % Nó raiz
    \docmaproot{Tese de\\Doutorado}
    
    % Capítulos principais (nível 1)
    \docmaplevelone{introducao}{30}{Introdução}
    \docmaplevelone{revisao}{90}{Revisão de\\Literatura}
    \docmaplevelone{metodologia}{150}{Metodologia}
    \docmaplevelone{resultados}{210}{Resultados}
    \docmaplevelone{discussao}{270}{Discussão}
    \docmaplevelone{conclusao}{330}{Conclusão}
    
    % Elementos da Introdução (nível 2)
    \docmapleveltwo{contexto}{introducao}{0}{Contextualização}
    \docmapleveltwo{problema}{introducao}{40}{Problema de\\Pesquisa}
    \docmapleveltwo{objetivos}{introducao}{80}{Objetivos}
    
    % Elementos da Metodologia (nível 2)
    \docmapleveltwo{desenho}{metodologia}{130}{Desenho do\\Estudo}
    \docmapleveltwo{amostra}{metodologia}{150}{Amostragem}
    \docmapleveltwo{analise}{metodologia}{170}{Análise de\\Dados}
    
    % Elementos do Resultado (nível 2)
    \docmapleveltwo{res1}{resultados}{190}{Resultado 1}
    \docmapleveltwo{res2}{resultados}{210}{Resultado 2}
    \docmapleveltwo{res3}{resultados}{230}{Resultado 3}
    
    % Elementos de um resultado específico (nível 3)
    \docmaplevelthree{res1_1}{res1}{180}{Achado 1.1}
    \docmaplevelthree{res1_2}{res1}{200}{Achado 1.2}
\end{docstructmap}

\docmaplegend{
    Este mapa mental ilustra a estrutura típica de uma tese de doutorado, destacando os capítulos 
    principais e detalhando especialmente as seções de Introdução, Metodologia e Resultados.
}

\subsubsection{Mapa Mental para Visualização de Conceitos}

Os mapas mentais também podem ser usados para visualizar conceitos teóricos:

\begin{docstructmap}
    % Nó raiz
    \docmaproot{Teoria da\\Aprendizagem}
    
    % Principais teorias (nível 1)
    \docmaplevelone{comportamental}{30}{Comportamentalismo}
    \docmaplevelone{cognitiva}{150}{Cognitivismo}
    \docmaplevelone{construtivista}{270}{Construtivismo}
    
    % Teóricos do comportamentalismo (nível 2)
    \docmapleveltwo{pavlov}{comportamental}{0}{Pavlov}
    \docmapleveltwo{skinner}{comportamental}{60}{Skinner}
    
    % Conceitos do cognitivismo (nível 2)
    \docmapleveltwo{esquemas}{cognitiva}{120}{Esquemas\\Mentais}
    \docmapleveltwo{memoria}{cognitiva}{180}{Memória\\de Trabalho}
    
    % Vertentes do construtivismo (nível 2)
    \docmapleveltwo{piaget}{construtivista}{240}{Piaget}
    \docmapleveltwo{vygotsky}{construtivista}{300}{Vygotsky}
    
    % Conceitos específicos (nível 3)
    \docmaplevelthree{condclass}{pavlov}{-15}{Condicionamento\\Clássico}
    \docmaplevelthree{condoper}{skinner}{75}{Condicionamento\\Operante}
\end{docstructmap}

\docmaplegend{
    Este mapa mental apresenta as principais teorias da aprendizagem, seus principais 
    representantes e alguns conceitos centrais de cada abordagem teórica.
}

\subsection{Dicas para Criação de Mapas Mentais Efetivos}

\begin{rule}
\textbf{Princípios para Criação de Mapas Mentais Efetivos}

\begin{enumerate}
    \item \textbf{Hierarquia clara:} Mantenha uma estrutura hierárquica bem definida
    \item \textbf{Concisão:} Use palavras-chave ou frases curtas nos nós
    \item \textbf{Equilíbrio visual:} Distribua os nós de forma equilibrada
    \item \textbf{Profundidade adequada:} Limite-se a 3-4 níveis para evitar poluição visual
    \item \textbf{Legendas explicativas:} Inclua sempre uma legenda explicativa
\end{enumerate}
\end{rule}

\rpgnote{Para mapas muito complexos, considere dividir em múltiplos mapas menores, cada um focando em uma parte específica da estrutura.}

\subsection{Integração com o Fluxo do Documento}

Os mapas mentais podem ser integrados em diferentes pontos do documento:

\begin{itemize}
\rpgitem{No início, como visão geral da estrutura completa}
\rpgitem{No início de cada capítulo, mostrando a estrutura específica daquela seção}
\rpgitem{Em apêndices, para visualizar conceitos complexos}
\end{itemize}

\begin{quotebox}
"Um mapa mental bem construído é como um mapa de navegação para o leitor. Ele mostra não apenas onde cada conteúdo está localizado, mas também como os diferentes elementos se relacionam entre si, criando um entendimento holístico da obra."
\end{quotebox}

\subsection{Caso de Uso: Mapa Mental para Planejamento de Escrita}

Um uso particularmente valioso dos mapas mentais é o planejamento do processo de escrita:

\begin{docstructmap}
    % Nó raiz
    \docmaproot{Processo de\\Escrita}
    
    % Fases principais (nível 1)
    \docmaplevelone{planejamento}{30}{Planejamento}
    \docmaplevelone{primeira}{150}{Primeira\\Versão}
    \docmaplevelone{revisao}{270}{Revisão e\\Finalização}
    
    % Elementos do planejamento (nível 2)
    \docmapleveltwo{tema}{planejamento}{0}{Definição\\do Tema}
    \docmapleveltwo{pesquisa}{planejamento}{60}{Pesquisa\\Bibliográfica}
    
    % Elementos da primeira versão (nível 2)
    \docmapleveltwo{rascunho}{primeira}{120}{Rascunho\\Inicial}
    \docmapleveltwo{expansao}{primeira}{180}{Expansão\\de Ideias}
    
    % Elementos da revisão (nível 2)
    \docmapleveltwo{revisao1}{revisao}{240}{Revisão\\de Conteúdo}
    \docmapleveltwo{revisao2}{revisao}{300}{Revisão\\de Forma}
    
    % Elementos específicos (nível 3)
    \docmaplevelthree{bib}{pesquisa}{30}{Seleção de\\Bibliografia}
    \docmaplevelthree{notas}{pesquisa}{90}{Tomada\\de Notas}
    
    \docmaplevelthree{gramatical}{revisao2}{270}{Revisão\\Gramatical}
    \docmaplevelthree{formatacao}{revisao2}{330}{Formatação\\e Estilo}
\end{docstructmap}

\docmaplegend{
    Este mapa mental apresenta o processo de escrita acadêmica, destacando as principais
    fases e atividades em cada etapa. Pode ser usado como guia para organizar o trabalho
    de redação de um documento acadêmico.
}

\section{Conclusão}

Os mapas mentais são uma ferramenta poderosa para visualização da estrutura do documento, planejamento da escrita e organização de conceitos complexos. O sistema implementado no Projeto Homebrew oferece flexibilidade para criar diferentes tipos de mapas, desde estruturas predefinidas até visualizações completamente personalizadas.

\begin{highlight}
Para mais detalhes sobre a implementação técnica e opções avançadas de personalização, consulte o arquivo de documentação \texttt{docs/tutorial-mapamental.md} e os exemplos em \texttt{exemplos/exemplo-mapamental.tex}.
\end{highlight}



% Componentes pós-textuais
% Elementos pós-textuais

% Carrega todos os elementos pós-textuais
% Conteúdo dos elementos pós-textuais

\cleardoublepage
\chapter*{Conclusão}

Este documento demonstra as capacidades do modelo modular LaTeX com estética de RPG. Combine a organização de documentos acadêmicos com o estilo visual dos livros de RPG para criar documentos únicos e interessantes.

\cleardoublepage
\chapter*{Glossário}

\begin{description}[style=nextline, leftmargin=0cm]
    \item[LaTeX] Sistema de tipografia que permite a criação de documentos com alta qualidade tipográfica.
    \item[RPG] Role-Playing Game, jogo onde os jogadores assumem os papéis de personagens em um cenário fictício.
    \item[Homebrewery] Estilo de formatação inspirado em livros de RPG, especialmente Dungeons \& Dragons.
    \item[Modular] Estrutura organizada em componentes independentes que podem ser combinados.
\end{description}

\cleardoublepage
\chapter*{Apêndices}

\begin{rpgtable}
\begin{tabular}{lcc}
\toprule
\textbf{Item} & \textbf{Categoria} & \textbf{Valor} \\
\midrule
Espada Longa & Arma & 15 po \\
Poção de Cura & Consumível & 50 po \\
Grimório & Mágico & 100 po \\
\bottomrule
\end{tabular}
\end{rpgtable}

\cleardoublepage
\printbibliography[title=Referências]


% Imprime o glossário
\makeglossaries


% Imprimir o glossário
\printglossary[title=Glossário de Termos]

% Imprimir o índice remissivo
\printindex

\end{document}
