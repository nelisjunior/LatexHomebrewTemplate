% Exemplo de uso de mapas mentais em contexto de RPG
% Este arquivo demonstra como usar a funcionalidade de mapas mentais para visualizar
% estruturas de documentos relacionados a RPG, como aventuras, descrição de mundos, etc.

\documentclass[12pt, a4paper]{book}

% Carrega as configurações e ambientes personalizados
% Pacotes básicos
\usepackage{geometry}
\usepackage{hyperref}
\usepackage{graphicx}
\usepackage{fancyhdr}
\usepackage{titlesec}
\usepackage[usenames,dvipsnames]{xcolor}
\usepackage{microtype}
\usepackage{lipsum}
\usepackage{tcolorbox}
\usepackage{enumitem}
\usepackage{booktabs}
\usepackage{array}
\usepackage{multicol}
\usepackage{afterpage}
\usepackage{tikz}
\usepackage{setspace}
\usepackage{bookmark}
\usepackage{float}
\usepackage{lastpage}
\usepackage[framemethod=TikZ]{mdframed}
\usepackage{subfig}
\usepackage{csquotes}
\usepackage{url}
\usepackage{tabularx}
\usepackage{verbatim}
\usepackage{tocbibind}
\usepackage{newfloat}
\usepackage[useregional]{datetime2}
\usepackage{ragged2e}
\usepackage{marginnote}
\usepackage{pifont}
\usepackage{etoolbox}

% Pacotes para mapa mental/estrutura do documento
\usetikzlibrary{mindmap,trees,shadows,arrows,positioning}
\usetikzlibrary{decorations.pathmorphing}
\usetikzlibrary{decorations.markings}
\usetikzlibrary{shapes.geometric}

% Suporte a língua portuguesa
\usepackage[brazilian]{babel}

% Bibliografia
\usepackage[backend=biber, style=alphabetic]{biblatex}
\addbibresource{referencias.bib}

% Configurações de hyperlinks
\hypersetup{
    colorlinks=true,
    linkcolor=secaotitulo,
    citecolor=blue,
    urlcolor=blue
}

% Configurações de fontes usando fontspec (para XeLaTeX)
\usepackage{fontspec}
\defaultfontfeatures{Ligatures=TeX}

% Configuração de fontes para documentos RPG
\setmainfont{Latin Modern Roman}

% Definição de cores inspiradas em RPG
\definecolor{pergaminho}{RGB}{249, 240, 181}
\definecolor{capa}{RGB}{121, 26, 25}
\definecolor{titulo}{RGB}{72, 26, 19}
\definecolor{boxbg}{RGB}{253, 245, 196}
\definecolor{boxborder}{RGB}{190, 150, 86}
\definecolor{secaotitulo}{RGB}{140, 26, 20}
\definecolor{spell}{RGB}{70, 26, 100}
\definecolor{magicitem}{RGB}{26, 70, 100}
\definecolor{rule}{RGB}{140, 20, 20}
\definecolor{note}{RGB}{20, 100, 120}

% Configuração de margens
\geometry{
    a4paper,
    top=2.5cm,
    bottom=2.5cm,
    left=3cm,
    right=3cm,
    headheight=15pt,
    marginparwidth=2.5cm,
    marginparsep=0.5cm
}

% Configuração de espaçamento
\setlength{\parindent}{1.5cm}
\onehalfspacing

% Estilo de cabeçalho e rodapé
\pagestyle{fancy}
\fancyhf{}
\fancyhead[L]{\leftmark}
\fancyhead[R]{\thepage}
\fancyfoot[C]{\textit{Projeto Homebrew}}
\renewcommand{\headrulewidth}{0.4pt}
\renewcommand{\footrulewidth}{0.4pt}

% Personalização dos títulos de capítulos
\titleformat{\chapter}[display]
{\normalfont\huge\bfseries\color{secaotitulo}}
{\chaptertitlename\ \thechapter}{20pt}{\Huge}
\titlespacing*{\chapter}{0pt}{50pt}{40pt}

% Cor de fundo da página
\pagecolor{pergaminho}

% Ambientes personalizados com tcolorbox para estilo RPG

% Configurações comuns para caixas RPG
\tcbset{
    common/.style={
        enhanced,
        frame hidden,
        interior hidden,
        colback=boxbg,
        colframe=boxborder,
        fonttitle=\bfseries\Large,
        coltitle=white,
        colbacktitle=boxborder,
        attach boxed title to top left={yshift=-2mm, xshift=5mm},
        boxed title style={sharp corners, frame hidden},
        underlay={\begin{tcbclipinterior}
            \draw[boxborder, line width=2pt] 
            (frame.south west) rectangle (frame.north east);
            \end{tcbclipinterior}},
        breakable,
        drop shadow=boxborder,
        before skip=0.5cm,
        after skip=0.5cm,
    }
}

% Estilo alternativo usando mdframed
\mdfdefinestyle{rpgstyle}{%
    linecolor=boxborder,
    linewidth=2pt,
    backgroundcolor=boxbg,
    roundcorner=5pt,
    shadow=true,
    shadowcolor=black!30,
}

% Ambiente para descrição de itens mágicos
\newtcolorbox{magicitem}[1][]{
    common,
    colbacktitle=magicitem,
    title=Item Mágico,
    #1
}

% Ambiente para feitiços
\newtcolorbox{spell}[1][]{
    common,
    colbacktitle=spell,
    title=Feitiço,
    #1
}

% Ambiente para personagens
\newtcolorbox{character}[1][]{
    common,
    colbacktitle=Brown!80!black,
    title=Personagem,
    #1
}

% Ambiente para notas do mestre
\newtcolorbox{dmnote}[1][]{
    common,
    colbacktitle=note,
    title=Nota do Mestre,
    #1
}

% Ambiente para regras
\newtcolorbox{rule}[1][]{
    common,
    colbacktitle=rule,
    title=Regra,
    #1
}

% Ambiente para tabelas
\newtcolorbox{rpgtable}[1][]{
    common,
    colbacktitle=Mahogany!70!black,
    title=Tabela,
    #1
}

% Ambiente para citações
\newtcolorbox{quotebox}[1][]{
    common,
    fonttitle=\itshape\large,
    colbacktitle=Gray!70!black,
    title=Citação,
    #1
}

% Ambiente para destaque
\newtcolorbox{highlight}[1][]{
    common,
    colbacktitle=Orange!70!black,
    title=Destaque,
    #1
}

% Comandos para estatísticas de personagem
\newcommand{\statnumber}[1]{%
    \begingroup
    \setlength{\fboxsep}{2pt}%
    \colorbox{boxbg}{\textbf{#1}}%
    \endgroup
}

\newcommand{\stat}[2]{%
    \textbf{#1} \statnumber{#2}%
}

% Comando para criar barras de atributos
\newcommand{\attrbar}[2]{%
    \begingroup
    \setlength{\unitlength}{1mm}%
    \begin{picture}(30,5)%
    \put(0,0){\color{boxborder}\rule{30mm}{5mm}}%
    \put(0,0){\color{secaotitulo}\rule{#2mm}{5mm}}%
    \put(15,2.5){\makebox(0,0)[c]{\textcolor{white}{\textbf{#1}}}}%
    \end{picture}%
    \endgroup
}

% Comando para criar listas de itens estilizadas
\newcommand{\rpgitem}[1]{%
    \item[\textcolor{secaotitulo}{\small\ding{108}}] #1
}

% Comando para titulos estilizados
\newcommand{\rpgtitle}[1]{%
    \begin{center}
        \begingroup
        \setlength{\fboxsep}{5pt}%
        \colorbox{capa}{\textcolor{white}{\Large\bfseries #1}}%
        \endgroup
    \end{center}
}

% Comando para notas de margem (anotações)
\newcommand{\rpgnote}[1]{%
    \marginpar{%
        \begin{mdframed}[style=rpgstyle]
        {\small\itshape #1}
        \end{mdframed}%
    }%
}

% Comando para adição de glossário
\newcommand{\glossaryentry}[2]{%
    \newglossaryentry{#1}{%
        name=#1,%
        description={#2}%
    }%
}

% Comando para encabezamento de seção RPG
\newcommand{\rpgsection}[1]{%
    \vspace{0.5cm}
    \begin{center}
        \begin{tikzpicture}
            \node[draw=boxborder, fill=boxbg, line width=2pt, inner sep=8pt] 
                 {\large\bfseries\color{secaotitulo} #1};
        \end{tikzpicture}
    \end{center}
    \vspace{0.3cm}
}

% Importação dos ambientes para mapa mental de estrutura do documento
% Funções automatizadas para geração de mapas mentais de estrutura de documentos

% Definição de cores para os diferentes níveis do mapa mental
\definecolor{level0color}{RGB}{121, 26, 25}  % Raiz (capa)
\definecolor{level1color}{RGB}{70, 26, 100}  % Nível 1 (spell)
\definecolor{level2color}{RGB}{26, 70, 100}  % Nível 2 (magicitem)
\definecolor{level3color}{RGB}{140, 20, 20}  % Nível 3 (rule)
\definecolor{level4color}{RGB}{20, 100, 120} % Nível 4 (note)

% Ambiente para o mapa mental da estrutura do documento
\newenvironment{docstructmap}[1][]{%
    \begin{center}
    \begin{tikzpicture}[
        mindmap,
        level 1 concept/.append style={font=\large\bfseries, sibling angle=60, level distance=5cm},
        level 2 concept/.append style={font=\normalsize\bfseries, sibling angle=45, level distance=3.5cm},
        level 3 concept/.append style={font=\small, sibling angle=40, level distance=2.5cm},
        level 4 concept/.append style={font=\scriptsize, sibling angle=30, level distance=2cm},
        concept/.append style={
            text width=4cm, 
            font=\bfseries, 
            minimum size=2cm, 
            fill=boxbg, 
            text=secaotitulo, 
            line width=1pt, 
            draw=boxborder
        },
        concept connection/.append style={line width=1pt, draw=boxborder}
    ]
}{%
    \end{tikzpicture}
    \end{center}
}

% Comando para adicionar o nó raiz do mapa mental
\newcommand{\docmaproot}[2][]{%
    \node[concept, font=\Large\bfseries, minimum size=3cm, fill=level0color!40!boxbg, text=white, #1] (docroot) {#2};
}

% Comando para adicionar um nó de nível 1
\newcommand{\docmaplevelone}[4][]{%
    % #1 = opções adicionais
    % #2 = ID do nó
    % #3 = ângulo de crescimento
    % #4 = conteúdo do nó
    \node[concept, fill=level1color!10!boxbg, #1] (#2) [grow=#3] at (docroot.#3) {#4};
    \path (docroot) to[circle connection bar] (#2);
}

% Comando para adicionar um nó de nível 2
\newcommand{\docmapleveltwo}[5][]{%
    % #1 = opções adicionais
    % #2 = ID do nó
    % #3 = ID do nó pai
    % #4 = ângulo de crescimento
    % #5 = conteúdo do nó
    \node[concept, fill=level2color!5!boxbg, #1] (#2) [grow=#4] at (#3.#4) {#5};
    \path (#3) to[circle connection bar] (#2);
}

% Comando para adicionar um nó de nível 3
\newcommand{\docmaplevelthree}[5][]{%
    % #1 = opções adicionais
    % #2 = ID do nó
    % #3 = ID do nó pai
    % #4 = ângulo de crescimento
    % #5 = conteúdo do nó
    \node[concept, scale=0.7, fill=level3color!3!boxbg, #1] (#2) [grow=#4] at (#3.#4) {#5};
    \path (#3) to[circle connection bar] (#2);
}

% Comando para adicionar um nó de nível 4
\newcommand{\docmaplevelfour}[5][]{%
    % #1 = opções adicionais
    % #2 = ID do nó
    % #3 = ID do nó pai
    % #4 = ângulo de crescimento
    % #5 = conteúdo do nó
    \node[concept, scale=0.5, fill=level4color!2!boxbg, #1] (#2) [grow=#4] at (#3.#4) {#5};
    \path (#3) to[circle connection bar] (#2);
}

% Comando para criar uma legenda para o mapa mental
\newcommand{\docmaplegend}[1][]{%
    \begin{center}
    \begin{tcolorbox}[
        colback=boxbg,
        colframe=boxborder,
        width=0.8\textwidth,
        arc=5mm,
        boxrule=1mm,
        title=Sobre este Mapa Mental
    ]
    #1
    \end{tcolorbox}
    \end{center}
}
% Implementação de um sistema automatizado para geração de mapa mental da estrutura do documento
% Este arquivo fornece funcionalidades para analisar automaticamente a estrutura do documento

% Pacotes necessários para manipulação de listas e strings
\usepackage{etoolbox}
\usepackage{xstring}

% CORREÇÕES PARA COMPATIBILIDADE COM XELATEX 2023+
% Estas correções previnem a exposição de comandos internos no documento final

% Proteções para evitar que comandos internos apareçam no documento
\makeatletter
% Este comando formata texto para evitar que variáveis internas sejam expostas
\newcommand{\protect@internal@cmd}[1]{%
  % Remove prefixos comuns de comandos internos
  \IfBeginWith{#1}{@}{}{%
    \IfBeginWith{#1}{autodocmap@}{}{%
      \IfBeginWith{#1}{sectionbaseangle}{}{%
        \IfBeginWith{#1}{subsectionbaseangle}{}{%
          \IfBeginWith{#1}{sec1textual}{}{%
            #1%
          }%
        }%
      }%
    }%
  }%
}
\makeatother

% Contador para manter controle de seções para o mapa mental
\newcounter{docmapnodecounter}
\setcounter{docmapnodecounter}{0}

% Armazenamento dos títulos das seções utilizando listas mais robustas
\def\autodocmap@sectionlist{}
\def\autodocmap@subsectionlist{}
\def\autodocmap@subsubsectionlist{}

% Contadores para o número de seções, subseções, etc.
\newcounter{sectioncount}
\newcounter{subsectioncount}
\newcounter{subsubsectioncount}
\newcounter{sectionangle}
\newcounter{subsectionangle}
\newcounter{subsubsectionangle}

% Salva os comandos originais
\let\oldsection\section
\let\oldsubsection\subsection
\let\oldsubsubsection\subsubsection
\let\oldchapter\chapter

% Define os ângulos base para cada nível como comandos protegidos
% Isso previne que esses valores apareçam no documento final
\makeatletter
\protected\def\autodocmap@sectionbaseangle{120}
\protected\def\autodocmap@subsectionbaseangle{30}
\protected\def\autodocmap@subsubsectionbaseangle{45}
\makeatother

% Armazenar informações do capítulo atual
\protected\def\autodocmap@currentchapter{}

% Redefine o comando de capítulo para rastrear títulos
\renewcommand{\chapter}[2][]{%
    \renewcommand{\autodocmap@currentchapter}{#2}%
    \oldchapter[#1]{#2}%
}

% Redefine o comando de seção para rastrear títulos
\renewcommand{\section}[2][]{%
    \stepcounter{sectioncount}%
    \setcounter{subsectioncount}{0}%
    \setcounter{sectionangle}{\value{sectioncount}}%
    \multiply\value{sectionangle} by 60%
    % Armazena os dados da seção para uso posterior no mapa
    \protected@edef\autodocmap@sectionlist{%
        \autodocmap@sectionlist
        \noexpand\docmapleveltwo{sec\arabic{sectioncount}}{textual}{\thesectionangle}{#2}%
    }%
    \oldsection[#1]{#2}%
}

% Redefine o comando de subseção para rastrear títulos
\renewcommand{\subsection}[2][]{%
    \stepcounter{subsectioncount}%
    \setcounter{subsubsectioncount}{0}%
    \setcounter{subsectionangle}{\autodocmap@subsectionbaseangle}%
    \multiply\value{subsectionangle} by \value{subsectioncount}%
    % Armazena os dados da subseção para uso posterior no mapa
    \protected@edef\autodocmap@subsectionlist{%
        \autodocmap@subsectionlist
        \noexpand\docmaplevelthree{subsec\arabic{sectioncount}_\arabic{subsectioncount}}{sec\arabic{sectioncount}}{\thesubsectionangle}{#2}%
    }%
    \oldsubsection[#1]{#2}%
}

% Redefine o comando de subsubseção para rastrear títulos
\renewcommand{\subsubsection}[2][]{%
    \stepcounter{subsubsectioncount}%
    \setcounter{subsubsectionangle}{\autodocmap@subsubsectionbaseangle}%
    \multiply\value{subsubsectionangle} by \value{subsubsectioncount}%
    % Armazena os dados da subsubseção para uso posterior no mapa
    \protected@edef\autodocmap@subsubsectionlist{%
        \autodocmap@subsubsectionlist
        \noexpand\docmaplevelfour{subsubsec\arabic{sectioncount}_\arabic{subsectioncount}_\arabic{subsubsectioncount}}{subsec\arabic{sectioncount}_\arabic{subsectioncount}}{\thesubsubsectionangle}{#2}%
    }%
    \oldsubsubsection[#1]{#2}%
}

% Comando para gerar automaticamente o mapa mental com base nos dados coletados
\newcommand{\generateautodocmap}{%
    \section*{Mapa Automático da Estrutura do Documento}
    
    \begin{docstructmap}
        % Nó raiz - Documento principal
        \docmaproot{Projeto Homebrew\\LaTeX Modular}
        
        % Elementos principais (nível 1)
        \docmaplevelone{pretextual}{30}{Elementos\\Pré-textuais}
        \docmaplevelone{textual}{150}{Elementos\\Textuais}
        \docmaplevelone{postextual}{270}{Elementos\\Pós-textuais}
        
        % Nós pré-definidos para elementos comuns (nível 2)
        \docmapleveltwo{capa}{pretextual}{0}{Capa e Título}
        \docmapleveltwo{toc}{pretextual}{90}{Sumário}
        
        \docmapleveltwo{ref}{postextual}{240}{Referências\\Bibliográficas}
        \docmapleveltwo{gloss}{postextual}{300}{Glossário}
        \docmapleveltwo{ind}{postextual}{340}{Índice Remissivo}
        
        % Gerar nós para cada seção registrada (nível 2)
        \autodocmap@sectionlist
        
        % Gerar nós para cada subseção registrada (nível 3)
        \autodocmap@subsectionlist
        
        % Gerar nós para cada subsubseção registrada (nível 4)
        \autodocmap@subsubsectionlist
        
    \end{docstructmap}
    
    \docmaplegend{
        Este mapa mental foi gerado automaticamente com base na estrutura real do documento.
        Ele representa a hierarquia de capítulos, seções e subseções no formato de um mapa mental.
        As cores indicam diferentes níveis na hierarquia do documento.
    }
}

% Comando para gerar automaticamente um mapa mental focado na estrutura atual
\newcommand{\generatesectiondocmap}[1][Capítulo Atual]{%
    \section*{Mapa da Estrutura do Capítulo}
    
    \begin{docstructmap}
        % Nó raiz - Capítulo atual
        \docmaproot{#1}
        
        % Gerar nós para cada seção registrada (nível 1)
        % Esta é uma versão simplificada que mostra apenas o capítulo atual
        % Em uma implementação mais completa, seria filtrado pelo capítulo atual
        \autodocmap@sectionlist
        
        % Gerar nós para cada subseção registrada (nível 2)
        \autodocmap@subsectionlist
        
    \end{docstructmap}
    
    \docmaplegend{
        Este mapa mental mostra a estrutura específica deste capítulo, 
        destacando suas seções e subseções principais.
    }
}

% Comando para gerar um mapa mental complexo predefinido
\newcommand{\generatecomplexdocmap}{%
    \section*{Mapa da Estrutura do Documento}
    
    \begin{docstructmap}
        % Nó raiz - Documento principal
        \docmaproot{Projeto Homebrew\\LaTeX Modular}
        
        % Elementos principais (nível 1)
        \docmaplevelone{pretextual}{30}{Elementos\\Pré-textuais}
        \docmaplevelone{textual}{150}{Elementos\\Textuais}
        \docmaplevelone{postextual}{270}{Elementos\\Pós-textuais}
        
        % Elementos pré-textuais (nível 2)
        \docmapleveltwo{capa}{pretextual}{0}{Capa e Título}
        \docmapleveltwo{toc}{pretextual}{60}{Sumário}
        \docmapleveltwo{mapament}{pretextual}{120}{Mapa Mental}
        
        % Elementos textuais - Capítulos (nível 2)
        \docmapleveltwo{cap1}{textual}{120}{Capítulo 1\\Introdução}
        \docmapleveltwo{cap2}{textual}{180}{Capítulo 2\\Bibliografia}
        \docmapleveltwo{cap3}{textual}{240}{Capítulo 3\\Glossário e Índice}
        
        % Elementos pós-textuais (nível 2)
        \docmapleveltwo{ref}{postextual}{240}{Referências\\Bibliográficas}
        \docmapleveltwo{gloss}{postextual}{300}{Glossário}
        \docmapleveltwo{ind}{postextual}{340}{Índice Remissivo}
        
        % Conteúdo do Capítulo 1 (nível 3)
        \docmaplevelthree{cap1_1}{cap1}{90}{Ambientes RPG}
        \docmaplevelthree{cap1_2}{cap1}{150}{Comandos\\Personalizados}
        \docmaplevelthree{cap1_3}{cap1}{210}{Mapas Mentais}
        
        % Conteúdo do Capítulo 2 (nível 3)
        \docmaplevelthree{cap2_1}{cap2}{150}{Sistema de Citações}
        \docmaplevelthree{cap2_2}{cap2}{210}{Referências Cruzadas}
        
        % Conteúdo do Capítulo 3 (nível 3)
        \docmaplevelthree{cap3_1}{cap3}{210}{Glossário}
        \docmaplevelthree{cap3_2}{cap3}{270}{Índice Remissivo}
        
        % Exemplos de nível 4 (detalhes mais específicos)
        \docmaplevelfour{amb1}{cap1_1}{60}{Spell}
        \docmaplevelfour{amb2}{cap1_1}{120}{Character}
        \docmaplevelfour{amb3}{cap1_1}{180}{Rule}
        
        \docmaplevelfour{cmd1}{cap1_2}{120}{rpgnote}
        \docmaplevelfour{cmd2}{cap1_2}{180}{rpgsection}
        \docmaplevelfour{cmd3}{cap1_2}{240}{rpgitem}
        
        % Conteúdo do mapa mental (nível 4)
        \docmaplevelfour{map1}{cap1_3}{120}{Estrutura\\Automática}
        \docmaplevelfour{map2}{cap1_3}{180}{Mapas\\Personalizados}
        \docmaplevelfour{map3}{cap1_3}{240}{Visualização\\Hierárquica}
    
    \end{docstructmap}
    
    \docmaplegend{
        Este mapa mental apresenta a estrutura completa do documento, incluindo capítulos, 
        seções e elementos específicos. Os níveis mais externos representam as divisões 
        principais do documento, enquanto os níveis internos mostram os detalhes específicos
        de cada seção.
    }
}

% Comando para simplificar a geração rápida de um mapa mental personalizado
\newcommand{\quickdocmap}[1]{%
    \section*{Mapa da Estrutura: #1}
    
    \begin{docstructmap}
        % Nó raiz personalizado
        \docmaproot{#1}
        
        % Elementos principais predefinidos (nível 1) - podem ser adaptados conforme necessário
        \docmaplevelone{introducao}{30}{Introdução}
        \docmaplevelone{desenvolvimento}{150}{Desenvolvimento}
        \docmaplevelone{conclusao}{270}{Conclusão}
    \end{docstructmap}
    
    \docmaplegend{
        Mapa mental simplificado para "#1". Este tipo de mapa pode ser usado
        para planejar novos documentos ou visualizar conceitos específicos.
    }
}


\begin{document}

\rpgtitle{Mapas Mentais para Documentos de RPG}

\rpgsection{Introdução}

Este documento demonstra como os mapas mentais podem ser usados para visualizar e organizar documentos relacionados a RPG, como manuais de regras, aventuras, descrições de mundos e outros materiais de jogo.

\begin{quotebox}
"Um bom mestre de RPG é tanto um narrador quanto um organizador. As ferramentas visuais que ajudam na organização do material também ajudam na criação de histórias mais ricas e coerentes."
\end{quotebox}

\rpgsection{Mapa de Aventura}

Abaixo está um mapa mental que estrutura uma aventura típica de RPG:

\begin{docstructmap}
    % Nó raiz
    \docmaproot{Aventura:\\A Cripta\\Esquecida}
    
    % Partes principais da aventura (nível 1)
    \docmaplevelone{introducao}{30}{Introdução}
    \docmaplevelone{ato1}{90}{Ato I:\\Descoberta}
    \docmaplevelone{ato2}{150}{Ato II:\\Exploração}
    \docmaplevelone{ato3}{210}{Ato III:\\Confronto}
    \docmaplevelone{apendices}{270}{Apêndices}
    \docmaplevelone{npcs}{330}{Personagens\\Não-Jogadores}
    
    % Elementos da Introdução (nível 2)
    \docmapleveltwo{gancho}{introducao}{0}{Ganchos de\\Aventura}
    \docmapleveltwo{resumo}{introducao}{60}{Resumo da\\Trama}
    
    % Elementos do Ato I (nível 2)
    \docmapleveltwo{cena1}{ato1}{60}{Cena 1:\\Vila Sombria}
    \docmapleveltwo{cena2}{ato1}{120}{Cena 2:\\Descoberta\\do Mapa}
    
    % Elementos do Ato II (nível 2)
    \docmapleveltwo{cena3}{ato2}{120}{Cena 3:\\Jornada Pela\\Floresta}
    \docmapleveltwo{cena4}{ato2}{180}{Cena 4:\\Entrada da\\Cripta}
    
    % Elementos do Ato III (nível 2)
    \docmapleveltwo{cena5}{ato3}{180}{Cena 5:\\Níveis\\Superiores}
    \docmapleveltwo{cena6}{ato3}{240}{Cena 6:\\Confronto\\Final}
    
    % Elementos dos Apêndices (nível 2)
    \docmapleveltwo{mapas}{apendices}{240}{Mapas e\\Diagramas}
    \docmapleveltwo{tesouros}{apendices}{300}{Tesouros e\\Recompensas}
    
    % Elementos de NPCs (nível 2)
    \docmapleveltwo{aliados}{npcs}{300}{Aliados}
    \docmapleveltwo{inimigos}{npcs}{360}{Inimigos}
    
    % Detalhes de uma cena específica (nível 3)
    \docmaplevelthree{enc1}{cena5}{150}{Encontro 1:\\Guardiões}
    \docmaplevelthree{enc2}{cena5}{210}{Encontro 2:\\Armadilhas}
    
    \docmaplevelthree{enc3}{cena6}{210}{Encontro 3:\\Ritual}
    \docmaplevelthree{enc4}{cena6}{270}{Encontro 4:\\Necromante}
\end{docstructmap}

\docmaplegend{
    Este mapa mental apresenta a estrutura de uma aventura de RPG, mostrando a organização 
    em atos e cenas, bem como os apêndices e personagens. Os nós de nível 3 mostram 
    encontros específicos dentro das cenas principais.
}

\rpgsection{Mapa de Sistema de Regras}

Os mapas mentais também são úteis para visualizar sistemas de regras:

\begin{docstructmap}
    % Nó raiz
    \docmaproot{Sistema de\\Regras RPG}
    
    % Seções principais do sistema (nível 1)
    \docmaplevelone{personagens}{30}{Criação de\\Personagens}
    \docmaplevelone{combate}{90}{Sistema de\\Combate}
    \docmaplevelone{magia}{150}{Magia e\\Habilidades}
    \docmaplevelone{equipamento}{210}{Equipamento\\e Itens}
    \docmaplevelone{mestre}{270}{Guia do\\Mestre}
    \docmaplevelone{aventuras}{330}{Criação de\\Aventuras}
    
    % Elementos de Criação de Personagens (nível 2)
    \docmapleveltwo{racas}{personagens}{0}{Raças}
    \docmapleveltwo{classes}{personagens}{30}{Classes}
    \docmapleveltwo{antecedentes}{personagens}{60}{Antecedentes}
    
    % Elementos de Combate (nível 2)
    \docmapleveltwo{iniciativa}{combate}{60}{Iniciativa}
    \docmapleveltwo{acoes}{combate}{90}{Ações}
    \docmapleveltwo{dano}{combate}{120}{Dano e\\Cura}
    
    % Elementos de Magia (nível 2)
    \docmapleveltwo{escolas}{magia}{120}{Escolas de\\Magia}
    \docmapleveltwo{magias}{magia}{150}{Lista de\\Magias}
    \docmapleveltwo{rituais}{magia}{180}{Rituais}
    
    % Detalhes específicos (nível 3)
    \docmaplevelthree{guerreiro}{classes}{15}{Guerreiro}
    \docmaplevelthree{mago}{classes}{45}{Mago}
    
    \docmaplevelthree{ataque}{acoes}{75}{Ação de\\Ataque}
    \docmaplevelthree{movimento}{acoes}{105}{Ação de\\Movimento}
\end{docstructmap}

\docmaplegend{
    Este mapa mental apresenta a estrutura de um sistema de regras de RPG, destacando
    as principais seções e seus componentes. É particularmente útil para visualizar como
    os diferentes elementos do sistema se relacionam entre si.
}

\rpgsection{Mapa de Construção de Mundo}

Os mapas mentais são excelentes ferramentas para desenvolver e organizar informações sobre mundos de RPG:

\begin{docstructmap}
    % Nó raiz
    \docmaproot{Reino de\\Valoria}
    
    % Elementos principais do mundo (nível 1)
    \docmaplevelone{geografia}{30}{Geografia}
    \docmaplevelone{historia}{90}{História}
    \docmaplevelone{politica}{150}{Política}
    \docmaplevelone{religiao}{210}{Religião}
    \docmaplevelone{culturas}{270}{Povos e\\Culturas}
    \docmaplevelone{magia}{330}{Magia e\\Tecnologia}
    
    % Elementos da Geografia (nível 2)
    \docmapleveltwo{regioes}{geografia}{0}{Regiões}
    \docmapleveltwo{cidades}{geografia}{60}{Cidades\\Principais}
    
    % Elementos da História (nível 2)
    \docmapleveltwo{fundacao}{historia}{60}{Era da\\Fundação}
    \docmapleveltwo{guerras}{historia}{120}{As Grandes\\Guerras}
    
    % Elementos de Política (nível 2)
    \docmapleveltwo{reinos}{politica}{120}{Reinos e\\Facções}
    \docmapleveltwo{comercio}{politica}{180}{Rotas\\Comerciais}
    
    % Elementos específicos de uma região (nível 3)
    \docmaplevelthree{montanhas}{regioes}{-15}{Montanhas\\do Norte}
    \docmaplevelthree{florestas}{regioes}{15}{Florestas\\do Leste}
    \docmaplevelthree{deserto}{regioes}{45}{Deserto\\do Sul}
    
    % Cidades específicas (nível 3)
    \docmaplevelthree{capital}{cidades}{45}{Alturan\\(Capital)}
    \docmaplevelthree{porto}{cidades}{75}{Porto\\Marítimo}
\end{docstructmap}

\docmaplegend{
    Este mapa mental apresenta a estrutura de um mundo de fantasia para RPG, organizando
    suas principais características geográficas, históricas, políticas e culturais. Este tipo
    de organização visual é especialmente útil para mestres criarem mundos coerentes.
}

\rpgsection{Mapa de Campanha}

Para campanhas de longo prazo, um mapa mental ajuda a manter a visão do arco narrativo:

\begin{docstructmap}
    % Nó raiz
    \docmaproot{Campanha:\\A Queda\\de Valoria}
    
    % Arcos principais da campanha (nível 1)
    \docmaplevelone{arco1}{30}{Arco 1:\\Sombras\\Emergentes}
    \docmaplevelone{arco2}{90}{Arco 2:\\Coalizão\\dos Reinos}
    \docmaplevelone{arco3}{150}{Arco 3:\\O Ritual\\Perdido}
    \docmaplevelone{arco4}{210}{Arco 4:\\Revelações\\Antigas}
    \docmaplevelone{arco5}{270}{Arco 5:\\A Batalha\\Final}
    \docmaplevelone{temas}{330}{Temas e\\Motivos}
    
    % Aventuras do Arco 1 (nível 2)
    \docmapleveltwo{av1_1}{arco1}{0}{Aventura 1.1:\\Presságios}
    \docmapleveltwo{av1_2}{arco1}{60}{Aventura 1.2:\\Investigação}
    
    % Aventuras do Arco 2 (nível 2)
    \docmapleveltwo{av2_1}{arco2}{60}{Aventura 2.1:\\Diplomacia}
    \docmapleveltwo{av2_2}{arco2}{120}{Aventura 2.2:\\Traição}
    
    % Elementos específicos de uma aventura (nível 3)
    \docmaplevelthree{local1_1}{av1_1}{-15}{Localização:\\Vila de Pedra}
    \docmaplevelthree{npcs1_1}{av1_1}{15}{NPCs:\\Oráculos}
    
    \docmaplevelthree{local1_2}{av1_2}{45}{Localização:\\Cavernas}
    \docmaplevelthree{item1_2}{av1_2}{75}{Item-Chave:\\Amuleto}
\end{docstructmap}

\docmaplegend{
    Este mapa mental apresenta a estrutura de uma campanha de RPG, organizando-a
    em arcos narrativos e aventuras individuais. Este tipo de visualização ajuda o mestre
    a manter a coerência narrativa ao longo de uma campanha extensa.
}

\rpgsection{Mapa de Classe de Personagem}

Os mapas mentais também são úteis para visualizar as opções de uma classe de personagem:

\begin{docstructmap}
    % Nó raiz
    \docmaproot{Mago}
    
    % Elementos principais da classe (nível 1)
    \docmaplevelone{escolas}{30}{Escolas de\\Especialização}
    \docmaplevelone{magias}{90}{Magias}
    \docmaplevelone{habilidades}{150}{Habilidades\\de Classe}
    \docmaplevelone{equipamento}{210}{Equipamento}
    \docmaplevelone{progressao}{270}{Progressão\\de Nível}
    \docmaplevelone{arquetipos}{330}{Arquetipos}
    
    % Escolas de Magia (nível 2)
    \docmapleveltwo{evocacao}{escolas}{0}{Evocação}
    \docmapleveltwo{abjuracao}{escolas}{30}{Abjuração}
    \docmapleveltwo{ilusao}{escolas}{60}{Ilusão}
    
    % Níveis de Magia (nível 2)
    \docmapleveltwo{nivel1}{magias}{60}{Nível 1}
    \docmapleveltwo{nivel2}{magias}{90}{Nível 2}
    \docmapleveltwo{nivel3}{magias}{120}{Nível 3}
    
    % Magias específicas (nível 3)
    \docmaplevelthree{misseis}{nivel1}{45}{Mísseis\\Mágicos}
    \docmaplevelthree{escudo}{nivel1}{75}{Escudo\\Arcano}
    
    \docmaplevelthree{invisibilidade}{nivel2}{75}{Invisibilidade}
    \docmaplevelthree{levitacao}{nivel2}{105}{Levitação}
\end{docstructmap}

\docmaplegend{
    Este mapa mental apresenta a estrutura de uma classe de personagem de RPG (Mago),
    mostrando suas escolas de especialização, progressão de magias, habilidades e
    equipamentos. É útil tanto para jogadores quanto para criadores de sistemas.
}

\rpgsection{Dicas para Uso Efetivo}

\begin{rule}
\textbf{Melhores Práticas para Mapas Mentais em RPG}

\begin{enumerate}
    \item \textbf{Priorize informações cruciais:} Mantenha o mapa focado nos elementos mais importantes
    \item \textbf{Use cores significativas:} As cores podem indicar diferentes tipos de elementos (combate, narrativa, NPCs)
    \item \textbf{Mantenha a consistência:} Use o mesmo estilo de organização em todos os mapas da sua obra
    \item \textbf{Inclua referências cruzadas:} Indique nos nós quando há conexões com outras partes do documento
    \item \textbf{Equilibre detalhes:} Nem muito complexo nem muito simples - encontre o equilíbrio ideal
\end{enumerate}
\end{rule}

\rpgnote{Use mapas mentais no início de cada sessão da sua mesa de RPG para orientar os jogadores sobre o progresso da campanha e as opções disponíveis.}

\begin{dmnote}
Como mestre, você pode criar mapas mentais progressivos que vão revelando mais detalhes à medida que os jogadores exploram o mundo, adicionando novos nós conforme a campanha avança.
\end{dmnote}

\rpgsection{Conclusão}

Os mapas mentais são ferramentas poderosas para mestres e criadores de conteúdo de RPG, ajudando a organizar informações complexas em formatos visualmente acessíveis. Seja para estruturar aventuras, sistemas de regras, mundos ou campanhas, esta técnica visual complementa perfeitamente o estilo visual do Projeto Homebrew.

\begin{highlight}
Experimente diferentes estilos e arranjos de mapas mentais para encontrar o formato que melhor funciona para o seu estilo de jogo e suas necessidades de documentação. A flexibilidade do sistema permite adaptações para qualquer tipo de conteúdo de RPG.
\end{highlight}

\end{document}