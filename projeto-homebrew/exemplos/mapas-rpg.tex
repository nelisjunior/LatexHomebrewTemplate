% Exemplo de uso de mapas mentais em contexto de RPG
% Este arquivo demonstra como usar a funcionalidade de mapas mentais para visualizar
% estruturas de documentos relacionados a RPG, como aventuras, descrição de mundos, etc.

\documentclass[12pt, a4paper]{book}

% Carrega as configurações e ambientes personalizados
% Pacotes básicos
\usepackage{geometry}
\usepackage{hyperref}
\usepackage{graphicx}
\usepackage{fancyhdr}
\usepackage{titlesec}
\usepackage{xcolor}
\usepackage{microtype}
\usepackage{lipsum}
\usepackage{tcolorbox}
\usepackage{enumitem}
\usepackage{booktabs}
\usepackage{array}
\usepackage{multicol}
\usepackage{afterpage}
\usepackage{tikz}
\usepackage{setspace}
\usepackage{bookmark}

% Suporte a língua portuguesa
\usepackage[brazilian]{babel}

% Bibliografia
\usepackage[backend=biber, style=alphabetic]{biblatex}
\addbibresource{referencias.bib}

% Configurações de fontes usando fontspec (para XeLaTeX)
\usepackage{fontspec}
\defaultfontfeatures{Ligatures=TeX}

% Tentativa de usar a fonte Dominican ou Luxurious Roman
\IfFileExists{Dominican.otf}{%
    \setmainfont{Dominican}
}{%
    \setmainfont{Luxurious Roman}[
        Path = /usr/share/fonts/truetype/luxurious-roman/,
        Extension = .ttf,
        UprightFont = *-Regular,
        BoldFont = *-Regular, % Se não houver variante negrito
        ItalicFont = *-Regular, % Se não houver variante itálica
        Renderer = Basic
    ]
}

% Definição de cores inspiradas em RPG
\definecolor{pergaminho}{RGB}{249, 240, 181}
\definecolor{capa}{RGB}{121, 26, 25}
\definecolor{titulo}{RGB}{72, 26, 19}
\definecolor{boxbg}{RGB}{253, 245, 196}
\definecolor{boxborder}{RGB}{190, 150, 86}
\definecolor{secaotitulo}{RGB}{140, 26, 20}

% Configuração de margens
\geometry{
    a4paper,
    top=2.5cm,
    bottom=2.5cm,
    left=3cm,
    right=3cm,
    headheight=15pt
}

% Configuração de espaçamento
\setlength{\parindent}{1.5cm}
\onehalfspacing

% Estilo de cabeçalho e rodapé
\pagestyle{fancy}
\fancyhf{}
\fancyhead[L]{\leftmark}
\fancyhead[R]{\thepage}
\fancyfoot[C]{\textit{Projeto Homebrew}}
\renewcommand{\headrulewidth}{0.4pt}
\renewcommand{\footrulewidth}{0.4pt}

% Personalização dos títulos de capítulos
\titleformat{\chapter}[display]
{\normalfont\huge\bfseries\color{secaotitulo}}
{\chaptertitlename\ \thechapter}{20pt}{\Huge}
\titlespacing*{\chapter}{0pt}{50pt}{40pt}

% Cor de fundo da página
\pagecolor{pergaminho}

% Ambientes personalizados com tcolorbox para estilo RPG

% Configurações comuns para caixas RPG
\tcbset{
    common/.style={
        enhanced,
        frame hidden,
        interior hidden,
        colback=boxbg,
        colframe=boxborder,
        fonttitle=\bfseries\Large,
        coltitle=white,
        colbacktitle=boxborder,
        attach boxed title to top left={yshift=-2mm, xshift=5mm},
        boxed title style={sharp corners, frame hidden},
        underlay={\begin{tcbclipinterior}
            \draw[boxborder, line width=2pt] 
            (frame.south west) rectangle (frame.north east);
            \end{tcbclipinterior}},
        breakable,
        drop shadow=boxborder,
    }
}

% Ambiente para descrição de itens mágicos
\newtcolorbox{magicitem}[1][]{
    common,
    title=Item Mágico,
    #1
}

% Ambiente para feitiços
\newtcolorbox{spell}[1][]{
    common,
    colbacktitle={rgb:blue,2;green,1;red,6},
    title=Feitiço,
    #1
}

% Ambiente para personagens
\newtcolorbox{character}[1][]{
    common,
    colbacktitle={rgb:red,2;green,2;blue,0},
    title=Personagem,
    #1
}

% Ambiente para notas do mestre
\newtcolorbox{dmnote}[1][]{
    common,
    colbacktitle={rgb:red,0;green,3;blue,5},
    title=Nota do Mestre,
    #1
}

% Ambiente para regras
\newtcolorbox{rule}[1][]{
    common,
    colbacktitle={rgb:red,4;green,0;blue,0},
    title=Regra,
    #1
}

% Ambiente para tabelas
\newtcolorbox{rpgtable}[1][]{
    common,
    colbacktitle={rgb:black,1;yellow,0.1},
    title=Tabela,
    #1
}

% Ambiente para citações
\newtcolorbox{quotebox}[1][]{
    common,
    fonttitle=\itshape\large,
    colbacktitle={rgb:gray,3;black,1},
    title=Citação,
    #1
}

% Ambiente para destaque
\newtcolorbox{highlight}[1][]{
    common,
    colbacktitle={rgb:orange,5;yellow,1},
    title=Destaque,
    #1
}

% Comandos para estatísticas de personagem
\newcommand{\statnumber}[1]{%
    \begingroup
    \setlength{\fboxsep}{2pt}%
    \colorbox{boxbg}{\textbf{#1}}%
    \endgroup
}

\newcommand{\stat}[2]{%
    \textbf{#1} \statnumber{#2}%
}

% Comando para criar barras de atributos
\newcommand{\attrbar}[2]{%
    \begingroup
    \setlength{\unitlength}{1mm}%
    \begin{picture}(30,5)%
    \put(0,0){\color{boxborder}\rule{30mm}{5mm}}%
    \put(0,0){\color{secaotitulo}\rule{#2mm}{5mm}}%
    \put(15,2.5){\makebox(0,0)[c]{\textcolor{white}{\textbf{#1}}}}%
    \end{picture}%
    \endgroup
}

% Comando para titulos estilizados
\newcommand{\rpgtitle}[1]{%
    \begin{center}
        \begingroup
        \setlength{\fboxsep}{5pt}%
        \colorbox{capa}{\textcolor{white}{\Large\bfseries #1}}%
        \endgroup
    \end{center}
}


\begin{document}

\rpgtitle{Mapas Mentais para Documentos de RPG}

\rpgsection{Introdução}

Este documento demonstra como os mapas mentais podem ser usados para visualizar e organizar documentos relacionados a RPG, como manuais de regras, aventuras, descrições de mundos e outros materiais de jogo.

\begin{quotebox}
"Um bom mestre de RPG é tanto um narrador quanto um organizador. As ferramentas visuais que ajudam na organização do material também ajudam na criação de histórias mais ricas e coerentes."
\end{quotebox}

\rpgsection{Mapa de Aventura}

Abaixo está um mapa mental que estrutura uma aventura típica de RPG:

\begin{docstructmap}
    % Nó raiz
    \docmaproot{Aventura:\\A Cripta\\Esquecida}
    
    % Partes principais da aventura (nível 1)
    \docmaplevelone{introducao}{30}{Introdução}
    \docmaplevelone{ato1}{90}{Ato I:\\Descoberta}
    \docmaplevelone{ato2}{150}{Ato II:\\Exploração}
    \docmaplevelone{ato3}{210}{Ato III:\\Confronto}
    \docmaplevelone{apendices}{270}{Apêndices}
    \docmaplevelone{npcs}{330}{Personagens\\Não-Jogadores}
    
    % Elementos da Introdução (nível 2)
    \docmapleveltwo{gancho}{introducao}{0}{Ganchos de\\Aventura}
    \docmapleveltwo{resumo}{introducao}{60}{Resumo da\\Trama}
    
    % Elementos do Ato I (nível 2)
    \docmapleveltwo{cena1}{ato1}{60}{Cena 1:\\Vila Sombria}
    \docmapleveltwo{cena2}{ato1}{120}{Cena 2:\\Descoberta\\do Mapa}
    
    % Elementos do Ato II (nível 2)
    \docmapleveltwo{cena3}{ato2}{120}{Cena 3:\\Jornada Pela\\Floresta}
    \docmapleveltwo{cena4}{ato2}{180}{Cena 4:\\Entrada da\\Cripta}
    
    % Elementos do Ato III (nível 2)
    \docmapleveltwo{cena5}{ato3}{180}{Cena 5:\\Níveis\\Superiores}
    \docmapleveltwo{cena6}{ato3}{240}{Cena 6:\\Confronto\\Final}
    
    % Elementos dos Apêndices (nível 2)
    \docmapleveltwo{mapas}{apendices}{240}{Mapas e\\Diagramas}
    \docmapleveltwo{tesouros}{apendices}{300}{Tesouros e\\Recompensas}
    
    % Elementos de NPCs (nível 2)
    \docmapleveltwo{aliados}{npcs}{300}{Aliados}
    \docmapleveltwo{inimigos}{npcs}{360}{Inimigos}
    
    % Detalhes de uma cena específica (nível 3)
    \docmaplevelthree{enc1}{cena5}{150}{Encontro 1:\\Guardiões}
    \docmaplevelthree{enc2}{cena5}{210}{Encontro 2:\\Armadilhas}
    
    \docmaplevelthree{enc3}{cena6}{210}{Encontro 3:\\Ritual}
    \docmaplevelthree{enc4}{cena6}{270}{Encontro 4:\\Necromante}
\end{docstructmap}

\docmaplegend{
    Este mapa mental apresenta a estrutura de uma aventura de RPG, mostrando a organização 
    em atos e cenas, bem como os apêndices e personagens. Os nós de nível 3 mostram 
    encontros específicos dentro das cenas principais.
}

\rpgsection{Mapa de Sistema de Regras}

Os mapas mentais também são úteis para visualizar sistemas de regras:

\begin{docstructmap}
    % Nó raiz
    \docmaproot{Sistema de\\Regras RPG}
    
    % Seções principais do sistema (nível 1)
    \docmaplevelone{personagens}{30}{Criação de\\Personagens}
    \docmaplevelone{combate}{90}{Sistema de\\Combate}
    \docmaplevelone{magia}{150}{Magia e\\Habilidades}
    \docmaplevelone{equipamento}{210}{Equipamento\\e Itens}
    \docmaplevelone{mestre}{270}{Guia do\\Mestre}
    \docmaplevelone{aventuras}{330}{Criação de\\Aventuras}
    
    % Elementos de Criação de Personagens (nível 2)
    \docmapleveltwo{racas}{personagens}{0}{Raças}
    \docmapleveltwo{classes}{personagens}{30}{Classes}
    \docmapleveltwo{antecedentes}{personagens}{60}{Antecedentes}
    
    % Elementos de Combate (nível 2)
    \docmapleveltwo{iniciativa}{combate}{60}{Iniciativa}
    \docmapleveltwo{acoes}{combate}{90}{Ações}
    \docmapleveltwo{dano}{combate}{120}{Dano e\\Cura}
    
    % Elementos de Magia (nível 2)
    \docmapleveltwo{escolas}{magia}{120}{Escolas de\\Magia}
    \docmapleveltwo{magias}{magia}{150}{Lista de\\Magias}
    \docmapleveltwo{rituais}{magia}{180}{Rituais}
    
    % Detalhes específicos (nível 3)
    \docmaplevelthree{guerreiro}{classes}{15}{Guerreiro}
    \docmaplevelthree{mago}{classes}{45}{Mago}
    
    \docmaplevelthree{ataque}{acoes}{75}{Ação de\\Ataque}
    \docmaplevelthree{movimento}{acoes}{105}{Ação de\\Movimento}
\end{docstructmap}

\docmaplegend{
    Este mapa mental apresenta a estrutura de um sistema de regras de RPG, destacando
    as principais seções e seus componentes. É particularmente útil para visualizar como
    os diferentes elementos do sistema se relacionam entre si.
}

\rpgsection{Mapa de Construção de Mundo}

Os mapas mentais são excelentes ferramentas para desenvolver e organizar informações sobre mundos de RPG:

\begin{docstructmap}
    % Nó raiz
    \docmaproot{Reino de\\Valoria}
    
    % Elementos principais do mundo (nível 1)
    \docmaplevelone{geografia}{30}{Geografia}
    \docmaplevelone{historia}{90}{História}
    \docmaplevelone{politica}{150}{Política}
    \docmaplevelone{religiao}{210}{Religião}
    \docmaplevelone{culturas}{270}{Povos e\\Culturas}
    \docmaplevelone{magia}{330}{Magia e\\Tecnologia}
    
    % Elementos da Geografia (nível 2)
    \docmapleveltwo{regioes}{geografia}{0}{Regiões}
    \docmapleveltwo{cidades}{geografia}{60}{Cidades\\Principais}
    
    % Elementos da História (nível 2)
    \docmapleveltwo{fundacao}{historia}{60}{Era da\\Fundação}
    \docmapleveltwo{guerras}{historia}{120}{As Grandes\\Guerras}
    
    % Elementos de Política (nível 2)
    \docmapleveltwo{reinos}{politica}{120}{Reinos e\\Facções}
    \docmapleveltwo{comercio}{politica}{180}{Rotas\\Comerciais}
    
    % Elementos específicos de uma região (nível 3)
    \docmaplevelthree{montanhas}{regioes}{-15}{Montanhas\\do Norte}
    \docmaplevelthree{florestas}{regioes}{15}{Florestas\\do Leste}
    \docmaplevelthree{deserto}{regioes}{45}{Deserto\\do Sul}
    
    % Cidades específicas (nível 3)
    \docmaplevelthree{capital}{cidades}{45}{Alturan\\(Capital)}
    \docmaplevelthree{porto}{cidades}{75}{Porto\\Marítimo}
\end{docstructmap}

\docmaplegend{
    Este mapa mental apresenta a estrutura de um mundo de fantasia para RPG, organizando
    suas principais características geográficas, históricas, políticas e culturais. Este tipo
    de organização visual é especialmente útil para mestres criarem mundos coerentes.
}

\rpgsection{Mapa de Campanha}

Para campanhas de longo prazo, um mapa mental ajuda a manter a visão do arco narrativo:

\begin{docstructmap}
    % Nó raiz
    \docmaproot{Campanha:\\A Queda\\de Valoria}
    
    % Arcos principais da campanha (nível 1)
    \docmaplevelone{arco1}{30}{Arco 1:\\Sombras\\Emergentes}
    \docmaplevelone{arco2}{90}{Arco 2:\\Coalizão\\dos Reinos}
    \docmaplevelone{arco3}{150}{Arco 3:\\O Ritual\\Perdido}
    \docmaplevelone{arco4}{210}{Arco 4:\\Revelações\\Antigas}
    \docmaplevelone{arco5}{270}{Arco 5:\\A Batalha\\Final}
    \docmaplevelone{temas}{330}{Temas e\\Motivos}
    
    % Aventuras do Arco 1 (nível 2)
    \docmapleveltwo{av1_1}{arco1}{0}{Aventura 1.1:\\Presságios}
    \docmapleveltwo{av1_2}{arco1}{60}{Aventura 1.2:\\Investigação}
    
    % Aventuras do Arco 2 (nível 2)
    \docmapleveltwo{av2_1}{arco2}{60}{Aventura 2.1:\\Diplomacia}
    \docmapleveltwo{av2_2}{arco2}{120}{Aventura 2.2:\\Traição}
    
    % Elementos específicos de uma aventura (nível 3)
    \docmaplevelthree{local1_1}{av1_1}{-15}{Localização:\\Vila de Pedra}
    \docmaplevelthree{npcs1_1}{av1_1}{15}{NPCs:\\Oráculos}
    
    \docmaplevelthree{local1_2}{av1_2}{45}{Localização:\\Cavernas}
    \docmaplevelthree{item1_2}{av1_2}{75}{Item-Chave:\\Amuleto}
\end{docstructmap}

\docmaplegend{
    Este mapa mental apresenta a estrutura de uma campanha de RPG, organizando-a
    em arcos narrativos e aventuras individuais. Este tipo de visualização ajuda o mestre
    a manter a coerência narrativa ao longo de uma campanha extensa.
}

\rpgsection{Mapa de Classe de Personagem}

Os mapas mentais também são úteis para visualizar as opções de uma classe de personagem:

\begin{docstructmap}
    % Nó raiz
    \docmaproot{Mago}
    
    % Elementos principais da classe (nível 1)
    \docmaplevelone{escolas}{30}{Escolas de\\Especialização}
    \docmaplevelone{magias}{90}{Magias}
    \docmaplevelone{habilidades}{150}{Habilidades\\de Classe}
    \docmaplevelone{equipamento}{210}{Equipamento}
    \docmaplevelone{progressao}{270}{Progressão\\de Nível}
    \docmaplevelone{arquetipos}{330}{Arquetipos}
    
    % Escolas de Magia (nível 2)
    \docmapleveltwo{evocacao}{escolas}{0}{Evocação}
    \docmapleveltwo{abjuracao}{escolas}{30}{Abjuração}
    \docmapleveltwo{ilusao}{escolas}{60}{Ilusão}
    
    % Níveis de Magia (nível 2)
    \docmapleveltwo{nivel1}{magias}{60}{Nível 1}
    \docmapleveltwo{nivel2}{magias}{90}{Nível 2}
    \docmapleveltwo{nivel3}{magias}{120}{Nível 3}
    
    % Magias específicas (nível 3)
    \docmaplevelthree{misseis}{nivel1}{45}{Mísseis\\Mágicos}
    \docmaplevelthree{escudo}{nivel1}{75}{Escudo\\Arcano}
    
    \docmaplevelthree{invisibilidade}{nivel2}{75}{Invisibilidade}
    \docmaplevelthree{levitacao}{nivel2}{105}{Levitação}
\end{docstructmap}

\docmaplegend{
    Este mapa mental apresenta a estrutura de uma classe de personagem de RPG (Mago),
    mostrando suas escolas de especialização, progressão de magias, habilidades e
    equipamentos. É útil tanto para jogadores quanto para criadores de sistemas.
}

\rpgsection{Dicas para Uso Efetivo}

\begin{rule}
\textbf{Melhores Práticas para Mapas Mentais em RPG}

\begin{enumerate}
    \item \textbf{Priorize informações cruciais:} Mantenha o mapa focado nos elementos mais importantes
    \item \textbf{Use cores significativas:} As cores podem indicar diferentes tipos de elementos (combate, narrativa, NPCs)
    \item \textbf{Mantenha a consistência:} Use o mesmo estilo de organização em todos os mapas da sua obra
    \item \textbf{Inclua referências cruzadas:} Indique nos nós quando há conexões com outras partes do documento
    \item \textbf{Equilibre detalhes:} Nem muito complexo nem muito simples - encontre o equilíbrio ideal
\end{enumerate}
\end{rule}

\rpgnote{Use mapas mentais no início de cada sessão da sua mesa de RPG para orientar os jogadores sobre o progresso da campanha e as opções disponíveis.}

\begin{dmnote}
Como mestre, você pode criar mapas mentais progressivos que vão revelando mais detalhes à medida que os jogadores exploram o mundo, adicionando novos nós conforme a campanha avança.
\end{dmnote}

\rpgsection{Conclusão}

Os mapas mentais são ferramentas poderosas para mestres e criadores de conteúdo de RPG, ajudando a organizar informações complexas em formatos visualmente acessíveis. Seja para estruturar aventuras, sistemas de regras, mundos ou campanhas, esta técnica visual complementa perfeitamente o estilo visual do Projeto Homebrew.

\begin{highlight}
Experimente diferentes estilos e arranjos de mapas mentais para encontrar o formato que melhor funciona para o seu estilo de jogo e suas necessidades de documentação. A flexibilidade do sistema permite adaptações para qualquer tipo de conteúdo de RPG.
\end{highlight}

\end{document}