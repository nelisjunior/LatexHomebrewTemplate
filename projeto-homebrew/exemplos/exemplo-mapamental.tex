% Exemplo de Uso dos Mapas Mentais no Projeto Homebrew
% Este arquivo demonstra as diferentes formas de usar os mapas mentais no projeto

\documentclass[12pt, a4paper]{book}

% Carrega as configurações e ambientes personalizados
% Pacotes básicos
\usepackage{geometry}
\usepackage{hyperref}
\usepackage{graphicx}
\usepackage{fancyhdr}
\usepackage{titlesec}
\usepackage{xcolor}
\usepackage{microtype}
\usepackage{lipsum}
\usepackage{tcolorbox}
\usepackage{enumitem}
\usepackage{booktabs}
\usepackage{array}
\usepackage{multicol}
\usepackage{afterpage}
\usepackage{tikz}
\usepackage{setspace}
\usepackage{bookmark}

% Suporte a língua portuguesa
\usepackage[brazilian]{babel}

% Bibliografia
\usepackage[backend=biber, style=alphabetic]{biblatex}
\addbibresource{referencias.bib}

% Configurações de fontes usando fontspec (para XeLaTeX)
\usepackage{fontspec}
\defaultfontfeatures{Ligatures=TeX}

% Tentativa de usar a fonte Dominican ou Luxurious Roman
\IfFileExists{Dominican.otf}{%
    \setmainfont{Dominican}
}{%
    \setmainfont{Luxurious Roman}[
        Path = /usr/share/fonts/truetype/luxurious-roman/,
        Extension = .ttf,
        UprightFont = *-Regular,
        BoldFont = *-Regular, % Se não houver variante negrito
        ItalicFont = *-Regular, % Se não houver variante itálica
        Renderer = Basic
    ]
}

% Definição de cores inspiradas em RPG
\definecolor{pergaminho}{RGB}{249, 240, 181}
\definecolor{capa}{RGB}{121, 26, 25}
\definecolor{titulo}{RGB}{72, 26, 19}
\definecolor{boxbg}{RGB}{253, 245, 196}
\definecolor{boxborder}{RGB}{190, 150, 86}
\definecolor{secaotitulo}{RGB}{140, 26, 20}

% Configuração de margens
\geometry{
    a4paper,
    top=2.5cm,
    bottom=2.5cm,
    left=3cm,
    right=3cm,
    headheight=15pt
}

% Configuração de espaçamento
\setlength{\parindent}{1.5cm}
\onehalfspacing

% Estilo de cabeçalho e rodapé
\pagestyle{fancy}
\fancyhf{}
\fancyhead[L]{\leftmark}
\fancyhead[R]{\thepage}
\fancyfoot[C]{\textit{Projeto Homebrew}}
\renewcommand{\headrulewidth}{0.4pt}
\renewcommand{\footrulewidth}{0.4pt}

% Personalização dos títulos de capítulos
\titleformat{\chapter}[display]
{\normalfont\huge\bfseries\color{secaotitulo}}
{\chaptertitlename\ \thechapter}{20pt}{\Huge}
\titlespacing*{\chapter}{0pt}{50pt}{40pt}

% Cor de fundo da página
\pagecolor{pergaminho}

% Ambientes personalizados com tcolorbox para estilo RPG

% Configurações comuns para caixas RPG
\tcbset{
    common/.style={
        enhanced,
        frame hidden,
        interior hidden,
        colback=boxbg,
        colframe=boxborder,
        fonttitle=\bfseries\Large,
        coltitle=white,
        colbacktitle=boxborder,
        attach boxed title to top left={yshift=-2mm, xshift=5mm},
        boxed title style={sharp corners, frame hidden},
        underlay={\begin{tcbclipinterior}
            \draw[boxborder, line width=2pt] 
            (frame.south west) rectangle (frame.north east);
            \end{tcbclipinterior}},
        breakable,
        drop shadow=boxborder,
    }
}

% Ambiente para descrição de itens mágicos
\newtcolorbox{magicitem}[1][]{
    common,
    title=Item Mágico,
    #1
}

% Ambiente para feitiços
\newtcolorbox{spell}[1][]{
    common,
    colbacktitle={rgb:blue,2;green,1;red,6},
    title=Feitiço,
    #1
}

% Ambiente para personagens
\newtcolorbox{character}[1][]{
    common,
    colbacktitle={rgb:red,2;green,2;blue,0},
    title=Personagem,
    #1
}

% Ambiente para notas do mestre
\newtcolorbox{dmnote}[1][]{
    common,
    colbacktitle={rgb:red,0;green,3;blue,5},
    title=Nota do Mestre,
    #1
}

% Ambiente para regras
\newtcolorbox{rule}[1][]{
    common,
    colbacktitle={rgb:red,4;green,0;blue,0},
    title=Regra,
    #1
}

% Ambiente para tabelas
\newtcolorbox{rpgtable}[1][]{
    common,
    colbacktitle={rgb:black,1;yellow,0.1},
    title=Tabela,
    #1
}

% Ambiente para citações
\newtcolorbox{quotebox}[1][]{
    common,
    fonttitle=\itshape\large,
    colbacktitle={rgb:gray,3;black,1},
    title=Citação,
    #1
}

% Ambiente para destaque
\newtcolorbox{highlight}[1][]{
    common,
    colbacktitle={rgb:orange,5;yellow,1},
    title=Destaque,
    #1
}

% Comandos para estatísticas de personagem
\newcommand{\statnumber}[1]{%
    \begingroup
    \setlength{\fboxsep}{2pt}%
    \colorbox{boxbg}{\textbf{#1}}%
    \endgroup
}

\newcommand{\stat}[2]{%
    \textbf{#1} \statnumber{#2}%
}

% Comando para criar barras de atributos
\newcommand{\attrbar}[2]{%
    \begingroup
    \setlength{\unitlength}{1mm}%
    \begin{picture}(30,5)%
    \put(0,0){\color{boxborder}\rule{30mm}{5mm}}%
    \put(0,0){\color{secaotitulo}\rule{#2mm}{5mm}}%
    \put(15,2.5){\makebox(0,0)[c]{\textcolor{white}{\textbf{#1}}}}%
    \end{picture}%
    \endgroup
}

% Comando para titulos estilizados
\newcommand{\rpgtitle}[1]{%
    \begin{center}
        \begingroup
        \setlength{\fboxsep}{5pt}%
        \colorbox{capa}{\textcolor{white}{\Large\bfseries #1}}%
        \endgroup
    \end{center}
}


\begin{document}

\rpgtitle{Exemplo de Mapas Mentais}

\rpgsection{Introdução}

Este documento demonstra as diversas formas de criar e usar mapas mentais para visualizar a estrutura do documento no Projeto Homebrew. Os mapas mentais ajudam leitores e autores a compreender a organização do conteúdo de forma visual e intuitiva.

\rpgsection{Mapa Mental Pré-definido}

Abaixo está o mapa mental pré-definido que apresenta uma estrutura típica de documento acadêmico:

% Mapa da estrutura do documento
\section*{Mapa do Documento}

\begin{center}
\begin{tikzpicture}[
    mindmap,
    level 1 concept/.append style={font=\large\bfseries, sibling angle=60, level distance=5cm},
    level 2 concept/.append style={font=\normalsize\bfseries, sibling angle=45, level distance=3.5cm},
    level 3 concept/.append style={font=\small, sibling angle=40, level distance=2.5cm},
    concept/.append style={
        text width=4cm, 
        font=\bfseries, 
        minimum size=2cm, 
        fill=boxbg, 
        text=secaotitulo, 
        line width=1pt, 
        draw=boxborder
    },
    concept connection/.append style={line width=1pt, draw=boxborder}
]

% Nó central/raiz - Título do documento
\node[concept, font=\Large\bfseries, minimum size=3cm, fill=capa!40!boxbg, text=white] (doc) {Projeto Homebrew\\LaTeX Modular};

% Elementos pré-textuais - Nível 1
\node[concept, fill=spell!10!boxbg] (pretextual) [grow=30] at (doc.30) {Elementos Pré-textuais};

% Elementos textuais - Nível 1
\node[concept, fill=magicitem!10!boxbg] (textual) [grow=150] at (doc.150) {Elementos Textuais};

% Elementos pós-textuais - Nível 1
\node[concept, fill=rule!10!boxbg] (postextual) [grow=270] at (doc.270) {Elementos Pós-textuais};

% Conexões do nó central aos nós de nível 1
\path (doc) to[circle connection bar] (pretextual);
\path (doc) to[circle connection bar] (textual);
\path (doc) to[circle connection bar] (postextual);

\end{tikzpicture}
\end{center}

\rpgsection{Mapa Mental Usando o Ambiente docstructmap}

\begin{spell}
Os comandos \texttt{\\docmaproot}, \texttt{\\docmaplevelone}, etc., facilitam a criação de mapas mentais sem precisar lidar diretamente com o código TikZ.
\end{spell}

Abaixo está um exemplo de mapa mental criado usando o ambiente \texttt{docstructmap} e os comandos relacionados:

\begin{docstructmap}
    % Nó raiz
    \docmaproot{Projeto\\Dissertação}
    
    % Nós de nível 1
    \docmaplevelone{metodologia}{30}{Metodologia}
    \docmaplevelone{resultados}{150}{Resultados}
    \docmaplevelone{conclusao}{270}{Conclusão}
    
    % Nós de nível 2
    \docmapleveltwo{coleta}{metodologia}{0}{Coleta de Dados}
    \docmapleveltwo{analise}{metodologia}{60}{Análise Estatística}
    
    \docmapleveltwo{primarios}{resultados}{120}{Resultados Primários}
    \docmapleveltwo{secundarios}{resultados}{180}{Resultados Secundários}
    
    \docmapleveltwo{achados}{conclusao}{240}{Principais Achados}
    \docmapleveltwo{futuro}{conclusao}{300}{Trabalhos Futuros}
    
    % Nós de nível 3
    \docmaplevelthree{qualitativa}{coleta}{-30}{Análise Qualitativa}
    \docmaplevelthree{quantitativa}{coleta}{30}{Análise Quantitativa}
\end{docstructmap}

\docmaplegend{
    Este mapa mental mostra a estrutura típica de uma dissertação acadêmica,
    destacando os principais componentes metodológicos, resultados e conclusões.
}

\rpgsection{Mapa Mental Complexo Gerado Automaticamente}

O comando \texttt{\\generatecomplexdocmap} gera um mapa mental detalhado com vários níveis hierárquicos. Este mapa é definido no arquivo \texttt{ambientes/auto-docmap.tex} e pode ser personalizado conforme necessário.

\generatecomplexdocmap

\rpgsection{Mapa Mental Auto-gerado com Base na Estrutura Real}

O comando \texttt{\\generateautodocmap} analisa a estrutura real do documento (seções, subseções, etc.) e gera um mapa mental dinamicamente. Este recurso é particularmente útil para documentos longos ou complexos.

\generateautodocmap

\rpgsection{Dicas de Uso}

\begin{itemize}
\rpgitem{Use mapas mentais no início de documentos longos para orientar o leitor}
\rpgitem{Inclua mapas mentais específicos para cada capítulo principal, mostrando sua estrutura interna}
\rpgitem{Combine mapas mentais estáticos e dinâmicos para diferentes propósitos}
\rpgitem{Personalize as cores, formas e estilos para adequar-se ao tema do seu documento}
\end{itemize}

\begin{dmnote}
Para documentos muito complexos, considere criar mapas mentais hierárquicos, com um mapa geral do documento e mapas específicos para cada seção principal.
\end{dmnote}

\rpgsection{Personalização Avançada}

\begin{quotebox}
"A visualização da estrutura de um documento é tão importante quanto o próprio conteúdo. Um bom mapa mental ajuda os leitores a navegar pelo documento e a compreender a relação entre as diferentes partes."
\end{quotebox}

Para personalização avançada dos mapas mentais, você pode modificar os arquivos:

\begin{itemize}
\rpgitem{\texttt{ambientes/docmap.tex} - Contém os comandos básicos e estilos para os mapas mentais}
\rpgitem{\texttt{ambientes/auto-docmap.tex} - Contém a implementação da geração automática de mapas}
\end{itemize}

\begin{magicitem}
\textbf{Dica Avançada}

Você pode personalizar os estilos dos nós em cada nível alterando as definições de cores e estilos no arquivo \texttt{ambientes/docmap.tex}. Por exemplo:

\begin{verbatim}
\definecolor{level0color}{RGB}{121, 26, 25}  % Cor da raiz
\definecolor{level1color}{RGB}{70, 26, 100}  % Cor do nível 1
\end{verbatim}
\end{magicitem}

\end{document}
