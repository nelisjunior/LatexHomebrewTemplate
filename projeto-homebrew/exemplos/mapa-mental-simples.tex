% Exemplo simplificado de mapa mental para teste no Overleaf
% Este arquivo é independente e pode ser compilado sozinho

\documentclass[12pt, a4paper]{article}

% Pacotes necessários
\usepackage[utf8]{inputenc}
\usepackage[T1]{fontenc}
\usepackage[brazilian]{babel}
\usepackage{tikz}
\usetikzlibrary{mindmap}
\usepackage{tcolorbox}

% Cores para o mapa mental
\definecolor{boxbg}{RGB}{253, 245, 196}
\definecolor{boxborder}{RGB}{190, 150, 86}
\definecolor{secaotitulo}{RGB}{140, 26, 20}
\definecolor{level0color}{RGB}{121, 26, 25}
\definecolor{level1color}{RGB}{70, 26, 100}
\definecolor{level2color}{RGB}{26, 70, 100}

% Configuração básica do ambiente para mapas mentais
\newenvironment{docstructmap}{%
    \begin{center}
    \begin{tikzpicture}[
        mindmap,
        level 1/.style={sibling angle=60, level distance=5cm},
        level 2/.style={sibling angle=45, level distance=3.5cm},
        level 3/.style={sibling angle=40, level distance=2.5cm},
        concept/.style={
            circle, draw=boxborder, thick, fill=boxbg,
            text width=4cm, text centered, font=\bfseries
        }
    ]
}{%
    \end{tikzpicture}
    \end{center}
}

% Comando para criar o nó raiz
\newcommand{\docmaproot}[1]{%
    \node[concept, fill=level0color!40, text=white, font=\large] (root) {#1};
}

% Comando para criar nós de nível 1
\newcommand{\docmaplevelone}[3]{%
    \node[concept, fill=level1color!20] (#1) at (#3:5cm) {#2};
    \draw[thick] (root) -- (#1);
}

% Comando para criar nós de nível 2
\newcommand{\docmapleveltwo}[4]{%
    \node[concept, fill=level2color!20, scale=0.8] (#1) at (#2) ++(#4:3.5cm) {#3};
    \draw[thick] (#2) -- (#1);
}

% Comando para legenda
\newcommand{\docmaplegend}[1]{%
    \begin{center}
    \begin{tcolorbox}[
        width=0.8\textwidth,
        colback=boxbg,
        colframe=boxborder,
        title=Sobre este Mapa Mental
    ]
    #1
    \end{tcolorbox}
    \end{center}
}

\begin{document}

\title{Exemplo Simples de Mapa Mental}
\author{Projeto Homebrew}
\date{\today}
\maketitle

\section{Introdução}

Este é um exemplo simplificado de mapa mental para teste no Overleaf. Este arquivo foi projetado para ser compilado independentemente, sem depender de outros arquivos do projeto.

\section{Exemplo de Mapa Mental Básico}

\begin{docstructmap}
    % Nó raiz
    \docmaproot{Documento\\Acadêmico}
    
    % Nós de nível 1
    \docmaplevelone{introducao}{Introdução}{0}
    \docmaplevelone{metodologia}{Metodologia}{60}
    \docmaplevelone{resultados}{Resultados}{120}
    \docmaplevelone{discussao}{Discussão}{180}
    \docmaplevelone{conclusao}{Conclusão}{240}
    \docmaplevelone{referencias}{Referências}{300}
    
    % Nós de nível 2
    \docmapleveltwo{problema}{introducao}{Problema de\\Pesquisa}{-15}
    \docmapleveltwo{objetivos}{introducao}{Objetivos}{15}
    
    \docmapleveltwo{desenho}{metodologia}{Desenho de\\Estudo}{45}
    \docmapleveltwo{analise}{metodologia}{Análise de\\Dados}{75}
    
    \docmapleveltwo{achados}{resultados}{Principais\\Achados}{105}
    \docmapleveltwo{limitacoes}{resultados}{Limitações}{135}
\end{docstructmap}

\docmaplegend{
    Este mapa mental apresenta a estrutura básica de um documento acadêmico,
    mostrando as principais seções e algumas subseções. Este é um exemplo
    simplificado para teste de compatibilidade com o Overleaf.
}

\section{Como Usar Este Exemplo}

Se este exemplo funcionar corretamente no Overleaf, você poderá adaptar o estilo e os comandos para seus próprios mapas mentais. Os elementos básicos são:

\begin{enumerate}
    \item O ambiente \texttt{docstructmap} que configura o mapa mental
    \item O comando \texttt{\textbackslash docmaproot\{texto\}} para o nó central
    \item O comando \texttt{\textbackslash docmaplevelone\{id\}\{texto\}\{ângulo\}} para nós de primeiro nível
    \item O comando \texttt{\textbackslash docmapleveltwo\{id\}\{id\_pai\}\{texto\}\{ângulo\}} para nós de segundo nível
\end{enumerate}

O ângulo é especificado em graus (0-360) e determina a posição do nó em relação ao seu pai.

\section{Explicação Técnica}

Este exemplo utiliza o pacote TikZ com a biblioteca mindmap para criar o mapa mental. A implementação é simplificada para maior compatibilidade com o Overleaf. Os principais elementos são:

\begin{itemize}
    \item Nós circulares para representar conceitos
    \item Conexões simples entre os nós
    \item Cores diferenciadas para cada nível hierárquico
    \item Posicionamento radial dos nós a partir do centro
\end{itemize}

\section{Próximos Passos}

Se este exemplo compilar corretamente, você pode:

\begin{enumerate}
    \item Copiar estes comandos para seu próprio projeto
    \item Personalizar as cores e estilos conforme necessário
    \item Expandir a estrutura para incluir mais níveis hierárquicos
    \item Integrar com o sistema completo de mapas mentais do Projeto Homebrew
\end{enumerate}

\end{document}