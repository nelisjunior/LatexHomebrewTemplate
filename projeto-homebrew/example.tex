\documentclass{article}

\usepackage{tcolorbox}
\usepackage{lipsum}
\usepackage{xcolor}

\begin{document}

\title{Exemplo de Uso dos Ambientes RPG}
\author{Projeto Homebrew}
\date{\today}

\maketitle

\section{Introdução}

Este é um exemplo simples para demonstrar os ambientes personalizados disponíveis no Projeto Homebrew. Em um documento completo, você teria acesso a todos os comandos e ambientes definidos nos arquivos configuracoes.tex e ambientes.tex.

\section{Exemplo de Ambientes}

Abaixo, alguns exemplos simples do que seria possível fazer com o projeto completo:

\begin{tcolorbox}[title=Feitiço]
\textbf{Bola de Fogo}\\
\textit{Evocação de 3º nível}\\
\textbf{Tempo de Conjuração:} 1 ação\\
\textbf{Alcance:} 45 metros\\
\textbf{Componentes:} V, S, M (uma pequena bola de guano de morcego e enxofre)\\
\textbf{Duração:} Instantânea

Uma luz brilhante surge de seu dedo indicador e se condensa em uma bola de fogo que voa até um ponto escolhido dentro do alcance e então explode. Cada criatura em uma esfera de 6 metros de raio centrada no ponto deve fazer um teste de resistência de Destreza. Um alvo sofre 8d6 de dano de fogo em uma falha, ou metade desse dano em um sucesso.
\end{tcolorbox}

\vspace{1cm}

\begin{tcolorbox}[title=Personagem]
\textbf{Gandalf, o Cinzento}\\
\textit{Humano Mago 12}\\
\textbf{Força:} 10 | \textbf{Destreza:} 12 | \textbf{Constituição:} 14\\
\textbf{Inteligência:} 18 | \textbf{Sabedoria:} 16 | \textbf{Carisma:} 15

Gandalf é um mago viajante, conhecido por seus fogos de artifício e sabedoria. Ele carrega a lendária espada Glamdring e seu cajado, canalizando poderosa magia. Como um dos Istari, ele foi enviado aos reinos mortais para combater a ameaça crescente de Sauron.
\end{tcolorbox}

\vspace{1cm}

\begin{tcolorbox}[title=Regra]
\textbf{Teste de Resistência}\\
Quando você é alvo de um efeito perigoso, como uma armadilha ou um feitiço, ou quando está em uma situação perigosa, como tentar atravessar uma sala cheia de armadilhas, você é chamado a fazer um teste de resistência para mitigar o efeito ou evitá-lo completamente.

A classe de dificuldade (CD) de um teste de resistência é determinada pelo efeito que o causa. Por exemplo, a CD para um teste de resistência contra um feitiço é determinada pela característica de conjuração do conjurador e seu bônus de proficiência.
\end{tcolorbox}

\vspace{1cm}

\begin{tcolorbox}[title=Item Mágico]
\textbf{Anel de Invisibilidade}\\
\textit{Anel, lendário (requer sintonização)}

Enquanto você estiver usando este anel, você pode usar uma ação para ficar invisível. Tudo que você estiver vestindo ou carregando fica invisível com você. Você permanece invisível até o anel ser removido, até você atacar ou conjurar uma magia, ou até você usar uma ação bônus para ficar visível novamente.
\end{tcolorbox}

\section{Considerações Finais}

No projeto completo, seria possível definir cores personalizadas, usar o estilo visual de pergaminho com margens maiores, e criar notas de margem para anotações.

Os ambientes podem ser personalizados conforme suas necessidades, tanto na aparência quanto no conteúdo.

\end{document}