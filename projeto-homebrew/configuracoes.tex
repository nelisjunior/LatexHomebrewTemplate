% Pacotes básicos
\usepackage{geometry}
\usepackage{hyperref}
\usepackage{graphicx}
\usepackage{fancyhdr}
\usepackage{titlesec}
\usepackage[usenames,dvipsnames]{xcolor}
\usepackage{microtype}
\usepackage{lipsum}
\usepackage{tcolorbox}
\usepackage{enumitem}
\usepackage{booktabs}
\usepackage{array}
\usepackage{multicol}
\usepackage{afterpage}
\usepackage{tikz}
\usepackage{setspace}
\usepackage{bookmark}
\usepackage{float}
\usepackage{lastpage}
\usepackage[framemethod=TikZ]{mdframed}
\usepackage{subfig}
\usepackage{csquotes}
\usepackage{url}
\usepackage{tabularx}
\usepackage{verbatim}
\usepackage{tocbibind}
\usepackage{newfloat}
\usepackage[useregional]{datetime2}
\usepackage{ragged2e}
\usepackage{marginnote}
\usepackage{pifont}
\usepackage{etoolbox}

% Pacotes para mapa mental/estrutura do documento
\usetikzlibrary{mindmap,trees,shadows,arrows,positioning}
\usetikzlibrary{decorations.pathmorphing}
\usetikzlibrary{decorations.markings}
\usetikzlibrary{shapes.geometric}

% Suporte a língua portuguesa
\usepackage[brazilian]{babel}

% Bibliografia
\usepackage[backend=biber, style=alphabetic]{biblatex}
\addbibresource{referencias.bib}

% Configurações de hyperlinks
\hypersetup{
    colorlinks=true,
    linkcolor=secaotitulo,
    citecolor=blue,
    urlcolor=blue
}

% Configurações de fontes usando fontspec (para XeLaTeX)
\usepackage{fontspec}
\defaultfontfeatures{Ligatures=TeX}

% Configuração de fontes para documentos RPG
% Usando Latin Modern Roman que é mais compatível com Overleaf
\setmainfont{Latin Modern Roman}

% Comentário: Se Latin Modern Roman não estiver disponível no seu sistema,
% você pode descomentar a linha abaixo e usar Times New Roman
% \setmainfont{Times New Roman}

% Definição de cores inspiradas em RPG
\definecolor{pergaminho}{RGB}{249, 240, 181}
\definecolor{capa}{RGB}{121, 26, 25}
\definecolor{titulo}{RGB}{72, 26, 19}
\definecolor{boxbg}{RGB}{253, 245, 196}
\definecolor{boxborder}{RGB}{190, 150, 86}
\definecolor{secaotitulo}{RGB}{140, 26, 20}
\definecolor{spell}{RGB}{70, 26, 100}
\definecolor{magicitem}{RGB}{26, 70, 100}
\definecolor{rule}{RGB}{140, 20, 20}
\definecolor{note}{RGB}{20, 100, 120}

% Configuração de margens
\geometry{
    a4paper,
    top=2.5cm,
    bottom=2.5cm,
    left=3cm,
    right=3cm,
    headheight=15pt,
    marginparwidth=2.5cm,
    marginparsep=0.5cm
}

% Configuração de espaçamento
\setlength{\parindent}{1.5cm}
\onehalfspacing

% Estilo de cabeçalho e rodapé
\pagestyle{fancy}
\fancyhf{}
\fancyhead[L]{\leftmark}
\fancyhead[R]{\thepage}
\fancyfoot[C]{\textit{Projeto Homebrew}}
\renewcommand{\headrulewidth}{0.4pt}
\renewcommand{\footrulewidth}{0.4pt}

% Personalização dos títulos de seções (equivalente a capítulos em article)
\titleformat{\section}[display]
{\normalfont\huge\bfseries\color{secaotitulo}}
{\thesection}{20pt}{\Huge}
\titlespacing*{\section}{0pt}{50pt}{40pt}

% Cor de fundo da página
\pagecolor{pergaminho}
