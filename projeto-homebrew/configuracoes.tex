% Pacotes básicos
\usepackage{geometry}
\usepackage{hyperref}
\usepackage{graphicx}
\usepackage{fancyhdr}
\usepackage{titlesec}
\usepackage{xcolor}
\usepackage{microtype}
\usepackage{lipsum}
\usepackage{tcolorbox}
\usepackage{enumitem}
\usepackage{booktabs}
\usepackage{array}
\usepackage{multicol}
\usepackage{afterpage}
\usepackage{tikz}
\usepackage{setspace}
\usepackage{bookmark}
\usepackage{float}
\usepackage{lastpage}
\usepackage{mdframed}
\usepackage{subfig}
\usepackage[usenames]{xcolor}
\usepackage{csquotes}
\usepackage{url}
\usepackage{tabularx}
\usepackage{verbatim}
\usepackage{tocbibind}
\usepackage{newfloat}

% Suporte a língua portuguesa
\usepackage[brazilian]{babel}

% Bibliografia
\usepackage[backend=biber, style=alphabetic]{biblatex}
\addbibresource{referencias.bib}

% Configurações de hyperlinks
\hypersetup{
    colorlinks=true,
    linkcolor=secaotitulo,
    citecolor=blue,
    urlcolor=blue
}

% Configurações de fontes usando fontspec (para XeLaTeX)
\usepackage{fontspec}
\defaultfontfeatures{Ligatures=TeX}

% Tentativa de usar a fonte Dominican ou Luxurious Roman
\IfFileExists{Dominican.otf}{%
    \setmainfont{Dominican}
}{%
    \setmainfont{Luxurious Roman}[
        Path = /usr/share/fonts/truetype/luxurious-roman/,
        Extension = .ttf,
        UprightFont = *-Regular,
        BoldFont = *-Regular, % Se não houver variante negrito
        ItalicFont = *-Regular, % Se não houver variante itálica
        Renderer = Basic
    ]
}

% Definição de cores inspiradas em RPG
\definecolor{pergaminho}{RGB}{249, 240, 181}
\definecolor{capa}{RGB}{121, 26, 25}
\definecolor{titulo}{RGB}{72, 26, 19}
\definecolor{boxbg}{RGB}{253, 245, 196}
\definecolor{boxborder}{RGB}{190, 150, 86}
\definecolor{secaotitulo}{RGB}{140, 26, 20}

% Configuração de margens
\geometry{
    a4paper,
    top=2.5cm,
    bottom=2.5cm,
    left=3cm,
    right=3cm,
    headheight=15pt
}

% Configuração de espaçamento
\setlength{\parindent}{1.5cm}
\onehalfspacing

% Estilo de cabeçalho e rodapé
\pagestyle{fancy}
\fancyhf{}
\fancyhead[L]{\leftmark}
\fancyhead[R]{\thepage}
\fancyfoot[C]{\textit{Projeto Homebrew}}
\renewcommand{\headrulewidth}{0.4pt}
\renewcommand{\footrulewidth}{0.4pt}

% Personalização dos títulos de capítulos
\titleformat{\chapter}[display]
{\normalfont\huge\bfseries\color{secaotitulo}}
{\chaptertitlename\ \thechapter}{20pt}{\Huge}
\titlespacing*{\chapter}{0pt}{50pt}{40pt}

% Cor de fundo da página
\pagecolor{pergaminho}
