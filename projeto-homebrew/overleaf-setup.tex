% Configurações específicas para compilação no Overleaf
% Inclua este arquivo no seu main.tex para resolver problemas de compilação

% Verificar e carregar pacotes necessários para os mapas mentais
\usepackage{tikz}
\usetikzlibrary{mindmap,trees,shadows,arrows,positioning}
\usetikzlibrary{decorations.pathmorphing}
\usetikzlibrary{decorations.markings}
\usetikzlibrary{shapes.geometric}

% Pacotes necessários para manipulação de listas e strings
\usepackage{etoolbox}
\usepackage{xstring}

% Em caso de erro no comando \protected@edef, descomente esta linha:
% \makeatletter\let\protected@edef\edef\makeatother

% Desativação temporária de recursos avançados para verificação
% Se o mapa mental automático estiver causando problemas, descomente esta linha:
% \newcommand{\generateautodocmap}{\textbf{[Mapa mental automático desativado temporariamente]}}

% Se o mapa mental complexo estiver causando problemas, descomente esta linha:
% \newcommand{\generatecomplexdocmap}{\textbf{[Mapa mental complexo desativado temporariamente]}}

% Se o ambiente docstructmap estiver causando problemas, descomente este bloco:
%\renewenvironment{docstructmap}[1][]
%  {\begin{center}\textbf{[Visualização de mapa mental]}\\}
%  {\end{center}}
%\newcommand{\docmaproot}[2][]{#2}
%\newcommand{\docmaplevelone}[4][]{#4}
%\newcommand{\docmapleveltwo}[5][]{#5}
%\newcommand{\docmaplevelthree}[5][]{#5}
%\newcommand{\docmaplevelfour}[5][]{#5}
%\newcommand{\docmaplegend}[1]{#1}

% Nota: Ative as linhas acima conforme necessário para identificar a fonte dos problemas
% Uma vez identificado, você pode resolver os problemas específicos ao invés de
% desativar funcionalidades inteiras.