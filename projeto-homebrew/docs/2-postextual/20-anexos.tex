% Conteúdo dos elementos pós-textuais

\cleardoublepage
\chapter*{Conclusão}

Este documento demonstra as capacidades do modelo modular LaTeX com estética de RPG. Combine a organização de documentos acadêmicos com o estilo visual dos livros de RPG para criar documentos únicos e interessantes.

A estética inspirada nos livros de RPG, com seus ambientes customizados, formatação especial e uso de cores, permite criar documentos técnicos ou acadêmicos com uma apresentação visualmente atraente e diferenciada \cite{rpglayout}.

\cleardoublepage
\chapter*{Glossário}

\newglossaryentry{latex}{
    name={LaTeX},
    description={Sistema de tipografia que permite a criação de documentos com alta qualidade tipográfica}
}

\newglossaryentry{rpg}{
    name={RPG},
    description={Role-Playing Game, jogo onde os jogadores assumem os papéis de personagens em um cenário fictício}
}

\newglossaryentry{homebrewery}{
    name={Homebrewery},
    description={Estilo de formatação inspirado em livros de RPG, especialmente Dungeons \& Dragons}
}

\newglossaryentry{modular}{
    name={Modular},
    description={Estrutura organizada em componentes independentes que podem ser combinados}
}

\newglossaryentry{tcolorbox}{
    name={tcolorbox},
    description={Pacote LaTeX para criação de caixas coloridas e personalizadas \cite{tcolorbox}}
}

\printglossaries

\cleardoublepage
\chapter*{Apêndices}

\begin{rpgtable}
\begin{tabular}{lcc}
\toprule
\textbf{Item} & \textbf{Categoria} & \textbf{Valor} \\
\midrule
Espada Longa & Arma & 15 po \\
Poção de Cura & Consumível & 50 po \\
Grimório & Mágico & 100 po \\
Botas Élficas & Equipamento & 25 po \\
Manto da Invisibilidade & Mágico & 500 po \\
\bottomrule
\end{tabular}
\caption{Tabela de itens comuns em aventuras de RPG}
\label{tab:itens}
\end{rpgtable}

\cleardoublepage
\chapter*{Comandos de Referência}

\begin{rpgtable}
\begin{tabularx}{\textwidth}{llX}
\toprule
\textbf{Comando} & \textbf{Tipo} & \textbf{Descrição} \\
\midrule
\verb|\begin{magicitem}| & Ambiente & Cria uma caixa para descrição de itens mágicos \\
\verb|\begin{spell}| & Ambiente & Cria uma caixa para descrição de magias \\
\verb|\begin{character}| & Ambiente & Cria uma caixa para ficha de personagens \\
\verb|\begin{dmnote}| & Ambiente & Cria uma caixa para notas do mestre \\
\verb|\begin{rule}| & Ambiente & Cria uma caixa para regras especiais \\
\verb|\begin{rpgtable}| & Ambiente & Cria uma caixa para tabelas \\
\verb|\begin{quotebox}| & Ambiente & Cria uma caixa para citações \\
\verb|\begin{highlight}| & Ambiente & Cria uma caixa para destacar informações \\
\verb|\rpgtitle{Título}| & Comando & Cria um título estilizado \\
\verb|\rpgnote{Texto}| & Comando & Cria uma nota na margem \\
\verb|\rpgsection{Título}| & Comando & Cria um cabeçalho de seção estilizado \\
\verb|\stat{Atributo}{Valor}| & Comando & Exibe um atributo com valor destacado \\
\verb|\attrbar{Texto}{Valor}| & Comando & Cria uma barra de atributo \\
\bottomrule
\end{tabularx}
\caption{Comandos personalizados disponíveis neste modelo}
\label{tab:comandos}
\end{rpgtable}

\cleardoublepage
\printbibliography[title=Referências]
