% Ambientes personalizados com tcolorbox para estilo RPG

% Configurações comuns para caixas RPG
\tcbset{
    common/.style={
        enhanced,
        frame hidden,
        interior hidden,
        colback=boxbg,
        colframe=boxborder,
        fonttitle=\bfseries\Large,
        coltitle=white,
        colbacktitle=boxborder,
        attach boxed title to top left={yshift=-2mm, xshift=5mm},
        boxed title style={sharp corners, frame hidden},
        underlay={\begin{tcbclipinterior}
            \draw[boxborder, line width=2pt] 
            (frame.south west) rectangle (frame.north east);
            \end{tcbclipinterior}},
        breakable,
        drop shadow=boxborder,
        before skip=0.5cm,
        after skip=0.5cm,
    }
}

% Estilo alternativo usando mdframed
\mdfdefinestyle{rpgstyle}{%
    linecolor=boxborder,
    linewidth=2pt,
    backgroundcolor=boxbg,
    roundcorner=5pt,
    shadow=true,
    shadowcolor=black!30,
}

% Ambiente para descrição de itens mágicos
\newtcolorbox{magicitem}[1][]{
    common,
    colbacktitle=magicitem,
    title=Item Mágico,
    #1
}

% Ambiente para feitiços
\newtcolorbox{spell}[1][]{
    common,
    colbacktitle=spell,
    title=Feitiço,
    #1
}

% Ambiente para personagens
\newtcolorbox{character}[1][]{
    common,
    colbacktitle=Brown!80!black,
    title=Personagem,
    #1
}

% Ambiente para notas do mestre
\newtcolorbox{dmnote}[1][]{
    common,
    colbacktitle=note,
    title=Nota do Mestre,
    #1
}

% Ambiente para regras
\newtcolorbox{rule}[1][]{
    common,
    colbacktitle=rule,
    title=Regra,
    #1
}

% Ambiente para tabelas
\newtcolorbox{rpgtable}[1][]{
    common,
    colbacktitle=Mahogany!70!black,
    title=Tabela,
    #1
}

% Ambiente para citações
\newtcolorbox{quotebox}[1][]{
    common,
    fonttitle=\itshape\large,
    colbacktitle=Gray!70!black,
    title=Citação,
    #1
}

% Ambiente para destaque
\newtcolorbox{highlight}[1][]{
    common,
    colbacktitle=Orange!70!black,
    title=Destaque,
    #1
}

% Comandos para estatísticas de personagem
\newcommand{\statnumber}[1]{%
    \begingroup
    \setlength{\fboxsep}{2pt}%
    \colorbox{boxbg}{\textbf{#1}}%
    \endgroup
}

\newcommand{\stat}[2]{%
    \textbf{#1} \statnumber{#2}%
}

% Comando para criar barras de atributos
\newcommand{\attrbar}[2]{%
    \begingroup
    \setlength{\unitlength}{1mm}%
    \begin{picture}(30,5)%
    \put(0,0){\color{boxborder}\rule{30mm}{5mm}}%
    \put(0,0){\color{secaotitulo}\rule{#2mm}{5mm}}%
    \put(15,2.5){\makebox(0,0)[c]{\textcolor{white}{\textbf{#1}}}}%
    \end{picture}%
    \endgroup
}

% Comando para criar listas de itens estilizadas
\newcommand{\rpgitem}[1]{%
    \item[\textcolor{secaotitulo}{\small\ding{108}}] #1
}

% Comando para titulos estilizados
\newcommand{\rpgtitle}[1]{%
    \begin{center}
        \begingroup
        \setlength{\fboxsep}{5pt}%
        \colorbox{capa}{\textcolor{white}{\Large\bfseries #1}}%
        \endgroup
    \end{center}
}

% Comando para notas de margem (anotações)
\newcommand{\rpgnote}[1]{%
    \marginpar{%
        \begin{mdframed}[style=rpgstyle]
        {\small\itshape #1}
        \end{mdframed}%
    }%
}

% Comando para adição de glossário
\newcommand{\glossaryentry}[2]{%
    \newglossaryentry{#1}{%
        name=#1,%
        description={#2}%
    }%
}

% Comando para encabezamento de seção RPG
\newcommand{\rpgsection}[1]{%
    \vspace{0.5cm}
    \begin{center}
        \begin{tikzpicture}
            \node[draw=boxborder, fill=boxbg, line width=2pt, inner sep=8pt] 
                 {\large\bfseries\color{secaotitulo} #1};
        \end{tikzpicture}
    \end{center}
    \vspace{0.3cm}
}

% Importação dos ambientes para mapa mental de estrutura do documento
% Mapa da estrutura do documento
\section*{Mapa do Documento}

\begin{center}
\begin{tikzpicture}[
    mindmap,
    level 1 concept/.append style={font=\large\bfseries, sibling angle=60, level distance=5cm},
    level 2 concept/.append style={font=\normalsize\bfseries, sibling angle=45, level distance=3.5cm},
    level 3 concept/.append style={font=\small, sibling angle=40, level distance=2.5cm},
    concept/.append style={
        text width=4cm, 
        font=\bfseries, 
        minimum size=2cm, 
        fill=boxbg, 
        text=secaotitulo, 
        line width=1pt, 
        draw=boxborder
    },
    concept connection/.append style={line width=1pt, draw=boxborder}
]

% Nó central/raiz - Título do documento
\node[concept, font=\Large\bfseries, minimum size=3cm, fill=capa!40!boxbg, text=white] (doc) {Projeto Homebrew\\LaTeX Modular};

% Elementos pré-textuais - Nível 1
\node[concept, fill=spell!10!boxbg] (pretextual) [grow=30] at (doc.30) {Elementos Pré-textuais};

% Elementos textuais - Nível 1
\node[concept, fill=magicitem!10!boxbg] (textual) [grow=150] at (doc.150) {Elementos Textuais};

% Elementos pós-textuais - Nível 1
\node[concept, fill=rule!10!boxbg] (postextual) [grow=270] at (doc.270) {Elementos Pós-textuais};

% Conexões do nó central aos nós de nível 1
\path (doc) to[circle connection bar] (pretextual);
\path (doc) to[circle connection bar] (textual);
\path (doc) to[circle connection bar] (postextual);

% Elementos pré-textuais - Nível 2
\node[concept, fill=spell!5!boxbg] (title) [grow=0] at (pretextual.0) {Título e Capa};
\node[concept, fill=spell!5!boxbg] (summary) [grow=90] at (pretextual.90) {Sumário};

% Elementos textuais - Nível 2
\node[concept, fill=magicitem!5!boxbg] (chap1) [grow=120] at (textual.120) {Capítulo 1\\Introdução};
\node[concept, fill=magicitem!5!boxbg] (chap2) [grow=180] at (textual.180) {Capítulo 2\\Bibliografia};
\node[concept, fill=magicitem!5!boxbg] (chap3) [grow=240] at (textual.240) {Capítulo 3\\Glossário e Índice};

% Elementos pós-textuais - Nível 2
\node[concept, fill=rule!5!boxbg] (refs) [grow=270] at (postextual.270) {Referências};
\node[concept, fill=rule!5!boxbg] (glossary) [grow=320] at (postextual.320) {Glossário};
\node[concept, fill=rule!5!boxbg] (index) [grow=220] at (postextual.220) {Índice Remissivo};

% Conexões dos nós de nível 1 aos nós de nível 2
\path (pretextual) to[circle connection bar] (title);
\path (pretextual) to[circle connection bar] (summary);

\path (textual) to[circle connection bar] (chap1);
\path (textual) to[circle connection bar] (chap2);
\path (textual) to[circle connection bar] (chap3);

\path (postextual) to[circle connection bar] (refs);
\path (postextual) to[circle connection bar] (glossary);
\path (postextual) to[circle connection bar] (index);

% Elementos do capítulo 1 - Nível 3
\node[concept, scale=0.7, fill=magicitem!2!boxbg] (chap1_1) [grow=90] at (chap1.90) {Ambientes RPG};
\node[concept, scale=0.7, fill=magicitem!2!boxbg] (chap1_2) [grow=150] at (chap1.150) {Comandos Personalizados};

% Elementos do capítulo 2 - Nível 3
\node[concept, scale=0.7, fill=magicitem!2!boxbg] (chap2_1) [grow=150] at (chap2.150) {Sistema de Citações};
\node[concept, scale=0.7, fill=magicitem!2!boxbg] (chap2_2) [grow=210] at (chap2.210) {Referências Cruzadas};

% Elementos do capítulo 3 - Nível 3
\node[concept, scale=0.7, fill=magicitem!2!boxbg] (chap3_1) [grow=210] at (chap3.210) {Glossário};
\node[concept, scale=0.7, fill=magicitem!2!boxbg] (chap3_2) [grow=270] at (chap3.270) {Índice Remissivo};

% Conexões dos nós de nível 2 aos nós de nível 3
\path (chap1) to[circle connection bar] (chap1_1);
\path (chap1) to[circle connection bar] (chap1_2);

\path (chap2) to[circle connection bar] (chap2_1);
\path (chap2) to[circle connection bar] (chap2_2);

\path (chap3) to[circle connection bar] (chap3_1);
\path (chap3) to[circle connection bar] (chap3_2);

\end{tikzpicture}
\end{center}

\vspace{1cm}

\begin{center}
\begin{tcolorbox}[
    colback=boxbg,
    colframe=boxborder,
    width=0.8\textwidth,
    arc=5mm,
    boxrule=1mm,
    title=Sobre este Mapa Mental
]
Este mapa mental apresenta a estrutura geral do documento, mostrando a organização hierárquica do conteúdo. Ele está dividido em três seções principais: elementos pré-textuais, textuais e pós-textuais, cada um contendo seus respectivos componentes.
\end{tcolorbox}
\end{center}

% Implementação de um sistema automatizado para geração de mapa mental da estrutura do documento
% Este arquivo fornece funcionalidades para analisar automaticamente a estrutura do documento

% Pacotes necessários para manipulação de listas e strings
\usepackage{etoolbox}
\usepackage{xstring}

% Contador para manter controle de seções para o mapa mental
\newcounter{docmapnodecounter}
\setcounter{docmapnodecounter}{0}

% Armazenamento dos títulos das seções utilizando listas mais robustas
\def\autodocmap@sectionlist{}
\def\autodocmap@subsectionlist{}
\def\autodocmap@subsubsectionlist{}

% Contadores para o número de seções, subseções, etc.
\newcounter{sectioncount}
\newcounter{subsectioncount}
\newcounter{subsubsectioncount}
\newcounter{sectionangle}
\newcounter{subsectionangle}
\newcounter{subsubsectionangle}

% Salva os comandos originais
\let\oldsection\section
\let\oldsubsection\subsection
\let\oldsubsubsection\subsubsection
\let\oldchapter\chapter

% Define os ângulos base para cada nível
\newcommand{\autodocmap@sectionbaseangle}{120}
\newcommand{\autodocmap@subsectionbaseangle}{30}
\newcommand{\autodocmap@subsubsectionbaseangle}{45}

% Armazenar informações do capítulo atual
\newcommand{\autodocmap@currentchapter}{}

% Redefine o comando de capítulo para rastrear títulos
\renewcommand{\chapter}[2][]{%
    \renewcommand{\autodocmap@currentchapter}{#2}%
    \oldchapter[#1]{#2}%
}

% Redefine o comando de seção para rastrear títulos
\renewcommand{\section}[2][]{%
    \stepcounter{sectioncount}%
    \setcounter{subsectioncount}{0}%
    \setcounter{sectionangle}{\value{sectioncount}}%
    \multiply\value{sectionangle} by 60%
    % Armazena os dados da seção para uso posterior no mapa
    \protected@edef\autodocmap@sectionlist{%
        \autodocmap@sectionlist
        \noexpand\docmapleveltwo{sec\arabic{sectioncount}}{textual}{\thesectionangle}{#2}%
    }%
    \oldsection[#1]{#2}%
}

% Redefine o comando de subseção para rastrear títulos
\renewcommand{\subsection}[2][]{%
    \stepcounter{subsectioncount}%
    \setcounter{subsubsectioncount}{0}%
    \setcounter{subsectionangle}{\autodocmap@subsectionbaseangle}%
    \multiply\value{subsectionangle} by \value{subsectioncount}%
    % Armazena os dados da subseção para uso posterior no mapa
    \protected@edef\autodocmap@subsectionlist{%
        \autodocmap@subsectionlist
        \noexpand\docmaplevelthree{subsec\arabic{sectioncount}_\arabic{subsectioncount}}{sec\arabic{sectioncount}}{\thesubsectionangle}{#2}%
    }%
    \oldsubsection[#1]{#2}%
}

% Redefine o comando de subsubseção para rastrear títulos
\renewcommand{\subsubsection}[2][]{%
    \stepcounter{subsubsectioncount}%
    \setcounter{subsubsectionangle}{\autodocmap@subsubsectionbaseangle}%
    \multiply\value{subsubsectionangle} by \value{subsubsectioncount}%
    % Armazena os dados da subsubseção para uso posterior no mapa
    \protected@edef\autodocmap@subsubsectionlist{%
        \autodocmap@subsubsectionlist
        \noexpand\docmaplevelfour{subsubsec\arabic{sectioncount}_\arabic{subsectioncount}_\arabic{subsubsectioncount}}{subsec\arabic{sectioncount}_\arabic{subsectioncount}}{\thesubsubsectionangle}{#2}%
    }%
    \oldsubsubsection[#1]{#2}%
}

% Comando para gerar automaticamente o mapa mental com base nos dados coletados
\newcommand{\generateautodocmap}{%
    \section*{Mapa Automático da Estrutura do Documento}
    
    \begin{docstructmap}
        % Nó raiz - Documento principal
        \docmaproot{Projeto Homebrew\\LaTeX Modular}
        
        % Elementos principais (nível 1)
        \docmaplevelone{pretextual}{30}{Elementos\\Pré-textuais}
        \docmaplevelone{textual}{150}{Elementos\\Textuais}
        \docmaplevelone{postextual}{270}{Elementos\\Pós-textuais}
        
        % Nós pré-definidos para elementos comuns (nível 2)
        \docmapleveltwo{capa}{pretextual}{0}{Capa e Título}
        \docmapleveltwo{toc}{pretextual}{90}{Sumário}
        
        \docmapleveltwo{ref}{postextual}{240}{Referências\\Bibliográficas}
        \docmapleveltwo{gloss}{postextual}{300}{Glossário}
        \docmapleveltwo{ind}{postextual}{340}{Índice Remissivo}
        
        % Gerar nós para cada seção registrada (nível 2)
        \autodocmap@sectionlist
        
        % Gerar nós para cada subseção registrada (nível 3)
        \autodocmap@subsectionlist
        
        % Gerar nós para cada subsubseção registrada (nível 4)
        \autodocmap@subsubsectionlist
        
    \end{docstructmap}
    
    \docmaplegend{
        Este mapa mental foi gerado automaticamente com base na estrutura real do documento.
        Ele representa a hierarquia de capítulos, seções e subseções no formato de um mapa mental.
        As cores indicam diferentes níveis na hierarquia do documento.
    }
}

% Comando para gerar automaticamente um mapa mental focado na estrutura atual
\newcommand{\generatesectiondocmap}[1][Capítulo Atual]{%
    \section*{Mapa da Estrutura do Capítulo}
    
    \begin{docstructmap}
        % Nó raiz - Capítulo atual
        \docmaproot{#1}
        
        % Gerar nós para cada seção registrada (nível 1)
        % Esta é uma versão simplificada que mostra apenas o capítulo atual
        % Em uma implementação mais completa, seria filtrado pelo capítulo atual
        \autodocmap@sectionlist
        
        % Gerar nós para cada subseção registrada (nível 2)
        \autodocmap@subsectionlist
        
    \end{docstructmap}
    
    \docmaplegend{
        Este mapa mental mostra a estrutura específica deste capítulo, 
        destacando suas seções e subseções principais.
    }
}

% Comando para gerar um mapa mental complexo predefinido
\newcommand{\generatecomplexdocmap}{%
    \section*{Mapa da Estrutura do Documento}
    
    \begin{docstructmap}
        % Nó raiz - Documento principal
        \docmaproot{Projeto Homebrew\\LaTeX Modular}
        
        % Elementos principais (nível 1)
        \docmaplevelone{pretextual}{30}{Elementos\\Pré-textuais}
        \docmaplevelone{textual}{150}{Elementos\\Textuais}
        \docmaplevelone{postextual}{270}{Elementos\\Pós-textuais}
        
        % Elementos pré-textuais (nível 2)
        \docmapleveltwo{capa}{pretextual}{0}{Capa e Título}
        \docmapleveltwo{toc}{pretextual}{60}{Sumário}
        \docmapleveltwo{mapament}{pretextual}{120}{Mapa Mental}
        
        % Elementos textuais - Capítulos (nível 2)
        \docmapleveltwo{cap1}{textual}{120}{Capítulo 1\\Introdução}
        \docmapleveltwo{cap2}{textual}{180}{Capítulo 2\\Bibliografia}
        \docmapleveltwo{cap3}{textual}{240}{Capítulo 3\\Glossário e Índice}
        
        % Elementos pós-textuais (nível 2)
        \docmapleveltwo{ref}{postextual}{240}{Referências\\Bibliográficas}
        \docmapleveltwo{gloss}{postextual}{300}{Glossário}
        \docmapleveltwo{ind}{postextual}{340}{Índice Remissivo}
        
        % Conteúdo do Capítulo 1 (nível 3)
        \docmaplevelthree{cap1_1}{cap1}{90}{Ambientes RPG}
        \docmaplevelthree{cap1_2}{cap1}{150}{Comandos\\Personalizados}
        \docmaplevelthree{cap1_3}{cap1}{210}{Mapas Mentais}
        
        % Conteúdo do Capítulo 2 (nível 3)
        \docmaplevelthree{cap2_1}{cap2}{150}{Sistema de Citações}
        \docmaplevelthree{cap2_2}{cap2}{210}{Referências Cruzadas}
        
        % Conteúdo do Capítulo 3 (nível 3)
        \docmaplevelthree{cap3_1}{cap3}{210}{Glossário}
        \docmaplevelthree{cap3_2}{cap3}{270}{Índice Remissivo}
        
        % Exemplos de nível 4 (detalhes mais específicos)
        \docmaplevelfour{amb1}{cap1_1}{60}{Spell}
        \docmaplevelfour{amb2}{cap1_1}{120}{Character}
        \docmaplevelfour{amb3}{cap1_1}{180}{Rule}
        
        \docmaplevelfour{cmd1}{cap1_2}{120}{rpgnote}
        \docmaplevelfour{cmd2}{cap1_2}{180}{rpgsection}
        \docmaplevelfour{cmd3}{cap1_2}{240}{rpgitem}
        
        % Conteúdo do mapa mental (nível 4)
        \docmaplevelfour{map1}{cap1_3}{120}{Estrutura\\Automática}
        \docmaplevelfour{map2}{cap1_3}{180}{Mapas\\Personalizados}
        \docmaplevelfour{map3}{cap1_3}{240}{Visualização\\Hierárquica}
    
    \end{docstructmap}
    
    \docmaplegend{
        Este mapa mental apresenta a estrutura completa do documento, incluindo capítulos, 
        seções e elementos específicos. Os níveis mais externos representam as divisões 
        principais do documento, enquanto os níveis internos mostram os detalhes específicos
        de cada seção.
    }
}

% Comando para simplificar a geração rápida de um mapa mental personalizado
\newcommand{\quickdocmap}[1]{%
    \section*{Mapa da Estrutura: #1}
    
    \begin{docstructmap}
        % Nó raiz personalizado
        \docmaproot{#1}
        
        % Elementos principais predefinidos (nível 1) - podem ser adaptados conforme necessário
        \docmaplevelone{introducao}{30}{Introdução}
        \docmaplevelone{desenvolvimento}{150}{Desenvolvimento}
        \docmaplevelone{conclusao}{270}{Conclusão}
    \end{docstructmap}
    
    \docmaplegend{
        Mapa mental simplificado para "#1". Este tipo de mapa pode ser usado
        para planejar novos documentos ou visualizar conceitos específicos.
    }
}
