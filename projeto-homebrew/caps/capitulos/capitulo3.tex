% Conteúdo do Capítulo 3
\section{Glossário e Índice Remissivo}

Esta seção demonstra como utilizar o glossário e o índice remissivo no projeto.

\subsection{Utilizando o Glossário}

O glossário permite criar uma lista de termos importantes e suas definições. Para adicionar um termo ao glossário, use o comando:

\begin{verbatim}
\glossaryentry{termo}{definição}
\end{verbatim}

Exemplo de termos adicionados ao glossário:

\glossaryentry{arcano}{Termo que se refere à magia ou conhecimento místico.}
\glossaryentry{grimório}{Livro contendo feitiços, encantamentos e instruções mágicas.}
\glossaryentry{runa}{Símbolo mágico utilizado em rituais e encantamentos.}

Para referenciar um termo do glossário no texto, use \verb|\gls{termo}|. Por exemplo: 
\gls{arcano}, \gls{grimório} e \gls{runa}.

\subsection{Utilizando o Índice Remissivo}

O índice remissivo permite que os leitores encontrem facilmente termos específicos no documento. Para adicionar um termo ao índice, use:

\begin{verbatim}
\index{termo}
\end{verbatim}

Exemplos de termos indexados:

Magos\index{magos} são estudantes de magia\index{magia} arcana. Eles aprendem feitiços\index{feitiço} 
através de estudo intenso e prática constante. Muitos magos mantêm grimórios\index{grimório} 
onde registram seus conhecimentos arcanos\index{arcano}.

Druidas\index{druida} são praticantes de magia\index{magia!natural} natural. Eles canalizam a 
energia da natureza\index{natureza} e podem assumir formas animais\index{forma animal}.

\subsection{Demonstração de Ambientes}

\begin{spell}
\textbf{Proteção Arcana}\\
\textit{Abjuração de 2º nível}\\
\textbf{Tempo de Conjuração:} 1 ação\\
\textbf{Alcance:} Toque\\
\textbf{Componentes:} V, S, M (um pequeno diamante)\\
\textbf{Duração:} 1 hora

Você toca uma criatura disposta e cria uma barreira mágica ao seu redor. Até o fim da duração, o alvo recebe +2 na CA e tem vantagem em testes de resistência contra magias.
\end{spell}

\rpgnote{Esta magia é especialmente útil antes de enfrentar criaturas que utilizam magias ofensivas.}

\begin{magicitem}
\textbf{Amuleto de Proteção}\\
\textit{Item maravilhoso, raro (requer sintonização)}

Enquanto estiver usando este amuleto, você recebe +1 nas jogadas de resistência e é imune a magias de adivinhação e efeitos sensoriais mágicos que detectariam sua presença.
\end{magicitem}

\begin{dmnote}
Considere dar este item a personagens que enfrentarão inimigos com forte capacidade de rastreamento mágico ou que precisam realizar missões furtivas.
\end{dmnote}

\rpgsection{Lista de Atributos e Estatísticas}

Aqui demonstramos como apresentar estatísticas de personagens usando os comandos personalizados:

\stat{Força}{16}  \stat{Destreza}{14}  \stat{Constituição}{15}  

\stat{Inteligência}{18}  \stat{Sabedoria}{12}  \stat{Carisma}{10}

Barras de atributos também podem ser usadas para representar níveis de habilidade:

\attrbar{Conjuração}{25}

\attrbar{Combate}{15}

\attrbar{Furtividade}{10}

\begin{rule}
\textbf{Sintonização com Itens Mágicos}

Para sintonizar-se com um item, você deve passar 1 hora em contato físico ininterrupto com ele, concentrando-se nele e tentando compreender suas propriedades. Esta concentração pode ocorrer durante um descanso curto e não pode ser interrompida por nenhuma outra atividade.
\end{rule}

\rpgtitle{Tabela de Encontros Aleatórios}

\begin{rpgtable}
\begin{tabular}{|c|p{10cm}|}
\hline
\textbf{d20} & \textbf{Encontro} \\
\hline
1-3 & 1d4 bandidos tentando roubar viajantes \\
\hline
4-6 & Um mercador com carroça quebrada pedindo ajuda \\
\hline
7-10 & 1d6 lobos caçando na área \\
\hline
11-14 & Um druida realizando um ritual para purificar a terra \\
\hline
15-17 & Uma patrulha de 1d4+1 guardas da cidade \\
\hline
18-19 & Um mago errante estudando fenômenos mágicos locais \\
\hline
20 & Um dragão jovem sobrevoando a região \\
\hline
\end{tabular}
\end{rpgtable}

\begin{quotebox}
"A verdadeira magia consiste em compreender a conexão entre todas as coisas, visíveis e invisíveis. Não é apenas poder, mas sabedoria."

— Arquimago Elminster
\end{quotebox}

\begin{highlight}
\textbf{Dica importante para jogadores:} 
Sempre verifique portas, baús e outros objetos suspeitos em busca de armadilhas antes de interagir com eles. Uma simples verificação pode salvar sua vida!
\end{highlight}