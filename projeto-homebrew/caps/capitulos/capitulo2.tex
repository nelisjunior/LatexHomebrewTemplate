% Conteúdo do Capítulo 2
\section{Bibliografia e Referências}

Este capítulo demonstra como trabalhar com bibliografia e referências no modelo.

\subsection{Sistema de Citações}

O modelo utiliza o pacote biblatex para gerenciamento de referências. Isso permite citar obras de forma consistente e gerar uma bibliografia formatada automaticamente.

\begin{rule}
\textbf{Citações no Texto}

Para citar uma referência no texto, utilize o comando \texttt{\\cite\{chave\}}, onde \texttt{chave} corresponde ao identificador da referência no arquivo \texttt{referencias.bib}.
\end{rule}

Por exemplo, podemos citar a obra seminal de Donald Knuth sobre tipografia digital \cite{knuth1984texbook} ou discutir as contribuições de Leslie Lamport para o LaTeX \cite{lamport1994latex}.

\rpgnote{O estilo de citação pode ser modificado alterando o parâmetro \texttt{style} na configuração do pacote biblatex em \texttt{configuracoes.tex}.}

\subsection{Gerenciando o Arquivo de Referências}

O arquivo \texttt{referencias.bib} contém todas as entradas bibliográficas no formato BibTeX. Cada entrada segue um padrão como este:

\begin{verbatim}
@book{knuth1984texbook,
  title={The TeXbook},
  author={Knuth, Donald E},
  year={1984},
  publisher={Addison-Wesley}
}
\end{verbatim}

\begin{dmnote}
Recomenda-se utilizar ferramentas como JabRef, Zotero ou Mendeley para gerenciar suas referências bibliográficas. Estas ferramentas permitem exportar entradas no formato BibTeX diretamente para o arquivo \texttt{referencias.bib}.
\end{dmnote}

\subsection{Tipos de Citações}

Existem diferentes formas de citar referências:

\begin{rpgtable}
\begin{tabular}{|l|l|p{8cm}|}
\hline
\textbf{Comando} & \textbf{Exemplo} & \textbf{Resultado} \\
\hline
\texttt{\\cite\{chave\}} & \texttt{\\cite\{lamport1994latex\}} & Citação padrão \cite{lamport1994latex} \\
\hline
\texttt{\\parencite\{chave\}} & \texttt{\\parencite\{knuth1984texbook\}} & Citação entre parênteses \parencite{knuth1984texbook} \\
\hline
\texttt{\\textcite\{chave\}} & \texttt{\\textcite\{lamport1994latex\}} & Citação no texto \textcite{lamport1994latex} \\
\hline
\texttt{\\footcite\{chave\}} & \texttt{\\footcite\{knuth1984texbook\}} & Citação em nota de rodapé\footcite{knuth1984texbook} \\
\hline
\end{tabular}
\end{rpgtable}

\subsection{Referências Cruzadas}

Além de referências bibliográficas, o LaTeX permite criar referências cruzadas dentro do seu próprio documento.

\begin{spell}
\textbf{Mecanismo de Referências Cruzadas}\\
\textit{*Feature de nível 3*}

Para criar uma referência cruzada, primeiro rotule um elemento com \texttt{\\label\{identificador\}} e depois referencie-o usando \texttt{\\ref\{identificador\}}.
\end{spell}

Por exemplo, podemos rotular esta seção com \label{sec:refcruzadas} e referenciá-la como "Seção \ref{sec:refcruzadas}".

\begin{highlight}
\textbf{Dica importante:} 
Sempre compile seu documento pelo menos duas vezes para garantir que todas as referências cruzadas sejam resolvidas corretamente. Na primeira compilação, o LaTeX identifica os rótulos e suas posições; na segunda, ele substitui as referências pelos números corretos.
\end{highlight}

\begin{quotebox}
"O verdadeiro poder do LaTeX está em sua capacidade de gerenciar automaticamente aspectos como numeração, referências e formatação bibliográfica, permitindo que o autor se concentre exclusivamente no conteúdo."

— Um entusiasta do LaTeX
\end{quotebox}