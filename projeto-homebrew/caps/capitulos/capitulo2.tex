\chapter{Utilizando o Modelo}

Este capítulo apresenta orientações sobre como utilizar e personalizar este modelo LaTeX.

\section{Estrutura de Arquivos}

A estrutura modular deste modelo permite que cada parte do documento seja mantida em um arquivo separado, facilitando a organização e manutenção.

\begin{highlight}
Os arquivos estão organizados da seguinte forma:
\begin{itemize}
    \item \textbf{main.tex:} Arquivo principal que compila todo o documento
    \item \textbf{configuracoes.tex:} Contém os pacotes e configurações do documento
    \item \textbf{ambientes.tex:} Define os ambientes personalizados com tcolorbox
    \item \textbf{0-pretextual.tex:} Carrega os elementos pré-textuais
    \item \textbf{1-textual.tex:} Carrega os elementos textuais
    \item \textbf{2-postextual.tex:} Carrega os elementos pós-textuais
    \item \textbf{caps/pretextual.tex:} Conteúdo adicional dos elementos pré-textuais
    \item \textbf{caps/textual.tex:} Gerencia a inclusão dos capítulos
    \item \textbf{caps/postextual.tex:} Conteúdo dos elementos pós-textuais
    \item \textbf{caps/capitulos/:} Diretório com os arquivos de cada capítulo
\end{itemize}
\end{highlight}

\section{Personalização}

O modelo pode ser personalizado de diversas formas para atender às necessidades específicas do projeto.

\subsection{Cores e Estilo}

As cores utilizadas no documento estão definidas no arquivo \texttt{configuracoes.tex}. Para alterar o esquema de cores, modifique as definições:

\begin{quotebox}
\verb|\definecolor{pergaminho}{RGB}{249, 240, 181}|\\
\verb|\definecolor{capa}{RGB}{121, 26, 25}|\\
\verb|\definecolor{titulo}{RGB}{72, 26, 19}|\\
\verb|\definecolor{boxbg}{RGB}{253, 245, 196}|\\
\verb|\definecolor{boxborder}{RGB}{190, 150, 86}|\\
\verb|\definecolor{secaotitulo}{RGB}{140, 26, 20}|
\end{quotebox}

\subsection{Fontes}

O modelo utiliza o pacote \texttt{fontspec} para carregar fontes personalizadas. Por padrão, tenta utilizar a fonte Dominican ou Luxurious Roman. Para utilizar outra fonte, modifique as configurações no arquivo \texttt{configuracoes.tex}:

\begin{quotebox}
\verb|\setmainfont{NomeDaFonte}[|\\
\verb|    Path = /caminho/para/fonte/,|\\
\verb|    Extension = .ttf,|\\
\verb|    UprightFont = *-Regular,|\\
\verb|    BoldFont = *-Bold,|\\
\verb|    ItalicFont = *-Italic,|\\
\verb|    BoldItalicFont = *-BoldItalic,|\\
\verb|    Renderer = Basic|\\
\verb|]|
\end{quotebox}

\section{Adicionando Conteúdo}

Para adicionar novos capítulos ao documento, siga os passos:

\begin{rpgtable}
\begin{enumerate}
    \item Crie um novo arquivo .tex na pasta \texttt{caps/capitulos/}
    \item Adicione o conteúdo do capítulo, começando com \verb|\chapter{Título do Capítulo}|
    \item Inclua o novo capítulo no arquivo \texttt{caps/textual.tex} usando \verb|\input{caps/capitulos/nomeDoArquivo}|
\end{enumerate}
\end{rpgtable}

\section{Tabelas e Figuras}

O modelo suporta tabelas e figuras com estilo personalizado:

\begin{rpgtable}
\begin{tabular}{lcr}
\toprule
\textbf{Classe} & \textbf{Dado de Vida} & \textbf{Habilidade Principal} \\
\midrule
Bárbaro & d12 & Força \\
Bardo & d8 & Carisma \\
Clérigo & d8 & Sabedoria \\
Druida & d8 & Sabedoria \\
Guerreiro & d10 & Força ou Destreza \\
Ladino & d8 & Destreza \\
Mago & d6 & Inteligência \\
Monge & d8 & Destreza e Sabedoria \\
Paladino & d10 & Força e Carisma \\
Patrulheiro & d10 & Destreza e Sabedoria \\
Feiticeiro & d6 & Carisma \\
Bruxo & d8 & Carisma \\
\bottomrule
\end{tabular}
\end{rpgtable}

Para incluir figuras:

\begin{quotebox}
\verb|\begin{figure}[htb]|\\
\verb|    \centering|\\
\verb|    \includegraphics[width=0.7\textwidth]{imgs/nomeDaImagem}|\\
\verb|    \caption{Descrição da figura}|\\
\verb|    \label{fig:identificador}|\\
\verb|\end{figure}|
\end{quotebox}

\section{Bibliografia}

O modelo utiliza o pacote \texttt{biblatex} para gerenciar as referências bibliográficas. Para adicionar novas referências, edite o arquivo \texttt{referencias.bib} e cite-as no texto usando \verb|\cite{chave}|.

Para imprimir a bibliografia, o comando \verb|\printbibliography| já está incluído no arquivo \texttt{caps/postextual.tex}.
