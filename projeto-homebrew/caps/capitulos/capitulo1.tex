\chapter{Ambientes Personalizados}

Este capítulo apresenta os diferentes ambientes personalizados disponíveis neste modelo, inspirados no estilo visual de livros de RPG. Todo o visual foi inspirado pelo estilo do \textit{Homebrewery} \cite{homebrewery}, combinado com a estrutura de documentos acadêmicos.

\rpgnote{Este modelo utiliza fontes personalizadas que funcionam apenas com XeLaTeX. Certifique-se de compilar o documento com este mecanismo.}

\section{Caixas de Destaque}

As caixas de destaque são utilizadas para ressaltar informações importantes, regras, itens mágicos, entre outros elementos \cite{wizardsdnd}. A formatação visual de caixas coloridas é uma característica marcante dos livros de RPG modernos \cite{wizardsdnd5e, paizopathfinder}.

\begin{magicitem}
\rpgtitle{Espada Flamejante}

Esta espada mágica concede ao portador a capacidade de conjurar chamas. Quando empunhada, pequenas labaredas dançam ao longo da lâmina sem causar dano ao portador.

\textbf{Propriedades:}
\begin{itemize}
    \item +1 de bônus em jogadas de ataque e dano
    \item Adiciona 1d6 de dano de fogo ao acertar
    \item Pode lançar a magia \textit{Mãos Flamejantes} uma vez por dia
\end{itemize}
\end{magicitem}

\begin{spell}
\rpgtitle{Rajada Arcana}

\textbf{Tempo de Conjuração:} 1 ação\\
\textbf{Alcance:} 36 metros\\
\textbf{Componentes:} V, S\\
\textbf{Duração:} Instantânea

Três dardos de energia mágica surgem das pontas dos seus dedos e atingem criaturas à sua escolha dentro do alcance. Cada dardo causa 1d4+1 de dano de força.

\textbf{Em Níveis Superiores:} Quando lançada usando um espaço de magia de 2º nível ou superior, a magia cria um dardo adicional para cada nível do espaço acima do 1º.
\end{spell}

\section{Estatísticas de Personagens}

Este modelo também inclui comandos para exibir estatísticas de personagens no estilo RPG:

\begin{character}
\rpgtitle{Thorian, o Sábio}

\begin{multicols}{2}
\textbf{Raça:} Humano\\
\textbf{Classe:} Mago\\
\textbf{Nível:} 5

\columnbreak

\stat{FOR}{8}\\
\stat{DES}{14}\\
\stat{CON}{12}\\
\stat{INT}{17}\\
\stat{SAB}{13}\\
\stat{CAR}{10}
\end{multicols}

\textbf{Habilidades:}
\begin{itemize}
    \item \textbf{Tradição Arcana:} Especialista em magia de evocação
    \item \textbf{Familiar:} Coruja chamada Arquimedes
    \item \textbf{Linguagens:} Comum, Élfico, Anão, Dracônico
\end{itemize}

\textbf{Vitalidade:}\\
\attrbar{25/30}{25}

\textbf{Mana:}\\
\attrbar{18/20}{18}
\end{character}

\section{Notas do Mestre}

Use o ambiente dmnote para incluir informações exclusivas para o Mestre:

\begin{dmnote}
As estatísticas dos inimigos foram balanceadas para um grupo de 4-5 personagens de nível 3-4. Ajuste conforme necessário para se adequar ao seu grupo.

Se os jogadores tentarem negociar com o líder bandido, ele pode oferecer informações sobre o culto secreto em troca de sua liberdade. No entanto, ele tentará enganar o grupo na primeira oportunidade.
\end{dmnote}

\section{Regras Especiais}

O ambiente rule é ideal para destacar regras importantes:

\begin{rule}
\rpgtitle{Descanso Curto}

Um descanso curto é um período de pelo menos 1 hora, durante o qual um personagem não faz nada mais exigente do que comer, beber, ler e tratar de ferimentos.

Um personagem pode gastar um ou mais Dados de Vida ao fim de um descanso curto, até o máximo de Dados de Vida do personagem, que é igual ao seu nível. Para cada Dado de Vida gasto, o jogador rola o dado e adiciona o modificador de Constituição do personagem. O personagem recupera pontos de vida igual ao total. O jogador pode decidir gastar um Dado de Vida adicional depois de cada rolagem.
\end{rule}
