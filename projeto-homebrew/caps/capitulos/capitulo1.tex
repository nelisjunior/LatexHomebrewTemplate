% Conteúdo do Capítulo 1
\section{Introdução aos Ambientes de RPG}

Este capítulo apresenta uma visão geral dos ambientes personalizados disponíveis neste modelo LaTeX inspirado em RPG.

\subsection{A Estética RPG em Documentos Acadêmicos}

Combinar a estética visual de livros de RPG com o rigor e organização de documentos acadêmicos permite criar materiais que são visualmente atraentes e, ao mesmo tempo, informativos e bem estruturados.

\rpgnote{Este modelo é particularmente útil para criar manuais de RPG, livros de regras para jogos de tabuleiro, ou qualquer documento que se beneficie de uma estética de "pergaminho".}

\begin{dmnote}
Esta caixa destaca informações importantes para o mestre (ou, no contexto acadêmico, para o professor ou orientador). É ideal para notas metodológicas ou instruções especiais.
\end{dmnote}

\subsection{Ambientes Principais}

O modelo inclui diversos ambientes personalizados, cada um com um estilo visual distinto:

\begin{spell}
\textbf{Ambiente Spell (Feitiço)}

Este ambiente é ideal para destacar elementos especiais, como fórmulas matemáticas importantes, algoritmos, ou definições fundamentais.
\end{spell}

\begin{character}
\textbf{Ambiente Character (Personagem)}

Perfeito para biografias, perfis de estudo de caso, ou descrições de entidades importantes no seu trabalho acadêmico.
\end{character}

\begin{rule}
\textbf{Ambiente Rule (Regra)}

Use este ambiente para destacar regras, princípios, teoremas, ou outras declarações normativas que são centrais ao seu trabalho.
\end{rule}

\begin{magicitem}
\textbf{Ambiente Magic Item (Item Mágico)}

Ideal para destacar ferramentas, recursos ou conceitos especiais que têm um papel significativo no seu trabalho.
\end{magicitem}

\rpgsection{Comandos Personalizados}

Além dos ambientes, o modelo também oferece comandos personalizados:

\begin{itemize}
\rpgitem{O comando \texttt{\\rpgnote\{\}} permite adicionar notas na margem}
\rpgitem{O comando \texttt{\\rpgsection\{\}} cria cabeçalhos estilizados}
\rpgitem{O comando \texttt{\\rpgtitle\{\}} cria títulos destacados}
\rpgitem{Os comandos \texttt{\\stat\{\}\{\}} e \texttt{\\attrbar\{\}\{\}} são úteis para apresentar métricas}
\end{itemize}

Exemplo de métricas usando \texttt{\\stat}:

\stat{Clareza}{18} \stat{Organização}{16} \stat{Impacto Visual}{20}

E exemplo de barras de atributos usando \texttt{\\attrbar}:

\attrbar{Facilidade de Uso}{25}
\attrbar{Flexibilidade}{20}
\attrbar{Compatibilidade}{18}

\begin{quotebox}
"A visualização adequada de informações é tão importante quanto o próprio conteúdo. Um documento bem formatado convida à leitura e facilita a compreensão."

— Autor Desconhecido
\end{quotebox}

\begin{highlight}
O estilo visual inspirado em RPG não substitui o conteúdo acadêmico rigoroso. Use estes recursos para enriquecer e destacar seu conteúdo, mantendo sempre o rigor e a clareza característicos de trabalhos acadêmicos.
\end{highlight}

\section{Mapas Mentais da Estrutura do Documento}

Uma das características mais úteis deste modelo é a capacidade de gerar automaticamente mapas mentais que visualizam a estrutura do documento. Esta funcionalidade ajuda leitores e autores a compreender a organização do conteúdo de forma visual e intuitiva.

\subsection{Mapas Mentais Predefinidos}

O modelo inclui mapas mentais pré-definidos que podem ser incluídos em qualquer parte do documento usando o comando \texttt{\\input\{caps/pretextual/docmap\}}. Este mapa é estático e apresenta uma estrutura típica de documento acadêmico.

\subsection{Mapas Mentais Automatizados}

Além dos mapas pré-definidos, o modelo também oferece funcionalidades para gerar mapas mentais automaticamente baseados na estrutura real do documento. Existem dois comandos principais para esta função:

\begin{itemize}
\rpgitem{O comando \texttt{\\generateautodocmap} gera um mapa básico baseado na estrutura atual do documento}
\rpgitem{O comando \texttt{\\generatecomplexdocmap} gera um mapa detalhado com vários níveis de hierarquia}
\end{itemize}

\rpgnote{Os mapas mentais são gerados usando o pacote TikZ e podem ser personalizados conforme necessário. Veja o arquivo \texttt{ambientes/docmap.tex} para mais detalhes.}

\begin{spell}
\textbf{Criando Mapas Mentais Personalizados}

Para criar um mapa mental personalizado, use o ambiente \texttt{docstructmap} e os comandos \texttt{\\docmaproot}, \texttt{\\docmaplevelone}, \texttt{\\docmapleveltwo}, etc. Exemplo:

\begin{verbatim}
\begin{docstructmap}
    \docmaproot{Título Principal}
    \docmaplevelone{secao1}{30}{Seção 1}
    \docmapleveltwo{subsec1}{secao1}{60}{Subseção 1.1}
\end{docstructmap}
\end{verbatim}
\end{spell}

\begin{dmnote}
Os mapas mentais são especialmente úteis para documentos longos ou complexos, pois fornecem uma visão geral clara da estrutura. Considere incluir um no início de cada capítulo principal para orientar o leitor.
\end{dmnote}

\subsection{Exemplo de Mapa Mental Simples}

Abaixo está um exemplo de um mapa mental personalizado para um capítulo:

\begin{docstructmap}
    % Nó raiz - Documento principal
    \docmaproot{Capítulo 1\\Introdução}
    
    % Seções principais (nível 1)
    \docmaplevelone{secao1}{30}{Ambientes\\de RPG}
    \docmaplevelone{secao2}{150}{Comandos\\Personalizados}
    \docmaplevelone{secao3}{270}{Mapas\\Mentais}
    
    % Subseções da Seção 1 (nível 2)
    \docmapleveltwo{subsec1_1}{secao1}{0}{Estética\\Visual}
    \docmapleveltwo{subsec1_2}{secao1}{60}{Ambientes\\Principais}
    
    % Subseções da Seção 2 (nível 2)
    \docmapleveltwo{subsec2_1}{secao2}{120}{Comandos\\de Formatação}
    \docmapleveltwo{subsec2_2}{secao2}{180}{Comandos\\de Estilo}
    
    % Subseções da Seção 3 (nível 2)
    \docmapleveltwo{subsec3_1}{secao3}{240}{Mapas\\Predefinidos}
    \docmapleveltwo{subsec3_2}{secao3}{300}{Mapas\\Automáticos}
\end{docstructmap}

\docmaplegend{
    Este mapa mental ilustra a estrutura do Capítulo 1, mostrando as principais seções e 
    suas respectivas subseções. Os mapas mentais ajudam a visualizar a organização do 
    conteúdo e podem ser criados manualmente ou gerados automaticamente.
}