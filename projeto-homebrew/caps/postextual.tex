% Conteúdo dos elementos pós-textuais

\cleardoublepage
\chapter*{Conclusão}

Este documento demonstra as capacidades do modelo modular LaTeX com estética de RPG. Combine a organização de documentos acadêmicos com o estilo visual dos livros de RPG para criar documentos únicos e interessantes.

\cleardoublepage
\chapter*{Glossário}

\begin{description}[style=nextline, leftmargin=0cm]
    \item[LaTeX] Sistema de tipografia que permite a criação de documentos com alta qualidade tipográfica.
    \item[RPG] Role-Playing Game, jogo onde os jogadores assumem os papéis de personagens em um cenário fictício.
    \item[Homebrewery] Estilo de formatação inspirado em livros de RPG, especialmente Dungeons \& Dragons.
    \item[Modular] Estrutura organizada em componentes independentes que podem ser combinados.
\end{description}

\cleardoublepage
\chapter*{Apêndices}

\begin{rpgtable}
\begin{tabular}{lcc}
\toprule
\textbf{Item} & \textbf{Categoria} & \textbf{Valor} \\
\midrule
Espada Longa & Arma & 15 po \\
Poção de Cura & Consumível & 50 po \\
Grimório & Mágico & 100 po \\
\bottomrule
\end{tabular}
\end{rpgtable}

\cleardoublepage
\printbibliography[title=Referências]
