% Mapa da estrutura do documento
\section*{Mapa do Documento}

\begin{center}
\begin{tikzpicture}[
    mindmap,
    level 1 concept/.append style={font=\large\bfseries, sibling angle=60, level distance=5cm},
    level 2 concept/.append style={font=\normalsize\bfseries, sibling angle=45, level distance=3.5cm},
    level 3 concept/.append style={font=\small, sibling angle=40, level distance=2.5cm},
    concept/.append style={
        text width=4cm, 
        font=\bfseries, 
        minimum size=2cm, 
        fill=boxbg, 
        text=secaotitulo, 
        line width=1pt, 
        draw=boxborder
    },
    concept connection/.append style={line width=1pt, draw=boxborder}
]

% Nó central/raiz - Título do documento
\node[concept, font=\Large\bfseries, minimum size=3cm, fill=capa!40!boxbg, text=white] (doc) {Projeto Homebrew\\LaTeX Modular};

% Elementos pré-textuais - Nível 1
\node[concept, fill=spell!10!boxbg] (pretextual) [grow=30] at (doc.30) {Elementos Pré-textuais};

% Elementos textuais - Nível 1
\node[concept, fill=magicitem!10!boxbg] (textual) [grow=150] at (doc.150) {Elementos Textuais};

% Elementos pós-textuais - Nível 1
\node[concept, fill=rule!10!boxbg] (postextual) [grow=270] at (doc.270) {Elementos Pós-textuais};

% Conexões do nó central aos nós de nível 1
\path (doc) to[circle connection bar] (pretextual);
\path (doc) to[circle connection bar] (textual);
\path (doc) to[circle connection bar] (postextual);

% Elementos pré-textuais - Nível 2
\node[concept, fill=spell!5!boxbg] (title) [grow=0] at (pretextual.0) {Título e Capa};
\node[concept, fill=spell!5!boxbg] (summary) [grow=90] at (pretextual.90) {Sumário};

% Elementos textuais - Nível 2
\node[concept, fill=magicitem!5!boxbg] (chap1) [grow=120] at (textual.120) {Capítulo 1\\Introdução};
\node[concept, fill=magicitem!5!boxbg] (chap2) [grow=180] at (textual.180) {Capítulo 2\\Bibliografia};
\node[concept, fill=magicitem!5!boxbg] (chap3) [grow=240] at (textual.240) {Capítulo 3\\Glossário e Índice};

% Elementos pós-textuais - Nível 2
\node[concept, fill=rule!5!boxbg] (refs) [grow=270] at (postextual.270) {Referências};
\node[concept, fill=rule!5!boxbg] (glossary) [grow=320] at (postextual.320) {Glossário};
\node[concept, fill=rule!5!boxbg] (index) [grow=220] at (postextual.220) {Índice Remissivo};

% Conexões dos nós de nível 1 aos nós de nível 2
\path (pretextual) to[circle connection bar] (title);
\path (pretextual) to[circle connection bar] (summary);

\path (textual) to[circle connection bar] (chap1);
\path (textual) to[circle connection bar] (chap2);
\path (textual) to[circle connection bar] (chap3);

\path (postextual) to[circle connection bar] (refs);
\path (postextual) to[circle connection bar] (glossary);
\path (postextual) to[circle connection bar] (index);

% Elementos do capítulo 1 - Nível 3
\node[concept, scale=0.7, fill=magicitem!2!boxbg] (chap1_1) [grow=90] at (chap1.90) {Ambientes RPG};
\node[concept, scale=0.7, fill=magicitem!2!boxbg] (chap1_2) [grow=150] at (chap1.150) {Comandos Personalizados};

% Elementos do capítulo 2 - Nível 3
\node[concept, scale=0.7, fill=magicitem!2!boxbg] (chap2_1) [grow=150] at (chap2.150) {Sistema de Citações};
\node[concept, scale=0.7, fill=magicitem!2!boxbg] (chap2_2) [grow=210] at (chap2.210) {Referências Cruzadas};

% Elementos do capítulo 3 - Nível 3
\node[concept, scale=0.7, fill=magicitem!2!boxbg] (chap3_1) [grow=210] at (chap3.210) {Glossário};
\node[concept, scale=0.7, fill=magicitem!2!boxbg] (chap3_2) [grow=270] at (chap3.270) {Índice Remissivo};

% Conexões dos nós de nível 2 aos nós de nível 3
\path (chap1) to[circle connection bar] (chap1_1);
\path (chap1) to[circle connection bar] (chap1_2);

\path (chap2) to[circle connection bar] (chap2_1);
\path (chap2) to[circle connection bar] (chap2_2);

\path (chap3) to[circle connection bar] (chap3_1);
\path (chap3) to[circle connection bar] (chap3_2);

\end{tikzpicture}
\end{center}

\vspace{1cm}

\begin{center}
\begin{tcolorbox}[
    colback=boxbg,
    colframe=boxborder,
    width=0.8\textwidth,
    arc=5mm,
    boxrule=1mm,
    title=Sobre este Mapa Mental
]
Este mapa mental apresenta a estrutura geral do documento, mostrando a organização hierárquica do conteúdo. Ele está dividido em três seções principais: elementos pré-textuais, textuais e pós-textuais, cada um contendo seus respectivos componentes.
\end{tcolorbox}
\end{center}
