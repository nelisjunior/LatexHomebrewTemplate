% Versão simplificada dos mapas mentais para maior compatibilidade
% Este arquivo fornece uma versão mais leve e compatível com Overleaf

% Definição de cores para os diferentes níveis do mapa mental
\definecolor{level0color}{RGB}{121, 26, 25}  % Raiz (capa)
\definecolor{level1color}{RGB}{70, 26, 100}  % Nível 1 (spell)
\definecolor{level2color}{RGB}{26, 70, 100}  % Nível 2 (magicitem)
\definecolor{level3color}{RGB}{140, 20, 20}  % Nível 3 (rule)

% Ambiente para o mapa mental da estrutura do documento
\newenvironment{docstructmap}[1][]{%
    \begin{center}
    \begin{tikzpicture}[
        mindmap,
        level 1/.style={font=\large\bfseries, sibling angle=60, level distance=5cm},
        level 2/.style={font=\normalsize\bfseries, sibling angle=45, level distance=3.5cm},
        level 3/.style={font=\small, sibling angle=40, level distance=2.5cm},
        level 4/.style={font=\scriptsize, sibling angle=30, level distance=2cm},
        every node/.style={
            text width=4cm, 
            font=\bfseries, 
            minimum size=2cm, 
            fill=boxbg, 
            text=secaotitulo, 
            line width=1pt, 
            draw=boxborder
        },
        every edge/.style={line width=1pt, draw=boxborder}
    ]
}{%
    \end{tikzpicture}
    \end{center}
}

% Comando para adicionar o nó raiz do mapa mental
\newcommand{\docmaproot}[2][]{%
    \node[font=\Large\bfseries, minimum size=3cm, fill=level0color!40!boxbg, text=white, #1] (docroot) {#2};
}

% Comando para adicionar um nó de nível 1
\newcommand{\docmaplevelone}[4][]{%
    % #1 = opções adicionais
    % #2 = ID do nó
    % #3 = ângulo de crescimento
    % #4 = conteúdo do nó
    \node[fill=level1color!10!boxbg, #1] (#2) at (\thesection*#3:5cm) {#4};
    \draw[->] (docroot) -- (#2);
}

% Comando para adicionar um nó de nível 2
\newcommand{\docmapleveltwo}[5][]{%
    % #1 = opções adicionais
    % #2 = ID do nó
    % #3 = ID do nó pai
    % #4 = ângulo de crescimento
    % #5 = conteúdo do nó
    \node[fill=level2color!5!boxbg, #1] (#2) at (#3) ++(\thesection*#4:3.5cm) {#5};
    \draw[->] (#3) -- (#2);
}

% Comando para adicionar um nó de nível 3
\newcommand{\docmaplevelthree}[5][]{%
    % #1 = opções adicionais
    % #2 = ID do nó
    % #3 = ID do nó pai
    % #4 = ângulo de crescimento
    % #5 = conteúdo do nó
    \node[scale=0.7, fill=level3color!3!boxbg, #1] (#2) at (#3) ++(\thesection*#4:2.5cm) {#5};
    \draw[->] (#3) -- (#2);
}

% Comando para adicionar um nó de nível 4
\newcommand{\docmaplevelfour}[5][]{%
    % #1 = opções adicionais
    % #2 = ID do nó
    % #3 = ID do nó pai
    % #4 = ângulo de crescimento
    % #5 = conteúdo do nó
    \node[scale=0.5, fill=level3color!2!boxbg, #1] (#2) at (#3) ++(\thesection*#4:2cm) {#5};
    \draw[->] (#3) -- (#2);
}

% Comando para criar uma legenda para o mapa mental
\newcommand{\docmaplegend}[1][]{%
    \begin{center}
    \begin{tcolorbox}[
        colback=boxbg,
        colframe=boxborder,
        width=0.8\textwidth,
        arc=5mm,
        boxrule=1mm,
        title=Sobre este Mapa Mental
    ]
    #1
    \end{tcolorbox}
    \end{center}
}

% Comandos simplificados para gerar mapas mentais
\newcommand{\generateautodocmap}{%
    \section*{Mapa da Estrutura do Documento}
    
    \begin{docstructmap}
        % Nó raiz - Documento principal
        \docmaproot{Projeto Homebrew\\LaTeX Modular}
        
        % Elementos principais (nível 1)
        \docmaplevelone{pretextual}{1}{Elementos\\Pré-textuais}
        \docmaplevelone{textual}{3}{Elementos\\Textuais}
        \docmaplevelone{postextual}{5}{Elementos\\Pós-textuais}
        
        % Elementos textuais - Capítulos (nível 2)
        \docmapleveltwo{cap1}{textual}{2}{Capítulo 1\\Introdução}
        \docmapleveltwo{cap2}{textual}{3}{Capítulo 2\\Conteúdo}
        \docmapleveltwo{cap3}{textual}{4}{Capítulo 3\\Conclusão}
    \end{docstructmap}
    
    \docmaplegend{
        Este mapa mental foi gerado automaticamente e mostra a estrutura básica do documento.
        A versão simplificada é projetada para maior compatibilidade com diversos sistemas LaTeX.
    }
}

% Comando simplificado para mapa mental complexo
\newcommand{\generatecomplexdocmap}{%
    \section*{Mapa da Estrutura do Documento}
    
    \begin{docstructmap}
        % Nó raiz - Documento principal
        \docmaproot{Projeto Homebrew\\LaTeX Modular}
        
        % Elementos principais (nível 1)
        \docmaplevelone{pretextual}{1}{Elementos\\Pré-textuais}
        \docmaplevelone{textual}{3}{Elementos\\Textuais}
        \docmaplevelone{postextual}{5}{Elementos\\Pós-textuais}
        
        % Elementos pré-textuais (nível 2)
        \docmapleveltwo{capa}{pretextual}{0}{Capa e Título}
        \docmapleveltwo{toc}{pretextual}{1}{Sumário}
        \docmapleveltwo{mapament}{pretextual}{2}{Mapa Mental}
        
        % Elementos textuais - Capítulos (nível 2)
        \docmapleveltwo{cap1}{textual}{2}{Capítulo 1\\Introdução}
        \docmapleveltwo{cap2}{textual}{3}{Capítulo 2\\Conteúdo}
        \docmapleveltwo{cap3}{textual}{4}{Capítulo 3\\Conclusão}
        
        % Elementos pós-textuais (nível 2)
        \docmapleveltwo{ref}{postextual}{4}{Referências\\Bibliográficas}
        \docmapleveltwo{gloss}{postextual}{5}{Glossário}
        \docmapleveltwo{ind}{postextual}{6}{Índice Remissivo}
    \end{docstructmap}
    
    \docmaplegend{
        Este mapa mental apresenta a estrutura completa do documento, incluindo capítulos
        e elementos pré e pós-textuais. Esta versão simplificada é otimizada para 
        compatibilidade com o Overleaf.
    }
}

% Comando para mapa mental de capítulo específico
\newcommand{\generatesectiondocmap}[1][Capítulo Atual]{%
    \section*{Mapa da Estrutura do Capítulo}
    
    \begin{docstructmap}
        % Nó raiz - Capítulo atual
        \docmaproot{#1}
        
        % Seções do capítulo (nível 1)
        \docmaplevelone{sec1}{1}{Seção 1}
        \docmaplevelone{sec2}{3}{Seção 2}
        \docmaplevelone{sec3}{5}{Seção 3}
        
        % Subseções (nível 2)
        \docmapleveltwo{subsec1}{sec1}{0}{Subseção 1.1}
        \docmapleveltwo{subsec2}{sec1}{1}{Subseção 1.2}
        
        \docmapleveltwo{subsec3}{sec2}{3}{Subseção 2.1}
        \docmapleveltwo{subsec4}{sec2}{4}{Subseção 2.2}
    \end{docstructmap}
    
    \docmaplegend{
        Este mapa mental mostra a estrutura deste capítulo e suas seções.
        Use este mapa para orientar-se na leitura ou edição do capítulo.
    }
}