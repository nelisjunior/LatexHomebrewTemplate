% Funções automatizadas para geração de mapas mentais de estrutura de documentos

% Definição de cores para os diferentes níveis do mapa mental
\definecolor{level0color}{RGB}{121, 26, 25}  % Raiz (capa)
\definecolor{level1color}{RGB}{70, 26, 100}  % Nível 1 (spell)
\definecolor{level2color}{RGB}{26, 70, 100}  % Nível 2 (magicitem)
\definecolor{level3color}{RGB}{140, 20, 20}  % Nível 3 (rule)
\definecolor{level4color}{RGB}{20, 100, 120} % Nível 4 (note)

% Ambiente para o mapa mental da estrutura do documento
\newenvironment{docstructmap}[1][]{%
    \begin{center}
    \begin{tikzpicture}[
        mindmap,
        level 1 concept/.append style={font=\large\bfseries, sibling angle=60, level distance=5cm},
        level 2 concept/.append style={font=\normalsize\bfseries, sibling angle=45, level distance=3.5cm},
        level 3 concept/.append style={font=\small, sibling angle=40, level distance=2.5cm},
        level 4 concept/.append style={font=\scriptsize, sibling angle=30, level distance=2cm},
        concept/.append style={
            text width=4cm, 
            font=\bfseries, 
            minimum size=2cm, 
            fill=boxbg, 
            text=secaotitulo, 
            line width=1pt, 
            draw=boxborder
        },
        concept connection/.append style={line width=1pt, draw=boxborder}
    ]
}{%
    \end{tikzpicture}
    \end{center}
}

% Comando para adicionar o nó raiz do mapa mental
\newcommand{\docmaproot}[2][]{%
    \node[concept, font=\Large\bfseries, minimum size=3cm, fill=level0color!40!boxbg, text=white, #1] (docroot) {#2};
}

% Comando para adicionar um nó de nível 1
\newcommand{\docmaplevelone}[4][]{%
    % #1 = opções adicionais
    % #2 = ID do nó
    % #3 = ângulo de crescimento
    % #4 = conteúdo do nó
    \node[concept, fill=level1color!10!boxbg, #1] (#2) [grow=#3] at (docroot.#3) {#4};
    \path (docroot) to[circle connection bar] (#2);
}

% Comando para adicionar um nó de nível 2
\newcommand{\docmapleveltwo}[5][]{%
    % #1 = opções adicionais
    % #2 = ID do nó
    % #3 = ID do nó pai
    % #4 = ângulo de crescimento
    % #5 = conteúdo do nó
    \node[concept, fill=level2color!5!boxbg, #1] (#2) [grow=#4] at (#3.#4) {#5};
    \path (#3) to[circle connection bar] (#2);
}

% Comando para adicionar um nó de nível 3
\newcommand{\docmaplevelthree}[5][]{%
    % #1 = opções adicionais
    % #2 = ID do nó
    % #3 = ID do nó pai
    % #4 = ângulo de crescimento
    % #5 = conteúdo do nó
    \node[concept, scale=0.7, fill=level3color!3!boxbg, #1] (#2) [grow=#4] at (#3.#4) {#5};
    \path (#3) to[circle connection bar] (#2);
}

% Comando para adicionar um nó de nível 4
\newcommand{\docmaplevelfour}[5][]{%
    % #1 = opções adicionais
    % #2 = ID do nó
    % #3 = ID do nó pai
    % #4 = ângulo de crescimento
    % #5 = conteúdo do nó
    \node[concept, scale=0.5, fill=level4color!2!boxbg, #1] (#2) [grow=#4] at (#3.#4) {#5};
    \path (#3) to[circle connection bar] (#2);
}

% Comando para criar uma legenda para o mapa mental
\newcommand{\docmaplegend}[1][]{%
    \begin{center}
    \begin{tcolorbox}[
        colback=boxbg,
        colframe=boxborder,
        width=0.8\textwidth,
        arc=5mm,
        boxrule=1mm,
        title=Sobre este Mapa Mental
    ]
    #1
    \end{tcolorbox}
    \end{center}
}