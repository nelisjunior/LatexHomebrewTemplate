% Implementação de um sistema automatizado para geração de mapa mental da estrutura do documento
% Este arquivo fornece funcionalidades para analisar automaticamente a estrutura do documento

% Pacotes necessários para manipulação de listas e strings
\usepackage{etoolbox}
\usepackage{xstring}

% Contador para manter controle de seções para o mapa mental
\newcounter{docmapnodecounter}
\setcounter{docmapnodecounter}{0}

% Armazenamento dos títulos das seções utilizando listas mais robustas
\def\autodocmap@sectionlist{}
\def\autodocmap@subsectionlist{}
\def\autodocmap@subsubsectionlist{}

% Contadores para o número de seções, subseções, etc.
\newcounter{sectioncount}
\newcounter{subsectioncount}
\newcounter{subsubsectioncount}
\newcounter{sectionangle}
\newcounter{subsectionangle}
\newcounter{subsubsectionangle}

% Salva os comandos originais
\let\oldsection\section
\let\oldsubsection\subsection
\let\oldsubsubsection\subsubsection
\let\oldchapter\chapter

% Define os ângulos base para cada nível
\newcommand{\autodocmap@sectionbaseangle}{120}
\newcommand{\autodocmap@subsectionbaseangle}{30}
\newcommand{\autodocmap@subsubsectionbaseangle}{45}

% Armazenar informações do capítulo atual
\newcommand{\autodocmap@currentchapter}{}

% Redefine o comando de capítulo para rastrear títulos
\renewcommand{\chapter}[2][]{%
    \renewcommand{\autodocmap@currentchapter}{#2}%
    \oldchapter[#1]{#2}%
}

% Redefine o comando de seção para rastrear títulos
\renewcommand{\section}[2][]{%
    \stepcounter{sectioncount}%
    \setcounter{subsectioncount}{0}%
    \setcounter{sectionangle}{\value{sectioncount}}%
    \multiply\value{sectionangle} by 60%
    % Armazena os dados da seção para uso posterior no mapa
    \protected@edef\autodocmap@sectionlist{%
        \autodocmap@sectionlist
        \noexpand\docmapleveltwo{sec\arabic{sectioncount}}{textual}{\thesectionangle}{#2}%
    }%
    \oldsection[#1]{#2}%
}

% Redefine o comando de subseção para rastrear títulos
\renewcommand{\subsection}[2][]{%
    \stepcounter{subsectioncount}%
    \setcounter{subsubsectioncount}{0}%
    \setcounter{subsectionangle}{\autodocmap@subsectionbaseangle}%
    \multiply\value{subsectionangle} by \value{subsectioncount}%
    % Armazena os dados da subseção para uso posterior no mapa
    \protected@edef\autodocmap@subsectionlist{%
        \autodocmap@subsectionlist
        \noexpand\docmaplevelthree{subsec\arabic{sectioncount}_\arabic{subsectioncount}}{sec\arabic{sectioncount}}{\thesubsectionangle}{#2}%
    }%
    \oldsubsection[#1]{#2}%
}

% Redefine o comando de subsubseção para rastrear títulos
\renewcommand{\subsubsection}[2][]{%
    \stepcounter{subsubsectioncount}%
    \setcounter{subsubsectionangle}{\autodocmap@subsubsectionbaseangle}%
    \multiply\value{subsubsectionangle} by \value{subsubsectioncount}%
    % Armazena os dados da subsubseção para uso posterior no mapa
    \protected@edef\autodocmap@subsubsectionlist{%
        \autodocmap@subsubsectionlist
        \noexpand\docmaplevelfour{subsubsec\arabic{sectioncount}_\arabic{subsectioncount}_\arabic{subsubsectioncount}}{subsec\arabic{sectioncount}_\arabic{subsectioncount}}{\thesubsubsectionangle}{#2}%
    }%
    \oldsubsubsection[#1]{#2}%
}

% Comando para gerar automaticamente o mapa mental com base nos dados coletados
\newcommand{\generateautodocmap}{%
    \section*{Mapa Automático da Estrutura do Documento}
    
    \begin{docstructmap}
        % Nó raiz - Documento principal
        \docmaproot{Projeto Homebrew\\LaTeX Modular}
        
        % Elementos principais (nível 1)
        \docmaplevelone{pretextual}{30}{Elementos\\Pré-textuais}
        \docmaplevelone{textual}{150}{Elementos\\Textuais}
        \docmaplevelone{postextual}{270}{Elementos\\Pós-textuais}
        
        % Nós pré-definidos para elementos comuns (nível 2)
        \docmapleveltwo{capa}{pretextual}{0}{Capa e Título}
        \docmapleveltwo{toc}{pretextual}{90}{Sumário}
        
        \docmapleveltwo{ref}{postextual}{240}{Referências\\Bibliográficas}
        \docmapleveltwo{gloss}{postextual}{300}{Glossário}
        \docmapleveltwo{ind}{postextual}{340}{Índice Remissivo}
        
        % Gerar nós para cada seção registrada (nível 2)
        \autodocmap@sectionlist
        
        % Gerar nós para cada subseção registrada (nível 3)
        \autodocmap@subsectionlist
        
        % Gerar nós para cada subsubseção registrada (nível 4)
        \autodocmap@subsubsectionlist
        
    \end{docstructmap}
    
    \docmaplegend{
        Este mapa mental foi gerado automaticamente com base na estrutura real do documento.
        Ele representa a hierarquia de capítulos, seções e subseções no formato de um mapa mental.
        As cores indicam diferentes níveis na hierarquia do documento.
    }
}

% Comando para gerar automaticamente um mapa mental focado na estrutura atual
\newcommand{\generatesectiondocmap}[1][Capítulo Atual]{%
    \section*{Mapa da Estrutura do Capítulo}
    
    \begin{docstructmap}
        % Nó raiz - Capítulo atual
        \docmaproot{#1}
        
        % Gerar nós para cada seção registrada (nível 1)
        % Esta é uma versão simplificada que mostra apenas o capítulo atual
        % Em uma implementação mais completa, seria filtrado pelo capítulo atual
        \autodocmap@sectionlist
        
        % Gerar nós para cada subseção registrada (nível 2)
        \autodocmap@subsectionlist
        
    \end{docstructmap}
    
    \docmaplegend{
        Este mapa mental mostra a estrutura específica deste capítulo, 
        destacando suas seções e subseções principais.
    }
}

% Comando para gerar um mapa mental complexo predefinido
\newcommand{\generatecomplexdocmap}{%
    \section*{Mapa da Estrutura do Documento}
    
    \begin{docstructmap}
        % Nó raiz - Documento principal
        \docmaproot{Projeto Homebrew\\LaTeX Modular}
        
        % Elementos principais (nível 1)
        \docmaplevelone{pretextual}{30}{Elementos\\Pré-textuais}
        \docmaplevelone{textual}{150}{Elementos\\Textuais}
        \docmaplevelone{postextual}{270}{Elementos\\Pós-textuais}
        
        % Elementos pré-textuais (nível 2)
        \docmapleveltwo{capa}{pretextual}{0}{Capa e Título}
        \docmapleveltwo{toc}{pretextual}{60}{Sumário}
        \docmapleveltwo{mapament}{pretextual}{120}{Mapa Mental}
        
        % Elementos textuais - Capítulos (nível 2)
        \docmapleveltwo{cap1}{textual}{120}{Capítulo 1\\Introdução}
        \docmapleveltwo{cap2}{textual}{180}{Capítulo 2\\Bibliografia}
        \docmapleveltwo{cap3}{textual}{240}{Capítulo 3\\Glossário e Índice}
        
        % Elementos pós-textuais (nível 2)
        \docmapleveltwo{ref}{postextual}{240}{Referências\\Bibliográficas}
        \docmapleveltwo{gloss}{postextual}{300}{Glossário}
        \docmapleveltwo{ind}{postextual}{340}{Índice Remissivo}
        
        % Conteúdo do Capítulo 1 (nível 3)
        \docmaplevelthree{cap1_1}{cap1}{90}{Ambientes RPG}
        \docmaplevelthree{cap1_2}{cap1}{150}{Comandos\\Personalizados}
        \docmaplevelthree{cap1_3}{cap1}{210}{Mapas Mentais}
        
        % Conteúdo do Capítulo 2 (nível 3)
        \docmaplevelthree{cap2_1}{cap2}{150}{Sistema de Citações}
        \docmaplevelthree{cap2_2}{cap2}{210}{Referências Cruzadas}
        
        % Conteúdo do Capítulo 3 (nível 3)
        \docmaplevelthree{cap3_1}{cap3}{210}{Glossário}
        \docmaplevelthree{cap3_2}{cap3}{270}{Índice Remissivo}
        
        % Exemplos de nível 4 (detalhes mais específicos)
        \docmaplevelfour{amb1}{cap1_1}{60}{Spell}
        \docmaplevelfour{amb2}{cap1_1}{120}{Character}
        \docmaplevelfour{amb3}{cap1_1}{180}{Rule}
        
        \docmaplevelfour{cmd1}{cap1_2}{120}{rpgnote}
        \docmaplevelfour{cmd2}{cap1_2}{180}{rpgsection}
        \docmaplevelfour{cmd3}{cap1_2}{240}{rpgitem}
        
        % Conteúdo do mapa mental (nível 4)
        \docmaplevelfour{map1}{cap1_3}{120}{Estrutura\\Automática}
        \docmaplevelfour{map2}{cap1_3}{180}{Mapas\\Personalizados}
        \docmaplevelfour{map3}{cap1_3}{240}{Visualização\\Hierárquica}
    
    \end{docstructmap}
    
    \docmaplegend{
        Este mapa mental apresenta a estrutura completa do documento, incluindo capítulos, 
        seções e elementos específicos. Os níveis mais externos representam as divisões 
        principais do documento, enquanto os níveis internos mostram os detalhes específicos
        de cada seção.
    }
}

% Comando para simplificar a geração rápida de um mapa mental personalizado
\newcommand{\quickdocmap}[1]{%
    \section*{Mapa da Estrutura: #1}
    
    \begin{docstructmap}
        % Nó raiz personalizado
        \docmaproot{#1}
        
        % Elementos principais predefinidos (nível 1) - podem ser adaptados conforme necessário
        \docmaplevelone{introducao}{30}{Introdução}
        \docmaplevelone{desenvolvimento}{150}{Desenvolvimento}
        \docmaplevelone{conclusao}{270}{Conclusão}
    \end{docstructmap}
    
    \docmaplegend{
        Mapa mental simplificado para "#1". Este tipo de mapa pode ser usado
        para planejar novos documentos ou visualizar conceitos específicos.
    }
}