% Elementos pré-textuais

% Capa
\begin{titlepage}
    \begin{center}
        \vspace*{2cm}
        
        \begin{tikzpicture}
            \fill[capa] (0,0) rectangle (12,4);
            \node[white, font=\Huge\bfseries] at (6,2) {Projeto Homebrew};
        \end{tikzpicture}
        
        \vspace{1cm}
        {\Large\textit{Modelo Modular LaTeX com Estética RPG}}
        
        \vfill
        
        \Large\textbf{Autor}
        
        \vspace{0.5cm}
        
        \includegraphics[width=5cm]{imgs/logo.png}
        
        \vfill
        
        \large\textbf{\today}
    \end{center}
\end{titlepage}

\cleardoublepage

% Folha de rosto
\thispagestyle{empty}
\begin{center}
    \vspace*{2cm}
    
    {\Huge\bfseries Projeto Homebrew}
    
    \vspace{2cm}
    
    {\large\textit{Documento criado como modelo para projetos LaTeX com estilo RPG}}
    
    \vfill
    
    \begin{flushright}
        \begin{minipage}{8cm}
            \singlespacing
            Modelo de documento modular, combinando\\ 
            estética RPG e organização acadêmica.\\
            Ideal para manuais de jogos, suplementos e\\
            documentos acadêmicos com visual diferenciado.
        \end{minipage}
    \end{flushright}
    
    \vfill
    
    \today
\end{center}

\cleardoublepage

% Lista de siglas
\chapter*{Lista de Abreviaturas e Siglas}
\begin{description}[style=nextline, leftmargin=1cm]
    \item[RPG] Role-Playing Game
    \item[LaTeX] Sistema de preparação de documentos
    \item[DM] Dungeon Master (Mestre do Jogo)
    \item[PDF] Portable Document Format
\end{description}

\cleardoublepage

% Sumário
\tableofcontents

% Carrega os demais elementos pré-textuais
% Conteúdo dos elementos pré-textuais adicionais

\cleardoublepage
\chapter*{Introdução ao Projeto}

Este é um modelo modular de documento LaTeX, combinando a estética visual de livros de RPG com a organização estrutural de documentos acadêmicos. O modelo oferece uma estrutura organizada em pastas e arquivos, permitindo a edição isolada de cada componente.

\begin{quotebox}
A combinação de elementos visuais inspirados em RPG com a estrutura acadêmica cria um documento único, que pode ser utilizado para manuais de jogos, suplementos, ou até mesmo teses e dissertações com uma estética diferenciada.
\end{quotebox}

\cleardoublepage
\chapter*{Sobre este Modelo}

\begin{dmnote}
Este é um modelo base que pode e deve ser expandido e personalizado conforme as necessidades do usuário. Os arquivos estão organizados de forma modular para facilitar a manutenção e expandibilidade.
\end{dmnote}

\begin{highlight}
As partes do documento estão separadas em arquivos distintos, o que facilita a edição e manutenção. Cada capítulo está em um arquivo separado, assim como as seções pré-textuais, textuais e pós-textuais.
\end{highlight}

