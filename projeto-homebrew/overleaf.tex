% Configurações específicas para compatibilidade com Overleaf
% Este arquivo é automaticamente carregado pelo main.tex
% As configurações básicas já estão ativas; as avançadas podem ser ativadas se necessário

%%%%%%%%%%%%%%%%%%%%%%%%%%%%%%%%%%%%%%%%%%%%%%%%%%%%%%%%%%%%%%%
% CONFIGURAÇÕES AUTOMÁTICAS DO OVERLEAF
%%%%%%%%%%%%%%%%%%%%%%%%%%%%%%%%%%%%%%%%%%%%%%%%%%%%%%%%%%%%%%%
%
% Este arquivo aplica automaticamente ajustes básicos de compatibilidade
% para o Overleaf. Configurações adicionais podem ser ativadas 
% descomentando as linhas apropriadas abaixo.
%
% As configurações básicas incluem:
% - Otimizações de memória leves
% - Ajustes para compilação mais rápida
% - Compatibilidade com fontes padrão
%
%%%%%%%%%%%%%%%%%%%%%%%%%%%%%%%%%%%%%%%%%%%%%%%%%%%%%%%%%%%%%%%

% CONFIGURAÇÃO BÁSICA (JÁ ATIVA)
\usepackage{etoolbox}
\apptocmd{\tableofcontents}{\newpage}{}{}
\tikzset{every shadow/.style={opacity=0.7}}  % Reduz a opacidade das sombras para melhor desempenho

% Configuração para detecção automática de ambiente Overleaf
\newif\ifoverleaf
\overleaftrue  % Assume que estamos no Overleaf

%%%%%%%%%%%%%%%%%%%%%%%%%%%%%%%%%%%%%%%%%%%%%%%%%%%%%%%%%%%%%%%
% SEÇÃO 1: AJUSTES DE FONTES
%%%%%%%%%%%%%%%%%%%%%%%%%%%%%%%%%%%%%%%%%%%%%%%%%%%%%%%%%%%%%%%

% Overleaf suporta estas fontes que são compatíveis com XeLaTeX
% Descomente apenas UMA das linhas abaixo se necessário

% \setmainfont{Times New Roman}
% \setmainfont{TeX Gyre Termes}
% \setmainfont{TeX Gyre Pagella}
% \setmainfont{Latin Modern Roman}

%%%%%%%%%%%%%%%%%%%%%%%%%%%%%%%%%%%%%%%%%%%%%%%%%%%%%%%%%%%%%%%
% SEÇÃO 2: SIMPLIFICAÇÃO DE MAPAS MENTAIS
%%%%%%%%%%%%%%%%%%%%%%%%%%%%%%%%%%%%%%%%%%%%%%%%%%%%%%%%%%%%%%%

% Para problemas com mapas mentais muito complexos, descomente as linhas abaixo:

% Opção 1: Desativar completamente os mapas mentais automáticos
% \renewcommand{\generatecomplexdocmap}{\textit{(Mapa mental desativado para compatibilidade com Overleaf)}}
% \renewcommand{\generateautodocmap}{\textit{(Mapa mental desativado para compatibilidade com Overleaf)}}

% Opção 2: Usar versão simplificada dos mapas mentais
% \def\usesimplifiedmindmap{1}

% CORREÇÕES PARA PROBLEMAS DE FORMATAÇÃO EM XELATEX 2023+

% 1. CORREÇÃO PARA TEXTO QUEBRADO (variáveis e comandos expostos)
% Corrige o problema de exibição de comandos internos como:
% 'sectionbaseangle120 subsectionbaseangle30' nos documentos
\makeatletter
\AtBeginDocument{
  % Esconder variáveis internas do sistema
  \let\oltextual\@textual
  \renewcommand{\@textual}[1]{\oltextual{#1}}
  
  % Redefine os comandos base de ângulos para não serem impressos no documento
  \newcommand{\fixangles}{%
    \renewcommand{\autodocmap@sectionbaseangle}{120}%
    \renewcommand{\autodocmap@subsectionbaseangle}{30}%
    \renewcommand{\autodocmap@subsubsectionbaseangle}{45}%
  }
  \fixangles
}
\makeatother

% 2. CORREÇÃO PARA MAPAS MENTAIS QUEBRADOS (texto dentro dos nós)
\usepackage{varwidth}
\tikzset{
  % Estilo completamente novo para os nós dos mapas
  better concept/.style={
    concept,
    text width=3.5cm,
    align=center,
    font=\small,
    execute at begin node={\begin{varwidth}{3.2cm}\centering},
    execute at end node={\end{varwidth}}
  }
}

% Substitui o estilo padrão pelo corrigido
\tikzset{
  concept/.append style={better concept}
}

% 3. CORREÇÃO PARA PROBLEMAS DE TEXTO/ESPAÇOS
% Elimina alguns comandos problemáticos que podem aparecer no texto
\makeatletter
\AtBeginDocument{
  % Limpa tokens problemáticos
  \let\@defctionlist\@empty
  \let\sec1textual00\@empty
  
  % Fixes para elementos de cabeçalho/rodapé
  \renewcommand{\headrulewidth}{0.4pt}
  \renewcommand{\footrulewidth}{0.4pt}
}
\makeatother

% 4. CORREÇÃO DE QUEBRAS DE PÁGINA E TEXTO
\interfootnotelinepenalty=10000  % Evita quebras de nota de rodapé
\widowpenalty=10000              % Evita linhas órfãs
\clubpenalty=10000               % Evita linhas viúvas

% 5. CORREÇÃO DE FORMATAÇÃO DE TABELAS E LISTAS
\renewcommand{\arraystretch}{1.2}  % Melhora espaçamento em tabelas

% 6. CORREÇÃO ESPECÍFICA PARA VARIAVEIS EXPOSTAS EM MAPAS
% Esta macro substitui as variáveis expostas por espaços
\usepackage{etoolbox}
\makeatletter
\AtBeginDocument{
  \def\fix@exposed@vars#1{%
    \ifstrequal{#1}{@defctionlistctionlist}{}{\ifstrequal{#1}{ctionbaseangle}{}{\ifstrequal{#1}{sec1textual}{}{\ifstrequal{#1}{subsec}{}{#1}}}}%
  }
  % Aplica a substituição em ambientes específicos
  \pretocmd{\docstructmap}{\let\orig@textual\@textual\def\@textual##1{\orig@textual{\fix@exposed@vars{##1}}}}{}{}
}
\makeatother

%%%%%%%%%%%%%%%%%%%%%%%%%%%%%%%%%%%%%%%%%%%%%%%%%%%%%%%%%%%%%%%
% SEÇÃO 3: OTIMIZAÇÕES DE MEMÓRIA
%%%%%%%%%%%%%%%%%%%%%%%%%%%%%%%%%%%%%%%%%%%%%%%%%%%%%%%%%%%%%%%

% Para problemas de memória no Overleaf:

% Reduzir a complexidade de elementos decorativos
% \tikzset{every shadow/.style={opacity=0}}
% \tikzset{concept connection/.append style={line width=0.5pt}}

% Reduzir resolução de cores
% \tikzset{every tcolourbox/.append style={color=black, natural}}

% Adicionar quebras de página extras para evitar seções grandes
% \usepackage{etoolbox}
% \apptocmd{\tableofcontents}{\newpage}{}{}

%%%%%%%%%%%%%%%%%%%%%%%%%%%%%%%%%%%%%%%%%%%%%%%%%%%%%%%%%%%%%%%
% FIM DAS CONFIGURAÇÕES
%%%%%%%%%%%%%%%%%%%%%%%%%%%%%%%%%%%%%%%%%%%%%%%%%%%%%%%%%%%%%%%
